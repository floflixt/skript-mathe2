\documentclass[12pt, titlepage]{article}

\usepackage[ngerman]{babel}
\usepackage[utf8]{inputenc}
\usepackage{color}
\usepackage{amssymb}
\usepackage{amsmath}
\usepackage{dsfont}
\usepackage{mathtools}
\usepackage[colorlinks=true, linkcolor=black, citecolor=black filecolor=green, urlcolor=blue]{hyperref}
\usepackage[a4paper]{geometry}
\usepackage{tikz}

\begin{titlepage}
	\title{Mathematik II}
	\date{13.05.2016}
\end{titlepage}

\newcommand{\R}{\mathds{R}}
\newcommand{\N}{\mathds{N}}
\newcommand{\e}{\textrm{e}}
\newcommand{\infn}{n\rightarrow\infty}
\newcommand{\bmark}[1]{\textcolor{blue}{#1}}
\newcommand{\gmark}[1]{\textcolor{lightgray}{#1}}
\newcommand{\limsuperior}[1]{\lim\limits_{#1}\sup}
\newcommand{\liminferior}[1]{\lim\limits_{#1}\inf}

\renewcommand{\*}{\cdot}
\renewcommand{\epsilon}{\varepsilon}
\renewcommand{\limsup}[1]{\underset{#1}{\overline{\lim}}}
\renewcommand{\liminf}[1]{\underset{#1}{\underline{\lim}}}

\begin{document}
	\maketitle
	\tableofcontents
	\newpage
	\section{Reelle Funktionen}
	\subsection{Wiederholung Mathe 1: Funktionen}
	\subsubsection*{Definition}
		Eine \underline{Funktion/Abbildung} $f\colon A\rightarrow B$ besteht aus
		\begin{itemize}
			\item zwei Mengen:
			\begin{itemize}
				\item $A$: \underline{Definitionsbereich} von $f$
				\item $B$: \underline{Bildbereich} von $f$
			\end{itemize}
			\item und einer \underline{Zuordnungsvorschrift}, die jedem Element $a\in A$ genau ein Element $b\in B$ zuordnet.
		\end{itemize}
		Wir schreiben dann $b=f(a)$, nennen $b$ das \underline{Bild}/den \underline{Funktionswert} von $a$ (unter $f$) sowie $a$ (ein) \underline{Urbild} von $b$ (unter $f$).
	\subsubsection*{Notation}
		\vspace{-1cm}\begin{align*}
			f\colon A&\rightarrow B\\
			a&\mapsto f(a)
		\end{align*}
	\subsubsection*{Beispiel}
		$\rightarrow$ Folien 11.04.2016
	\subsection{Reelle Funktionen}
	\subsubsection*{Definition}
	Eine \underline{reelle Funktion} einer \underline{Veränderlichen} ist eine Abbildung $f\colon D\rightarrow \R$, wobei $D\subseteq \R$ (oft ist $D$ endliche Vereinigung von Intervallen, z.B.
	\begin{itemize}
		\item $ D=(-\infty,a]=\{x\in \R|x\leq a\} $
		\item $ D=\R^+_0=[0,\infty)=\{x\in\R|x\geq 0\}$
		\item $ D=(-\infty,\infty)=\R $
		\item $ D=\R\setminus\{0\}=(-\infty,0)\cup(0,\infty) $\hfill).
	\end{itemize}
	\subsection{Neue Funktionen aus Alten, Kompositionen}
	\subsubsection*{Definition}
	Seien $f,g\colon D\rightarrow \R$ reelle Funktionen.
	\begin{itemize}
		\item[a)] $(f\pm g)(x)\coloneqq f(x)\pm g(x)\qquad \forall x\in D$\\
				  \underline{Summe/Differenz} von $f$ und $g$\\
				  (genauer:\vspace{-0.5cm} \begin{align*}
				  	f\pm g\colon D&\rightarrow \R\\
				  	x&\mapsto (f\pm g)(x)=f(x)\pm g(x) \textrm{\qquad )}\end{align*}
		\item[b)] $(f\* g)(x)\coloneqq f(x)\* g(x)\qquad \forall x\in D$\\
		\underline{Produkt} von $f$ und $g$
		\item[c)] falls $g(x)\neq 0\quad \forall x\in D$, dann\\
		$(\frac{f}{g})(x)\coloneqq\frac{f(x)}{g(x)}\qquad \forall x\in D$\\
		\underline{Quotient} von $f$ und $g$
		\item[d)] \underline{Komposition/Hintereinanderausführung}\\
		$f\colon D_f\rightarrow \R,\quad g\colon D_g\rightarrow\R\textrm{, wobei } f(D_f)\subseteq D_g$\\
		\vspace{-0.5cm}\begin{align*}
			g\circ f\colon D_f&\rightarrow\R\\
			x&\mapsto g(f(x))
		\end{align*}
	\end{itemize}
	\subsection{Beispiel}
	\begin{align*}
		f,g&\colon \R\rightarrow\R\\
		f(x)&=x^2\\
		g(x)&=x-1\\
		\\
		(f+g)(x)&=x^2+x-1\\
		(f\* g)(x)&=x^2\* (x-1)=x^3-x^2\\
		(\frac{f}{g})(x)&=\frac{x^2}{x-1}\quad\textrm{für }x\neq 1\quad (D_g=\R\setminus \{1\})\\
		(g\circ f)(x)&=g(f(x))=g(x^2)=x^2-1\\
		(f\circ g)(x)&=f(g(x))=f(x-1)=(x-1)^2=x^2-2x+1\\
		\\
		\Rightarrow (g\circ f)(x)&\neq (f\circ g)(x)
	\end{align*}
	\subsection{Wiederholung Mathe 1: Injektivität, Surjektivität, Bijektivität; Umkehrfunktion}
	$\rightarrow$ Folien 13.04.2016	
	\subsection{Elementare Funktionen (naive Einführung)}
	\begin{itemize}
		\item[a)] \underline{Konstante Funktionen}\\
		für $c\in\R$ (fest):\\
		\begin{minipage}[c]{0.5\textwidth}
			\begin{align*}
				f\colon \R&\rightarrow\R\\
				x&\mapsto c
			\end{align*}
		\end{minipage}
		\begin{minipage}[c]{0.5\textwidth}
			\begin{tikzpicture}[scale=0.5,domain=0:4]
				\draw[->]
					(-0.2,0) -- (4.2,0) node[right] {$x$};
				\draw[->]
					(0,-0.2) -- (0,4.2) node[above] {$f(x)$};
				\draw[color=blue]
					plot(\x,1.5) node[right]{$f(x)=c$};
			\end{tikzpicture}
		\end{minipage}
		\item[b)] \underline{Die identische Funktion (Identität)}\\
		\begin{minipage}[c]{0.5\textwidth}
			\begin{align*}
				f\colon \R&\rightarrow \R\\
				x&\mapsto x
			\end{align*}
		\end{minipage}
		\begin{minipage}[c]{0.5\textwidth}
			\begin{tikzpicture}[scale=0.5,domain=0:4]
			\draw[->]
				(-0.2,0) -- (4.2,0) node[right] {$x$};
			\draw[->]
				(0,-0.2) -- (0,4.2) node[above] {$f(x)$};
			\draw[color=blue]
				plot(\x,\x) node[right]{$f(x)=x$};
			\end{tikzpicture}
		\end{minipage}
	\end{itemize}
	Durch mehrfache Anwendung von 1.3 entstehen aus a) und b) viele weitere Funktionen.
	\begin{itemize}
		\item[c)] \underline{Potenzen (Monome)}\\
		für $n\in\N_0$ (fest):
		\begin{align*}
			f\colon \R&\rightarrow\R\\
			x&\mapsto x^n
		\end{align*}
		\begin{itemize}
			\item $n=0$: die konstante 1-Funktion
			\begin{align*}
				f\colon \R&\rightarrow\R\\
				x&\mapsto x^0=1
			\end{align*}\\
			\item $n$ ungerade:\\
			\begin{minipage}[c]{0.5\textwidth}
				$f$ punktsymmetrisch zum Ursprung $(0|0)$, bijektiv
			\end{minipage}
			\begin{minipage}[c]{0.5\textwidth}
				\begin{tikzpicture}[scale=0.5]
				\draw[->]
				(-4.2,0) -- (4.2,0) node[right] {$x$};
				\draw[->]
				(0,-4.2) -- (0,4.2) node[above] {$f(x)$};
				\draw[color=blue, domain=-2:2]
				plot(\x,\x*\x*\x*0.5) node[right]{$f(x)=x^n$};
				\end{tikzpicture}
			\end{minipage}
			\item $n$ gerade:\\
			\begin{minipage}[c]{0.5\textwidth}
				$f$ achsensymmetrisch zur $y$-Achse, nicht bijektiv\\ $f(x)\geq 0\quad\forall x\in\R$
			\end{minipage}
			\begin{minipage}[c]{0.5\textwidth}
				\begin{tikzpicture}[scale=0.5]
				\draw[->]
				(-4.2,0) -- (4.2,0) node[right] {$x$};
				\draw[->]
				(0,-0.2) -- (0,4.2) node[above] {$f(x)$};
				\draw[color=blue, domain=-3:3]
				plot(\x,\x*\x*0.5) node[right]{$f(x)=x^n$};
				\end{tikzpicture}
			\end{minipage}
		\end{itemize}
		\item[d)] \underline{Wurzelfunktionen}\\
		Wurzelfunktionen sind die Umkehrfunktionen der Monome. Dazu musss die Gleichung $f(x)=x^n=y$ ($y\in\R$ gegeben) gelöst werden.
		\begin{itemize}
			\item $n$ ungerade:\\
			\\
			\begin{minipage}[c]{0.5\textwidth}
				$f$ ist bijektiv, dann gibt es zu jedem \\
				$y\in\R$ genau ein $x\in\R$ mit $x^n=y$. Dieses wird die $n$-te Wurzel aus $y$ genannt:\\ $x=\sqrt[n]{y}$.\\
				\vspace{-0.5cm}
				\begin{align*}
					\sqrt[n]{\quad}\colon \R&\rightarrow\R\\
					x&\mapsto\sqrt[n]{x}
				\end{align*}
			\end{minipage}
			\begin{minipage}[c]{0.5\textwidth}
				\begin{tikzpicture}[scale=0.5]
				\draw[->]
				(-4.2,0) -- (4.2,0) node[right] {$x$};
				\draw[->]
				(0,-4.2) -- (0,4.2) node[above] {$f(x)$};
				\draw[color=gray, domain=-2:2,very thin]
				plot(\x,\x*\x*\x*0.5) node[right]{$f(x)=x^n$};
				\draw[color=blue,domain=-1.7:1.7]
				plot(\x*\x*\x,\x) node[right]{$f(x)=\sqrt[n]{x}$};
				\end{tikzpicture}
			\end{minipage}
			\item $n$ gerade: Dann hat die Gleichung $x^n=y$ in $\R$
			\begin{itemize}
				\item keine Lösung, falls $y<0$
				\item genau eine Lösung, falls $y=0$ (nämlich $x=0$)
				\item zwei Lösungen, falls $y>0$:
				\begin{align*}
					x_1&=\sqrt[n]{y}\quad(>0)\\
					x_2&=-\sqrt[n]{y}\quad(<0)
				\end{align*}
				\begin{minipage}[c]{0.5\textwidth}
					Die \underline{positive} Lösung wird hier dann als $n$-te Wurzel bezeichnet:
					\begin{align*}
					\sqrt[n]{\quad}\colon\R_0^+&\rightarrow\R_0^+\\
					x&\mapsto\sqrt[n]{x}
					\end{align*}
				\end{minipage}
				\begin{minipage}[c]{0.5\textwidth}
					\begin{tikzpicture}[scale=0.5]
					\draw[->]
					(-4.2,0) -- (4.2,0) node[right] {$x$};
					\draw[->]
					(0,-4.2) -- (0,4.2) node[above] {$f(x)$};
					\draw[color=gray, domain=-3:3,very thin]
					plot(\x,\x*\x*0.5) node[right]{$f(x)=x^n$};
					\draw[color=blue,domain=0:3]
					plot(\x*\x*0.5,\x) node[right]{$f(x)=\sqrt[n]{x}$};
					\draw[color=gray,domain=-3:3,dotted]
					plot(\x*\x*0.5,-\x);
					\end{tikzpicture}
				\end{minipage}
			\end{itemize}
		\end{itemize}
		\item[e)] \underline{Polynome}\\
		$a_0,\quad...\quad,a_n\in\R$ (Koeffizienten)\\
		Ein Polynom ist eine Funktion $p$ mit 
		\begin{align*}
			p\colon \R&\rightarrow\R\\
			x&\mapsto a_nx^n+a_{n-1}x^{n-1}+...+a_1x^1+a_0x^0=\sum_{k=0}^{n}{a_kx^k}
		\end{align*}
		Falls $a_n\neq 0$ ist, heißt $n$ \underline{Grad} des Polynoms.
		\item[f)] \underline{Rationale Funktionen}\\
		Rationale Funktionen sind Quotienten von Polynomen (mit $p,q$...Polynome):
		\begin{align*}
			f\colon D&\rightarrow\R\\
			x&\mapsto\frac{p(x)}{q(x)}
		\end{align*}
		mit $D=\{x\in\R|q(x)\neq 0\}$
		\item[g)] \underline{Exponentialfunktionen}\\
		Exponentialfunktionen sind Funktionen
		\begin{align*}
			f\colon\R&\rightarrow\R^+\\
			x&\mapsto q^x
		\end{align*} wobei die Basis $\R\ni q>0$, $q\neq 1$ vorgegeben ist.
		\begin{align*}
			q>1&\colon\textrm{$f$ steigt}\\
			0<q<1&\colon\textrm{$f$ fällt}
		\end{align*}
		Bekannte Rechenregeln:
		\begin{itemize}
			\item $q^x\* q^y=q^{x+y}$
			\item $\frac{q^x}{q^y}=q^{x-y}$
			\item $(q^x)^y=q^{x\* y}$
			\item $(p\* q)^x=p^x\* q^x$
			\item $(\frac{p}{q})^x=\frac{p^x}{q^x}$
		\end{itemize}
		Zur Beschreibung von Exponentialfunktionen genügt es, \underline{eine} bestimmte Basis zu benutzen (man kann $g(x)=p^x$ durch $f(x)=q^x$ ausdrücken, siehe Teil h).\\
		Früher: Basis 10\\
		Heute: Basis $\e\approx 2.781828...$ (Eulersche Zahl)\\
		Informatik: oft Basis 2\\
		\begin{minipage}[c]{0.5\textwidth}
			\begin{align*}
			\textrm{exp}\colon\R&\rightarrow\R^+\\
			x&\mapsto \e^x\\
			\\
			\textrm{exp}(0)&=\e^0=1\\
			\textrm{exp}(1)&=\e^1=2.781828...
			\end{align*}
		\end{minipage}
		\begin{minipage}[c]{0.5\textwidth}
			\begin{tikzpicture}[scale=0.5]
			\draw[->]
				(-4.2,0) -- (4.2,0) node[right] {$x$};
			\draw[->]
			 	(0,-0.2) -- (0,4.2) node[above] {$f(x)$};
			\draw[color=blue,domain=-3:1.5]
				plot(\x,{exp(\x)}) node[right]{$f(x)=\textrm{exp}(x)$};;
			\end{tikzpicture}
		\end{minipage}
		\item[h)] \underline{Logarithmen}\\
		Die Exponentialfunktion
		\begin{align*}
			\textrm{exp}(x)\colon\R&\rightarrow\R^+\\
			x&\mapsto \e^x
		\end{align*} ist bijektiv.\\
		Um sie umzukehren, muss zu gegebenem $y\in\R^+$ die Gleichung $\e^x=y$ gelöst werden.\\
		Die Lösung ist für $y>0$ in $\R$ eindeutig und wird als der \underline{natürliche Logarithmus} von $y$ bezeichnet: $x=\ln y$.\\
		In $\R$ ist die Gleichung für $y\leq 0$ unlösbar.\\
		\begin{minipage}[c]{0.5\textwidth}
			\begin{align*}
			\ln\colon\R^+&\rightarrow\R\\
			x&\mapsto\ln x
			\end{align*}
		\end{minipage}
		\begin{minipage}[c]{0.5\textwidth}
			\begin{tikzpicture}[scale=0.5]
			\draw[->]
			(-4.2,0) -- (4.2,0) node[right] {$x$};
			\draw[->]
			(0,-4.2) -- (0,4.2) node[above] {$f(x)$};
			\draw[color=gray, domain=-3:1.5,very thin]
			plot(\x,{exp(\x)}) node[right]{$f(x)=\textrm{exp}(x)$};
			\draw[color=blue,domain=-3:1.5]
			plot({exp(\x)},\x) node[right]{$f(x)=\ln x$};;
			\end{tikzpicture}
		\end{minipage}
		Analoges gilt für andere Exponentialfunktionen.
		\begin{align*}
			f:\R&\rightarrow\R^+\\
			x&\mapsto q^x\quad (q>0,q\neq 1)
		\end{align*}
		Es gilt: $q^x=y\Leftrightarrow x=\log_qy$ (Logarithmus zur Basis $q$).\\
		\\
		Es genügt wieder, \underline{eine} feste Basis zu betrachten, z.B. $\e$, denn $q^x=(\e^{\ln q})^x=\e^{x\*\ln q}$. Es gilt:
		\begin{align*}
			q^x=y&\Leftrightarrow\e^{x\*\ln q}=y\\
			&\Leftrightarrow\ln(\e^{x\*\ln q})=\ln y\\
			&\Leftrightarrow x\*\ln q=\ln y\\
			&\Leftrightarrow x=\frac{\ln y}{\ln q}\quad\textrm{,}
		\end{align*}
	also gilt $\log_qy=\frac{\ln y}{\ln q}$.\\
	\\
	\underline{Rechenregeln} für den Logarihmus lassen sich aus den Regeln für die Exponentialfunktion herleiten:\\
	\\
	Sei $u\coloneqq\ln x$, $v\coloneqq\ln y$, dann ist $x=\e^u$ und $y=\e^v$, daraus folgt 
	\begin{align*}
	x\* y=\e^u\*\e^v=\e^{u+v}\quad\textrm{,}
	\end{align*} also ist 
	\begin{align*}
	\ln(x\* y)=\ln(\e^{u+v})=u+v=\ln x+\ln y\quad\textrm{.}
	\end{align*}
	Genauso kann man mit beliebiger Basis $q>0$, $q\neq1$ verfahren, wir erhalten für jede Logarithmusfunktion $\log\colon\R^+\rightarrow\R$:
	\begin{itemize}
		\item $\log(x\*y)=\log x+\log y\quad\forall x,y>0$
		\item $\log(\frac{x}{y})=\log x-\log y\quad\forall x,y>0$
		\item $\log(x^\alpha)=\alpha\*\log x\quad\forall x>0,\alpha\in\R$
	\end{itemize}
	\item[i)] \underline{Trigonometrische Funktionen}\\
	Wir betrachten einen Punkt $P$ auf dem Einheitskreis (Kreis um $O$, Radius 1).\\
	\\
	\begin{minipage}[c]{0.5\textwidth}
		Der Winkel, der von der positiven $x_1$-Achse und der Geraden durch $O$ und $P$ eingeschlossen wird, sei $x$.
	\end{minipage}
	\begin{minipage}[c]{0.5\textwidth}
		\begin{tikzpicture}[scale=0.5]
		\draw[->]
		(-4.2,0) -- (4.2,0) node[right] {$x_1$};
		\draw[->]
		(0,-4.2) -- (0,4.2) node[above] {$x_2$};
		\draw (0,0) circle (100pt);
		\draw[color=gray, domain=-0.5:3,very thin]
		plot(\x,\x) node[right]{};
		\draw[color=blue, domain=-1:2.45, style=dotted]
		plot(2.45,\x);
		\draw[color=blue, domain=0:2.45, line width=2pt]
		plot(\x,-1) node[below left]{$\cos x$};
		\draw[color=red, domain=-1:2.45, style=dotted]
		plot(\x,2.45);
		\draw[color=red, domain=0:2.45, line width=2pt]
		plot(-1,\x) node[below left]{$\sin x$};
		\draw[fill=black](2.45,2.45) circle(3pt) node[above right]{$P$};
		\draw[fill=black](0,0) circle(3pt) node[below left]{$O$};
		\draw[fill=black](3.5,0) circle(3pt) node[below right]{1};
		\draw[very thin, color=gray, ->]
		(2, 0) node[above left]{$x$} arc (0:45:2);
		\end{tikzpicture}
	\end{minipage}\\
	Dann heißt die $x_1$-Koordinate von $P$ der \underline{Kosinus} von $x$ ($\cos x$), die $x_2$-Koordinate heißt der \underline{Sinus} von $x$ ($\sin x$).\\
	\\
	Der Winkel $x$ kann im Gradmaß oder im Bogenmaß (Länge des Bogens von $(1|0)$ bis $P$) gemessen werden, es gilt:
	\begin{align*}
		\frac{\textrm{Gradmaß}}{360^\circ}=\frac{\textrm{Bogenmaß}}{2\pi}
	\end{align*}
	So lassen sich die Funktionen $\cos$ und $\sin$ definieren:
	\begin{align*}
		\cos\colon\R&\rightarrow[-1;1]\\
		x&\mapsto\cos x\\
		\\
		\sin\colon\R&\rightarrow[-1;1]\\
		x&\mapsto\sin x
	\end{align*}
	und weiter
	\begin{align*}
		\tan x&\coloneqq\frac{\sin x}{\cos x}\qquad\textrm{(Tangens)}\qquad
		\textrm{und}\\
		\\
		\cot x&\coloneqq\frac{\cos x}{\sin x}\qquad\textrm{(Kotangens)}
	\end{align*}
	(Tangens und Kotangens sind jeweils nur dort definiert, wo der Nenner $\neq 0$ ist!)\\
	\begin{minipage}[c]{0.5\textwidth}
		\begin{tikzpicture}
			\draw[->]
			(-0.2,0) -- (4.2,0) node[right] {$x_1$};
			\draw[->]
			(0,-0.2) -- (0,4.2) node[above] {$x_2$};
			\draw
			(4,0) arc(0:90:4);
			\draw[domain=-0.1:4.5, very thin]
			plot(\x,0.75*\x);
			\draw[color=gray, very thin, ->]
			(2,0) node[above left]{$x$} arc(0:37:2);
			\draw[fill=black](3.2,2.4) circle(1.5pt);
			\draw[color=gray, very thin, domain=0:4, <->]
			(-1.5,0) -- (-1.5,2) node[left]{1} -- (-1.5,4);
			\draw[color=gray, very thin, domain=0:4, <->]
			(0,-1) -- (2,-1) node[below]{1} -- (4,-1);		
			\draw[style=dotted]
			(4,-0.5) -- (4,4);
			\draw[color=green, line width=2pt]
			(4,0) -- (4,1.5) node[right]{$\tan x$} -- (4,3);
			\draw[color=blue]
			(0,-0.3) -- (1.6,-0.3) node[below]{$\cos x$} -- (3.2,-0.3);
			\draw[color=red]
			(-0.3,0) -- (-0.3,1) node[left]{$\sin x$} -- (-0.3,2.4);
		\end{tikzpicture}
	\end{minipage}
	\begin{minipage}[c]{0.5\textwidth}
		Strahlensatz: $\frac{\sin x}{\cos x}=\frac{\tan x}{1}$\\
		\\
		Wertetabelle: s. PÜ 02
	\end{minipage}
	Graphen:\\
	\begin{tikzpicture}
		\draw[->]
		(-6,0) -- (7,0) node[right] {$x$};
		\draw[->]
		(0,-2) -- (0,2) node[above] {$f(x)$};
		\draw
		(0,1) node[left]{$1$}
		(0,-1) node[left]{$-1$}
		(-6.283,0) node[below]{$-2\pi$}
		(-4.65,0) node[below]{$-\frac{3}{2}\pi$}
		(-3.1415,0) node[below]{$-\pi$}
		(-1.57,0) node[below]{$-\frac{1}{2}\pi$}
		(1.57,0) node[below]{$\frac{1}{2}\pi$}
		(3.1415,0) node[below]{$\pi$}
		(4.65,0) node[below]{$\frac{3}{2}\pi$}
		(6.283,0) node[below]{$2\pi$};
		\draw[color=red, domain=-6:6, samples=50]
		plot(\x,{sin(\x r)}) node[below left]{$\sin x$};
		\draw[color=blue, domain=-6:6, samples=50]
		plot(\x,{cos(\x r)}) node[above right]{$\cos x$};
		\draw[color=green, domain=-1.2:1.2]
		plot(\x,{tan(\x r)});
		\draw[style=dotted, color=green]
		(-1.57,-3)--(-1.57,3);
		\draw[color=green, domain=1.95:4.35]
		plot(\x,{tan(\x r)}) node[right]{$\tan x$};
		\draw[style=dotted, color=green]
		(1.57,-3)--(1.57,3);
		\draw[color=green, domain=-4.35:-1.95]
		plot(\x,{tan(\x r)});
	\end{tikzpicture}\\
	Additionstheoreme:
	\begin{align*}
		\sin (x+y)&=\sin x\*\cos y+\cos x\* \sin y\\
		\cos(x+y)&=\cos x\*\cos y-\sin x\*\sin y\\
		(\sin x)^2+(\cos x)^2&=\sin^2x+\cos^2x=1\qquad\textrm{(Satz des Pythagoras)}
	\end{align*}
	\\
	Es gilt: $\cos x=\sin (x+\frac{\pi}{2})$ (Verschiebung um $\frac{\pi}{2}$).\\
	\\
	$\sin$ und $\cos$ sind $2\pi$-periodisch, d.h.
	\begin{align*}
		\sin x&=\sin(x+2\pi)\qquad\forall x\\
		\cos x&=\cos (x+2\pi)\qquad\forall x
	\end{align*}
	$\tan$ ist $\pi$-periodisch:
	\begin{align*}
		\tan x&=\tan(x+\pi)\qquad\forall x\textrm{ auf Definitionsbereich}
	\end{align*}
	\end{itemize}
	\newpage
	\section{Folgen}
	\subsection{Definition: Folge}
	\subsubsection*{Definition}
	Eine \underline{Folge} $(a_n)_{n\in\N}$ ist eine Abbildung von der Menge der natürlichen Zahlen $\N$ in eine Menge $M$ (oft $M\subset \R$).\\
	Die $a_n$ ($n=1,2,3,...$) heißen \underline{Glieder} der Folge, $n$ heißt \underline{Index}.\\
	(Bemerkung: Das 1. Glied der Folge muss nicht $a_1$ sein. durch Umbenennung, z.B. $b_1\coloneqq a_7, b_2\coloneqq a_8$, ist auch $(a_7, a_8, a_9, ...)$ eine Folge im sinne der Definition 2.1)
	\subsubsection*{Schreibweisen}
	\begin{align*}
		&(a_n)_{n\in\N}\\
		&(a_n)_{n\geq n_0}\qquad \textrm{(z.B. $(a_n)_{n\geq 7}$) oder nur}\\
		&(a_n)
	\end{align*}
	\subsection{Beispiel}
	\begin{itemize}
		\item[a)] $a_n=c\qquad\forall n\geq 1, c\in\R\textrm{ konstant}$\\
		$(a_n)_{n\in\N}=(c)_n\qquad (c,c,c,c,...)$
		\item[b)] $a_n=n\qquad (1, 2, 3,4,...)$
		\item[c)] $a_n=(-1)^n\qquad (-1,1,-1,1,-1,...)$
		\item[d)] $a_n=\frac{1}{n}\qquad(1,\frac{1}{2},\frac{1}{3},\frac{1}{4},...)$
		\item[e)] $a_n=[0,\frac{2}{n})\qquad$ Folge von Intervallen
		\item[f)] $a_n$ rekursiv definiert:\\
		\begin{align*}
			a_1&\coloneqq 1\\
			a_{n+1}&\coloneqq(n+1)a_n\qquad(n\geq 1)\\
			a_2&=2\*a_1=2\\
			a_3&=3\*a_2=6\\
			a_4&=4\*a_3=24
		\end{align*}
	\end{itemize}
	\subsection{Definition: Eigenschaften von Folgen}
	Eine Folge $(a_n)_{n\in\N}$ reeller Zahlen heißt
	\begin{itemize}
		\item[a)] \underline{beschränkt}, wenn die Menge der Folgenglieder beschränkt ist (s. Mathe 1), d.h. wenn es eine Zahl $K\geq 0$ gibt mit $|a_n|\leq K\quad\forall n\in\N$ (d.h. alle Folgenglieder liegen im Intervall $[-K,K]\quad\forall n;\quad(-K\leq a_n\leq K)$).
		\item[b)] \underline{alternierend}, falls ihre Glieder abwechselnd positiv und negativ sind.
	\end{itemize}
	\subsection{Beispiel}
	Beispiele aus 2.2:\\
	beschränkt: a), c), d) [für c) und d) z.B. K=1] \\
	alternierend: c)
	\subsection{Definition: Konvergenz}
	\begin{itemize}
		\item[a)] Eine Folge $(a_n)_{n\in\N}$ reeller Zahlen heißt \underline{konvergent gegen $a\in\R$}, wenn es zu jeder positiven Zahl $\epsilon>0$ ein $N\in\N$ gibt (das von $\epsilon$ abhängen darf), so dass gilt: $|a_n-a|<\epsilon$ für alle $n\geq N$.\\
		(kurz: $\forall\epsilon>0\quad\exists N\in\N\quad\forall n\geq N\colon|a_n-a|<\epsilon$)
		\item[b)] Die Zahl a heißt dann \underline{Grenzwert} oder \underline{Limes} der Folge, wir schreiben:\\
		 $\lim\limits_{\infn}a_n=a$ oder\\
		 $a_n\rightarrow a$ für $\infn$ ($a_n$ strebt gegen $a$)
		 \item[c)] Eine Folge, die gegen $0$ konvergiert, heißt \underline{Nullfolge}.
		 \item[d)] Eine Folge, die nicht konvergiert, heißt \underline{divergent} (die Folge divergiert).
	\end{itemize}
	\subsection{Bemerkung}
	$\rightarrow$ Folien 20.04.16\\
	\subsection{Beispiel}
	\begin{itemize}
		\item[a)] $a_n=\frac{1}{n}$ ist Nulfolge, d.h. $\lim\limits_{\infn}\frac{1}{n}=a=0$, denn:\\
		Sei $\epsilon>0$ beliebig. Dann wähle $N$ als $N>\frac{1}{\epsilon}$, denn damit gilt für alle $a_n$ mit $n\geq N$:\\
		$|a_n-0|=|\frac{1}{n}-0|=\frac{1}{n}\leq\frac{1}{N}$, da $n\geq N$ und $\frac{1}{N}<\frac{1}{\frac{1}{\epsilon}}=\epsilon\Rightarrow|a_n-0|<\epsilon$.\\
		(z.B. falls $\epsilon=\frac{1}{10}$, wähle $N>10$, z.B. $N=11$; ab $a_{11}$ haben alle Folgenglieder einen Abstand $<\frac{1}{10}$ von 0)
		\item[b)] $(a_n)$ mit $a_n=\frac{n+1}{3n}$. Behauptung: $a=\frac{1}{3}$.\\
		Beweis: Sei $\epsilon>0$ beliebig. Dann wähle $N>\frac{1}{3\epsilon}$. Für alle $a_n$ mit $n\geq N$ gilt dann:\\
		$|a_n-a|=|\frac{n+1}{3n}-\frac{1}{3}|=|\frac{n+1-n}{3n}|=\frac{1}{3n}<\frac{1}{3N}<\epsilon$. $\frac{1}{3N}<\epsilon$ genau dann, wenn $N>\frac{1}{3\epsilon}$.
		\item[c)] $(a_n)_{n\in\N}$ mit $a_n=c\quad\forall n$.\\
		$\lim\limits_{\infn}a_n=c$\\
		Sei $\epsilon>0$ beliebig. Dann ist\\
		$|a_n-c|=|c-c|=0<\epsilon\quad\forall n\geq1$, hier ist also $N=1$, hängt nicht von $\epsilon$ ab, untypisch.
	\end{itemize}
	\subsection{Bemerkung}
	$N$ muss nicht optimal gewählt werden.\\
	Beispiel: $\lim\limits_{\infn}\frac{1}{n^3+n+5}=0$, [...]\\
	$|\frac{1}{n^3+n+5}-0|=\frac{1}{n^3+n^+5}\leq\frac{1}{N^3+N+5}\overset{!}{<}\epsilon$. Für optimales $N$: $\frac{1}{N^3+N+5}<\epsilon$ nach $N$ auflösen, schwer.\\
	Deshalb grob abschätzen, z.B. so:\\
	$\frac{1}{N13+N+5}<\frac{1}{N}<\epsilon$, also wähle $N>\frac{1}{\epsilon}$.
	\subsection{Satz: Beschränktheit von Folgen}
	Jede konvergente folge ist beschränkt.\\
	\\
	Beweis: (zu zeigen: $(a_n)$ konvergente Folge: $\exists K\in\N$, so dass $|a_n|\leq K\quad\forall n\in\N$)\\
	Sei $(a_n)_{n\in\N}$ konvergent gegen $a$.\\
	dann existiert für alle $\epsilon>0$, also auch speziell für $\epsilon=1$, ein $N\in\N$ mit $|a_n-a|<1\quad\forall b\geq N$.\\
	Also gilt für alle $n\geq N$:\\
	\begin{align*}
		|a_n|&=|a_n+a-a|&\leq |a_n-a|+|a|\\
		&\textrm{'Einschiebetrick'} &\textrm{Dreiecksungleichung}\\
		|a_n|&&<1+|a|
	\end{align*}
	(also für $n\geq N$ sind die $|a_n|<1+|a|$; aber für $n=1,2,3,..., N-1$?)\\
	Definiere $K$ als $K\coloneqq\max\{|a_1|,|a_2|,|a_3|,...,|a_{N-1}|,1+|a|\}$\\
	Dann gilt $|a_n|\leq K\quad\forall n$.\\
	(Anmerkung: Durch den vorletzten Schritt ist meist $K\in\R^+$.)\\
	\subsection{Bemerkung}
	Nach 2.9 gilt:\\
	 $(a_n)$ konvergiert $\Rightarrow$ $(a_n)$ ist beschränkt\\
	Das ist äquivalent zu:\\
	$(a_n)$ ist nicht beschränkt $\Rightarrow$ $(a_n)$ konvergiert nicht\\
	(Kontraposition). Unbeschränkte Folgen sind also immer divergent.\\
	Bsp. $(a_n)$ mit $a_n=n$
	\subsection{Wichtiges Beispiel (geometrische Folgen)}
	Für $q\in\R$ gilt: 
	$\lim\limits_{\infn}q^n=\begin{cases}
	0\textrm{, falls }|q|<1\\
	1\textrm{, falls }|q|=1
	\end{cases}$\\
	Die Folge $(q^n)_n\in\N$ divergiert, falls $q=-1$ oder $|q|>1$.\\
	Beweis: \begin{itemize}
		\item[1. Fall] $|q|<1$ (zu zeigen $q^n\rightarrow 0$ für $\infn$)\\
		Sei $\epsilon>0$ beliebig. Dann ist
		\begin{align*}
			|q^n-0|=|q^n|=|q|^n&<\epsilon\\
			\Leftrightarrow n\*\ln|q|&<\ln\epsilon\\
			\Leftrightarrow n \qquad &\mathclap{\overset{da |q|<1}{\geq}}\qquad \frac{\ln\epsilon}{\ln|q|}
		\end{align*}
		Wähle $\N\ni N>\frac{\ln\epsilon}{\ln|q|}$, dann ist also $|q|^n<\epsilon\quad\forall n\geq N$.
		\item[2. Fall] $q=1$ $\rightarrow$ konstante 1-Folge, konvergiert, s. 2.7 c)
		\item[3. Fall] $|q|\geq 1, q\neq 1$\\
		Für $|q|>1$ ist $(q^n)$ unbeschränkt, also divergent (s. 2,9/2.10).\\
		Für $q=-1$: können wir erst später beweisen ($\rightarrow$ Cauchy-Folgen)
	\end{itemize}
	\subsection{Beispiel}
	Nach 2.11 sind die Folgen $((\frac{1}{2})^n)_{n\in\N}=(\frac{1}{2^n})_{n\in\N},\quad ((-\frac{7}{8})^n)_n\in\N$ Nullfolgen.
	\subsection{Satz: Rechenregeln für konvergente Folgen}
	Seien $(a_n), (b_n)$ reelle Folgen mit $\lim\limits_{\infn}a_n=a$ und $\lim\limits_{\infn}b_n=b$. Dann gilt:
	\begin{itemize}
		\item[a)] Die Folge $(c\*a_n)$ konvergiert gegen $c\*a, c\in\R$.
		\item[b)] Die Folge $(a_n\pm b_n)$ konvergiert gegen $a\pm b$.
		\item[c)] Die Folge $(a_n\*b_n)$ konvergiert gegen $a\*b$.
		\item[d)] Die Folge $(\frac{a_n}{b_n})$ konvergiert gegen $\frac{a}{b}$, falls $b_n, b\neq0$ und $|a_n|\rightarrow|a|$.
	\end{itemize}
	Seien weiter $(d_n), (e_n)$ reelle Folgen mit $\lim\limits_{\infn}d_n=0$, dann gilt:
	\begin{itemize}
		\item[e)] Ist $(e_n)$ beschränkt, dann ist $(d_n\*e_n)$ auch eine Nullfolge.
		\item[f)] Gilt $|e_n|\leq d_n\quad\forall n$, so ist $(e_n)$ auch eine Nullfolge.
	\end{itemize}
	Beweis [exemplarisch für a) und b), Rest s. Moodle]:\\
	\begin{itemize}
		\item[a)] Falls $c=0$: klar, konstante 0-Folge.\\
		Falls $c\neq 0$: Sei $\epsilon>0$ beliebig. Dann existiert $N\in\N$, so dass $|a_n-a|<\frac{\epsilon}{|c|}\quad\forall n\in\N$ (denn $a_n\rightarrow a$)\\
		Dann ist aber $|c\*a_n-c\*a|=|c\*(a_n-a)|=|c|\*\overbrace{|a_n-a|}^{<\frac{\epsilon}{|c|}}<\epsilon\quad\forall n\geq N$, also $c\*a_n\rightarrow c\*a$
		\item[b)] Sei $\epsilon>0$ beliebig.\\
		Dann $\exists N_1\in\N$, so dass $|a_n-a|<\frac{\epsilon}{2}\quad\forall n\geq N_1$ (denn $a_n\rightarrow a$)\\
		und $\exists N_2\in\N$, so dass $|b_n-b|<\frac{\epsilon}{2}\quad\forall n\geq N_2$ (denn $b_n\rightarrow b$).\\
		Dann gilt:
		\begin{align*}
			|(a_n+b_n)-(a+b)&|=|\overbrace{(a_n-a)}^{<\frac{\epsilon}{2}}+\overbrace{(b_n-b)}^{<\frac{\epsilon}{2}}|\overset{\triangle\textrm{-Ungleichung}}{\leq}|a_n-a|+|b_n-b|\\
			&<\frac{\epsilon}{2}+\frac{\epsilon}{2}=\epsilon\quad\forall n\geq N_1\textrm{ und }N_2
		\end{align*}
		(also z.B. für $n\geq N\coloneqq\max\{N_1,N_2\}$).\\
		\\
		Also gilt $(a_n+b_n)\rightarrow a+b$.\hfill$\square$
	\end{itemize}
	\subsection{Beispiel}
	\begin{itemize}
		\item[a)] $\frac{(-1)^n+5}{n}\rightarrow0$ für $\infn$, denn $\frac{1}{n}\rightarrow0$ für $\infn$ und $(-1)^n+5$ ist beschränkt: $|(-1)^n+5|\leq6\quad\forall n\in\N$ (nach 2.13 d)
		\item[b)] $\frac{3n^2-2n+1}{-n^2+n}\rightarrow-3$ für $\infn$, denn\\ $\frac{3n^2-2n+1}{-n^2+n}=\frac{n^2\*(3-\frac{2}{n}+\frac{1}{n^2})}{n^2\*(-1+\frac{1}{n})}=\frac{3-\bmark{\frac{2}{n}}+\bmark{\frac{1}{n^2}}}{-1+\bmark{\frac{1}{n}}}\quad\gmark{\frac{\rightarrow3\textrm{ für }\infn}{\rightarrow-1\textrm{ für }\infn}}\longrightarrow\frac{3}{-1}$ für $\infn$ (nach 2.13 b,d)\begin{tiny}
			[\bmark{Nullfolgen}]
		\end{tiny}
		\item[c)] \underline{Wichtiges Beispiel}\\
		Sei $x\in\R$ mit $|x|<1$, d.h. $|x|=\frac{1}{1+t}$ mit $t>0$.\\
		Sei $k\in\N_0$. Dann ist $\lim\limits_{\infn}(n^k\*x^n)=0$, denn
		\begin{align*}
			(1+t)^n\quad&\qquad\mathclap{\overset{\overset{\textrm{Mathe 1: 7.17}}{\textrm{bin. Lehrsatz}}}{=}}\qquad\quad\sum_{j=0}^{n}[\binom{n}{j}\*1^{n-j}\*t^j]\\
			&\qquad\mathclap{=}\overbrace{1}^{j=0}+\overbrace{nt}^{j=1}+\overbrace{\frac{n\*(n-1)}{2!}t^2}^{j=2}+\overbrace{\frac{n\*(n-1)\*(n-2)}{3!}t^3}^{j=3}+...\\
			&\qquad\mathclap{\overset{\overset{\textrm{nur Term}}{j=k+1}}{\geq}}\quad\frac{n\*(n-1)\*(n-2)\*...\*(n-k)}{(k+1)!}t^{k+1}=\binom{n}{k+1}t^{k+1}
		\end{align*}
		Damit gilt:
		\begin{align*}
			|n^k\*x^n|=|\frac{n^k}{(1+t)^n}|\leq\frac{n^k}{\binom{n}{k+1}t^{k+1}}=\frac{n^k}{n^{k+1}+...}\rightarrow 0
		\end{align*}
		für $\infn$.\\
		Es gilt also z.B. ($k=10000, x=\frac{1}{2}$): $\frac{n^{10000}}{2^{n}} \rightarrow 0$ für $\infn$\\
		$\Rightarrow \overset{\textrm{Exponentialfkt.}}{(1+t)^n}$ wächst schneller als jede Potenz $\overset{Polynom}{n^k}$ !
	\end{itemize}
	\subsection{Anmerkung (Landau-Symbole, $\mathcal{O}$-Notation)}
	(Informatik, VL Algorithmen)\\
	Sei $(a_n)$ eine \underline{strikt positive} Folge, d.h. $a_n>0\quad\forall n\in\N$. Dann ist
	\begin{itemize}
		\item[a)] $\mathcal{O}(a_n)=\mathcal{O}((a_n))=\{(b_n)|(\frac{b_n}{a_n})\textrm{ ist beschränkt }\}$ ("Menge aller Folgen, für die ... gilt")
		\item[b)] $o(a_n)=\{(b_n)|\frac{b_n}{a_n}\textrm{ ist Nullfolge }\}$ ($(a_n)$ wächst schneller als $(b_n)$)
	\end{itemize}
	$\mathcal{O},o\colon $ Landau-Symbole
	\begin{itemize}
		\item[c)] $(a_n)\sim(b_n)$, falls $\lim\limits_{\infn}(\frac{a_n}{b_n})_n=1$
	\end{itemize}
	Beispiel:
	\begin{itemize}
		\item $(2n^2+5n+1)_n\in\mathcal{O}(n^2)$, denn\\
		$(\frac{2n^2+5n+1}{n^2})=\frac{n^2\*(2+\frac{5}{n}+\frac{1}{n^2})}{n^2}\rightarrow2$ für $\infn$, beschränkt
		\item $(n^2)\in o(n^3)$
		\item $(n^3)\in o(2^n)$
		\item $(n13-3)\sim(n^3)$, denn $(\frac{n^3}{n^3-3})=(\frac{n^3\*(1)}{n^3\*(1-\frac{3}{n^3})})\rightarrow1$ für $\infn$
		\item häufig auch laxe Schreibweise
		\begin{align*}
			2n^2+5n+1&=\mathcal{O}(n^2)\\
			n^2&=o(n^3)
		\end{align*}
	\end{itemize}
	Außerdem:
	\begin{align*}
		\mathcal{O}(1)&=\textrm{ Menge der beschränkten Folgen}\\
		o(1)&=\textrm{ Menge der Nullfolgen}
	\end{align*}
	Wichtige Formel: \underline{Stirling}: $(n!)\sim(\sqrt{2\pi n}(\frac{n}{\e})^n)$\\
	\\
	Problem: Wie zeigt man die Konvergenz einer Folge, wenn man den Grenzwert nicht kennt?
	\subsection{Definition}
	Eine Folge reeller Zahle $(a_n)_n$ heißt
	\begin{itemize}
		\item[a)] \underline{(\bmark{streng}) monoton steigend/wachsend}, falls $a_{n+1}\overset{\bmark{>}}{\geq}a_n\quad\forall n\in\N$, Schreibweise: $(a_n)\nearrow$
		\item[b)] \underline{(\bmark{streng}) monoton fallend} $(a_n)\searrow$, falls $a_{n+1}\overset{\bmark{<}}{\leq}a_n\quad\forall n\in\N$
		\item[c)] \underline{monoton}, falls a) oder b) gilt (oder beides)
	\end{itemize}	
	\subsection{Beispiel}
	\begin{itemize}
		\item $(a_n)=(\frac{1}{n})$ ist streng monoton fallend
		\item $(a_n)=(1)$ ist monoton fallend und monoton steigend
		\item $(a_n)=((-1)^n)$ ist nicht monoton
	\end{itemize}
	\subsection{Bemerkung}
	$(a_n)\nearrow$ zeigt man so:
	\begin{align*}
		a_{n-1}-a_n&\geq0\quad\textrm{ oder }\\
		\frac{a_{n+1}}{a_n}&\geq1 
	\end{align*}
	\subsection{Satz (Monotone Konvergenz)}
	Jede beschränkte, monotone Folge reeller Zahlen $(a_n)_n$ konvergiert, und zwar gegen
	\begin{itemize}
		\item $\sup\{a_n\colon n\in\N\}$, falls $(a_n)$ monoton steigend oder gegen
		\item $\inf\{a_n\colon n\in\N\}$, falls $(a_n)$ monoton fallend ist.
	\end{itemize}
	Beweis:\\
	\\
	Sei $(a_n)\nearrow$ und beschrönkt.
	\begin{align*}
		\Rightarrow \{a_n\colon n\in\N\}&\subseteq\quad\textrm{ ist beschränkt}\\
		\overset{\overset{\textrm{Vollst.-Axiom}}{\textrm{Mathe 1, 8.16}}}{\Rightarrow}S&\coloneqq\sup\{a_n\colon n\in\N\}\quad\textrm{ existiert.}
	\end{align*}
	Wir zeigen: $a_n\rightarrow S$ für $\infn$.\\
	Sei $\epsilon>0$ beliebig. Zu zeigen ist $\exists N\in\N$ mit $|a_n-S|<\epsilon\quad\forall n\geq N$.\\
	Es gilt $a_n\leq S\quad\forall n\in\N$, also zu zeigen: $S-a_n<\epsilon\quad\forall n\geq N$.\\
	$S$ ist \underline{kleinste} obere Schranke, d.h. $S-\epsilon$ ist \underline{keine} obere Schranke\\
	\begin{align*}
		\Rightarrow\exists N\in\N\quad\textrm{ mit }\quad a_n&>S-\epsilon\quad\forall n\geq N\\
		\Rightarrow S-a_n&<\epsilon\qquad\forall n\geq N
	\end{align*}
	$(a_n)\searrow$ analog\hfill$\square$
	\subsection{Beispiel}
	\begin{itemize}
		\item[a)] $x\in\R^+$, dann $(x^n)\in o(n!)\qquad(x^n=o(n!))$, d.h. $a_n=\frac{x^n}{n!}\rightarrow 0$ für $\infn$
		\begin{itemize}
			\item $a_n>0$
			\item $\frac{a_{n+1}}{a_n}=\frac{x^{n+1}\*n!}{(n+1)!\*x^n}=\frac{x}{n+1}\leq 1$ für $n+1\geq x$, also gilt $a_{n+1}\leq a_n$, d.h. $(a_n)\searrow$ und $(a_n)$ ist beschränkt
			\item $\inf\{a_n\colon n\in\N\}=0$
		\end{itemize}
		\item[b)] \underline{wichtige Folge}\\
		\begin{align*}
			(a_n)_{n\in\N}&=((a+\frac{1}{n})^n)_{n\in\N}\\
			\lim\limits_{\infn}(a_n)&=\e\qquad\textrm{ (Eulersche Zahl, $\e=2,71828...$)}
		\end{align*}
		Warum existiert dieser Limes?\\
		Zeige: $(a_n)\nearrow$ und $(a_n)$ beschränkt, benutze Satz 2.19
		\begin{itemize}
			\item $(a_n)\nearrow$
			\begin{align*}
				\frac{a_n}{a_{n+1}}&=(\frac{1+n}{n})^n\*(\frac{n-1}{n})^{n-1}=(\frac{n+1}{n})^n\*(\frac{n-1}{n})^n\*(\frac{n-1}{n})^{-1}\geq 1\\
				&=(\frac{n^2-1}{n^2})^n\*\frac{n}{n-1}\\
				&=(1-\frac{1}{n^2})^n\*\frac{n}{n-1}\geq 1
			\end{align*}
			Benutze die \underline{Bernoulli-Ungleichung}, für $h\in\R,\quad n\in\N$ gilt $(1+h)^n\geq 1+nh$ für $h\geq -1$ (hier: $h=-\frac{1}{n^2}$)
			\begin{align*}
				\frac{a_n}{a_{n-1}}&=(1-\frac{1}{n^2})^n\*\frac{n}{n-1}\geq (1-n\*\frac{1}{n^2})\*\frac{n}{n-1}\\
				&=(1-\frac{1}{n})\*\frac{n}{n-1}=1\qquad\textrm{ , }
			\end{align*}
			also $(a_n)\nearrow$
			\item $(a_n)$ beschränkt: Übung, benutze wieder Bernoulli
		\end{itemize}
	\end{itemize}
	\subsection{Satz (Intervallschachtelungsprinzip)}
	Seien $(a_n),\quad(b_n)$ reelle Folgen mit
	\begin{itemize}
		\item $(a_n)\nearrow$ (= linke Intervallgrenze)
		\item $(b_n)\searrow$ (= rechte Intervallgrenze)
		\item $a_n\leq b_n\quad\forall n\in\N$
		\item $b_n-a_n\rightarrow 0$ für $\infn$
	\end{itemize}
	Dann sind beide Folgen konvergent und besitzen denselben Limes.\\
	\\
	Beweis:\\
	\\
	$(a_n),\quad(b_n)$ konvergent nach Satz 2.19, denn
	\begin{itemize}
		\item $(a_n)\nearrow$; $(a_n)$ beschränkt, da $a_n\leq b_n\quad\forall n\in\N$, also gilt auch $a_n\leq b$ (alle anderen $b_n$ sind noch kleiner)
		\item $(b_n)\searrow$; $(b_n)$ beschränkt, da $b_n\geq a_n\quad\forall n\in\N$, also $b_n\geq a_n\geq a_1$
		\item Da $(b_n)-(a_n)$ Nulfolge ist, sind auch die Grenzwerte gleich.
	\end{itemize}\hfill$\square$
	\subsection{Beispiel (vgl. Beispiel 2.20 b))}
	$a_n=(1+\frac{1}{n})^n$, $b_n=(1+\frac{1}{n})^{n+1}$\\
	Man kann zeigen: $(a_n)\nearrow$, $(b_n)\searrow$\\
	$a_n\leq b_n$, $b_n-a_n\rightarrow 0$, also $\exists\lim\limits_{\infn}(1+\frac{1}{n})^n=\lim\limits_{\infn}(1+\frac{1}{n})^{n+1}$\\
	\\
	Ähnlich zeigt man $\lim\limits_{\infn}(1+\frac{x}{n})^n$ existiert $\forall x\in\R$\\
	So definiert man $\e^x\coloneqq\lim\limits_{\infn}(1+\frac{x}{n})^n$\\
	\\
	\underline{Bisher:}\\
	$(a_n)$ konvergiert $\Rightarrow (a_n)$ beschränkt, Umkehrung gilt nicht; z.B. $((-1)^n)$\\
	Allerdings besitzt diese Folge zwei konvergente Teilfolgen mit $\lim\quad+1$ und $\lim\quad-1$.
	\subsection{Definition}
	Sei $(a_n)_{n\in\N}$ eine Folge und $(n_k)_{k\in\N}$ ($n_1,n_2,...$) eine streng monoton steigende Folge von Indizes (d.h. $n_1<n_2<n_3<...$).\\
	Dann heißt die Folge $(a_{n_k})_{k\in\N}$ \underline{Teilfolge} von $(a_n)_{n\in\N}$ ("Teilfolgen entstehen durch Streichung von Gliedern").
	\subsection{Beispiel}
	\begin{align*}
		(a_n)&=((-1)^n)\\
		n_k&\coloneqq 2n\quad\textrm{ ergibt  }(n_1=2;\ n_2=4\ n_3=6)\\
		&a_n=1\quad\forall n\in\N\quad\textrm{ (Teilfolge 1,1,1,1...)}\\
		n_k&\coloneqq 2n-1\quad\textrm{ ergibt (Teilfolge -1,-1,-1,...)}\\
		&a_{2n-1}=-1\quad\forall n\in\N
	\end{align*}
	\subsection{Bemerkung}
	Es gilt: $(a_n)$ konvergiert gegen $a$ $\Rightarrow$ jede Teilfolge von $(a_n)$ konvergiert gegen $a$.
	\subsection{Definition}
	Sei $(a_n)$ eine reelle Folge.
	Eine Zahl $h\in\R$ heißt \underline{Häufungspunkt} von $(a_n)$, wenn es eine Teilfolge von $(a_N)$ gibt, die gegen $h$ konvergiert.
	\subsection{Beispiel}
	\begin{itemize}
		\item $(a_n)=((-1)^n+\frac{1}{n})$ esitzt zwei Häufungspunkte $-1$ und $1$
		\item $(a_n)=((-1)^n)$ besitzt die Häufungspunkte $-1$ und $1$
	\end{itemize}
	\subsection{Satz (Satz von Bolzano-Weierstraß)}
	Sei $(a_n)$ eine reelle Folge. Dann gilt:
	\begin{align*}
		(a_n)\textrm{  beschränkt  }\Rightarrow(a_n)\textrm{  besitzt eine konvergente Teilfolge  }
	\end{align*}
	Beweis: Intervallschachtelungsprinzip/Bisektionsverfahren\\
	(s. Folien/Blatt[$\leftarrow$s.u.])
	\\
	Wir verwenden das Intervallschachtelungsprinzip (Satz 2.21). Nach Voraussetzung ist $(a_n)_{n\in\N}$ beschränkt, es existiert also ein $K\in\N$, so dass ale Folgeglieder im Intervall $[-K,K]\eqqcolon[A_0,B_0]$ liegen. Halbiere dieses Intervall:
	\begin{itemize}
		\item Falls in der ersten Hälfte des Intervalls unendlich viele Folgenglieder liegen: wähle eines davon aus.
		\item Falls nicht (also falls nur endlich viele Folgenglieder in der ersten Hälfte des Intervalls liegen), dann liegen in der zweiten Hälfte unendlich viele Folgenglieder. Wähle davon eines aus.
	\end{itemize}
	Das ausgewählte Folgenglied nennen wir $a_{n1}$, die Intervallhälfte, aus der es stammt, nennen wir $[A_1,B_1]$. Fahre nun so fort: Halbiere $[A_1,B_1]$, wähle wie oben $a_{n2}$ aus, erhalte damit das Intervall $[A_2,B_2]$, usw. So erhalten wir eine Teilfolge $(a_{n_k})_{k\in\N}$. Für die Intervalgrenzen von $[A_k,B_k]$ gilt:
	\begin{itemize}
		\item $A_k\leq a_{n_k}\leq B_k$
		\item $(A_k)_{k\in\N}\nearrow,\qquad(B_k)_{k\in\N}\searrow$
		\item $A_k\leq B_k$
		\item $B_k-A_k\rightarrow 0$ für $k\rightarrow\infty$. 
	\end{itemize}
	Damit sind alle Voraussetzungen für Satz 2.21 (Intervallschachtelungsprinzip) erfüllt. Die Folgen $(A_k)_{k\in\N}$ und $(B_k)_{k\in\N}$ sind also konvergent und besitzen denselben Limes $a$. Damit gilt auch $a_{n_k}\rightarrow a$ für $k\rightarrow\infty$.\hfill$\square$
	\subsection{Bemerkung/Definition}
	Sei $(a_n)$ reell und beschränkt, dann gibt es inen größten und einen kleinsten Häufungspunkt, den
	\begin{itemize}
		\item \underline{Limes superior} von $(a_n)$: $\limsuperior{\infn}a_n$ oder $\limsup{\infn}a_n$ bzw. den
		\item \underline{Limes inferior} von $(a_n)$: $\liminferior{\infn}a_n$ oder $\liminf{\infn}a_n$.
	\end{itemize}
	Weiter setzt man
	\begin{itemize}
		\item $\limsup{\infn}a_n\coloneqq\begin{cases}
			\infty\textrm{, wenn }(a_n)\textrm{ nicht nach oben beschränkt ist}\\
			-\infty\textrm{, wenn }(a_n)\rightarrow-\infty\textrm{ gilt, d.h. }\forall K>0\quad\exists N\in\N\colon a_n\leq-K\quad\forall n\geq N 
		\end{cases}$
		\item $\liminf{\infn}a_n\coloneqq\begin{cases}
		-\infty\textrm{, wenn }(a_n)\textrm{ nicht nach unten beschränkt ist}\\
		\infty\textrm{, wenn }(a_n)\rightarrow\infty\textrm{ gilt, d.h. }\forall K>0\quad\exists N\in\N\colon a_n\geq K\quad\forall n\geq N
		\end{cases}$
	\end{itemize}
	\underline{Achtung:} $-\infty,\infty$ sind keine reellen Zahlen!\\
	Man erweitert hier $\R$ um zwei ideelle Elemente $-\infty,\infty$, setzt $\overline{\R}=\R\cup\{\infty,-\infty\}$ (Abschluss von $\R$) und erweitert die Ordnungsstruktur auf $\R$ durch\\ $-\infty<x<\infty\quad\forall x\in\R$.\\
	\\
	Mit dieser Festlegung besitzt \underline{jede} reelle Zahlenfolge sowohl $\limsuperior{}$ als auch $\liminferior{}$.\\
	Beispiel:
	\begin{itemize}
		\item[a)] $a_n=\frac{n+1}{n}\qquad\limsup{\infn}a_n=\liminf{\infn}a_n=1$
		\item[b)] $a_n=(-1)^n\qquad\limsup{\infn}a_n=1\qquad\liminf{\infn}a_n=-1$
		\item[c)] $a_n=(-1)^n\*n\qquad\limsup{\infn}a_n=\infty\qquad\liminf{\infn}a_n=-\infty$
		\item[d)] $a_n=n\*(1+(-1)^n)$ : Übung
	\end{itemize}
	\subsection{Definition (Cauchyfolge)}
	Eine Folge $(a_n)$ heißt \underline{Cauchyfolge}, falls es zu jedem $\epsilon>0$ ein $N\in\N$ gibt, so dass $|a_n-a_m|<\epsilon\quad\forall n,m\geq N$\\
	(kurz: $\forall\epsilon>0\quad\exists N\in\N\quad\forall n,m\geq N\colon|a_n-a_m|<\epsilon$) mit $|a_n-a_m|$... Abstand zweier Folgenglieder
	\subsection{Satz (Cauchykriterium)}
	Eine Folge konvergiert genau dann, wenn sie eine Cauchyfolge ist.
	\begin{align*}
		(a_n)\textrm{ konvergiert }\Leftrightarrow(a_n)\textrm{ ist eine Cauchyfolge }
	\end{align*}
	Beweisskizze (ausführlicher Beweis: s. Moodle):
	\begin{itemize}
		\item "$\Rightarrow$": Einschiebetrick, Dreiecksungleichung verwenden
		\item "$\Leftarrow$": Idee: $(a_n)$ ist Cauchyfolge (zu zeigen: konvergent)\\
		zeige: $(a_n)$ ist beschränkt\\
		$\Rightarrow$ 2.28 $\exists$ konvergente Teilfolge\\
		zeige: Limes der Teilfolge ist Limes der Folge
	\end{itemize}
	\subsection{Anwendung (Banachscher Fixpunktsatz)}
	Sei $f\colon[a,b]\rightarrow[a,b]$ eine Abbildung mit\\
	$\underbrace{|f(x)-f(y)|}_{\textrm{Abstand der Bildpunkte}}<\underbrace{|x-y|}_{\textrm{Abstand von 2 Punkten}}\qquad\forall x,y\in[a,b]$\\
	("$f$ ist strikte Kontraktion")\\
	Dann hat $f$ genau einen Fixpunkt, d.h.\\
	$\underbrace{\exists!}_{\textrm{es gibt genau ein...}}r\in[a,b]$ mit $f(r)=r$\\
	\\
	Beweisidee:\\
	Starte mit beliebigem $x_0\in[a,b]$.\\
	Berechne $x_1$ als $f(x_0)\qquad x_1\coloneqq f(x_0)$\\
	$x_2$ als $f(x_1)\qquad x_2\coloneqq f(x_1)$\\
	also $x_{n+1}\coloneqq f(x_n)$\\
	\\
	Zeige: Diese Folge konvergiert (Cauchyfolge), und zwar gegen $r=f(r)$; $r$ ist eindeutig (Annahme: es existieren 2 verschiedene $r$)
	%%%%%%%%%%%%%%%%%%%%%%%%%%%%%%%%%%%%%%%%%%%%%%%%%%%%%%%
	\newpage
	%%%%%%%%%%%%%%%%%%%%%%%%%%%%%%%%%%%%%%%%%%%%%%%%%%%%%%%
	\section{Reihen}
	\subsection{Definition}
	Sei $(a_n)_{n\in\N}$ eine Folge.\\
	Summiere die ersten $n$ Folgeglieder.
	\begin{align*}
		S\coloneqq\sum_{k=1}^{n}a_k\qquad\forall
		 n\in\N\qquad (=a_1+a_2+a_3+...+a_n)
	\end{align*}
	(\underline{$n$-te Partialsumme})
	\begin{align*}
		\underbrace{\underbrace{\underbrace{\underbrace{a_1}_{S_1}+a_2}_{S_2}+a_3}_{S_3}+...+a_n}_{S_n}
	\end{align*}
	Die Folge $(S_n)_{n\in\N}=(S_1,S_2,S_3,...)$ heißt \underline{unendliche Reihe}, schreibe $\sum_{k=1}^{\infty}a_k$\\
	Falls $(S_n)_{n\in\N}$ gegen $s\in\R$ konvergiert, heißt die Reihe \underline{konvergent gegen $s$} und ihr Grenzwert wird dann ebenfalls mit $\sum_{k=1}^{\infty}a_k$ bezeichnet.\\
	(Entsprechend kann man für eine Folge $(a_n)_{n\geq n_0}$ die Reihe $\sum_{k=n_0}^{\infty}a_k$ definieren)
	\subsection{Beispiel}
	\begin{itemize}
		\item[a)] $\sum_{k=1}^{\infty}k=1+2+3+...\qquad$ divergente Folge
		\item[b)] $\sum_{k=1}^{\infty}(-1)^k=(-1)+1+(-1)+...\qquad$ divergente Folge\\
		$S_n=\sum_{k=1}^{n}(-1)^k=\begin{cases}
		0\textrm{, falls }n\textrm{ gerade}\\
		-1\textrm{, falls }n\textrm{ ungerade}
		\end{cases}$
		\item[c)] \underline{Die harmonische Reihe}
		\begin{align*}
			\sum_{k=1}^{\infty}\frac{1}{k}&=1+\frac{1}{2}+\frac{1}{3}+\frac{1}{4}+\frac{1}{5}+...\qquad\textrm{divergiert}\\
			S_n&=1+\frac{1}{2}+\underbrace{\frac{1}{3}+\frac{1}{4}}_{>2\*\frac{1}{4}=\frac{1}{2}}+\underbrace{\frac{1}{5}+...+\frac{1}{8}}_{>4\*\frac{1}{8}=\frac{1}{2}}+\underbrace{\frac{1}{9}+...+\frac{1}{16}}_{>8\*\frac{1}{16}=\frac{1}{2}}+\underbrace{...+\frac{1}{n}}_{\textrm{usw.}}\\
			&>1+\frac{1}{2}+\quad\frac{1}{2}\quad+\qquad\frac{1}{2}\qquad+\qquad\frac{1}{2}\qquad+\quad...
		\end{align*}
		$\Rightarrow$ divergent (per Induktion: $S_{2^m}\geq1+\frac{m}{2}$)
		\item[d)] $\sum_{k=0}^{\infty}\frac{1}{2^k}=1+\frac{1}{2}+\frac{1}{4}+\frac{1}{8}+\frac{1}{16}+...\qquad$ it konvergent gegen den Grenzwert $\sum_{k=0}^{\infty}\frac{1}{2^k}=2$
		\item[e)] wichtiges Beispiel: \underline{Geometrische Reihe}\\
		Für $q\in\R$ mit $|q|<1$ gilt
		\begin{align*}
			\sum_{k=0}^{\infty}q^k&=\frac{1}{1-q}\qquad\textrm{, denn:}\\
			S_n=\sum_{k=0}^{n}q^k&=\frac{1-q^{n+1}}{1-q}\qquad\textrm{(Übung: geom. Summe, Induktion)}
		\end{align*}
		Aus 2.11:
		\begin{align*}
			\lim\limits_{\infn}q^n=0\qquad\textrm{, falls }|q|<1
		\end{align*}
		Geometrische Folge. Also gilt:
		\begin{align*}
			&S_n\rightarrow\frac{1-0}{1-q}=\frac{1}{1-q}\qquad\textrm{ für }\infn\\
			&\sum_{k=0}^{\infty}q^k\qquad\textrm{ divergiert für }|q|\geq 1
		\end{align*}
		Nochmal Beispiel d)\\
		$\sum_{k=0}^{\infty}\frac{1}{2^k}=\sum_{k=0}^{\infty}(\frac{1}{2})^k$, also geometrische Reihe mit $q=\frac{1}{2}\qquad1>|q|$, konvergiert gegen $\frac{1}{1-q}=\frac{1}{1-\frac{1}{2}}=\frac{1}{\frac{1}{2}}=2$\\
		\\
		Weitere Beispiele:
		\begin{itemize}
			\item $\sum_{k=0}^{\infty}\frac{(-1)^k}{2^k}=\sum_{k=0}^{\infty}(\frac{1}{2})^k=\frac{2}{3}$
			\item $\sum_{k=3}^{\infty}q^k=\sum_{k=0}^{\infty}q^{k+3}=q^3\*\sum_{k=0}^{\infty}q^k=\frac{q^3}{1-q}\qquad$ (falls $|q|<1$)
		\end{itemize}
	\end{itemize}
		\subsection{Rechenregeln für Reihen}
		folgen aus den Rechenregeln für Folgen. Sei
		\begin{itemize}
			\item $\sum_{k=1}^{\infty}a_k$ konvergiert gegen $a$,
			\item $\sum_{k=1}^{\infty}b_k$ konvergiert gegen $b$.
		\end{itemize}
		Dann gilt mit $c\in\R$:
		\begin{itemize}
			\item[a)] $\sum_{k=1}^{\infty}(a_k+b_k)$ konvergiert gegen $a+b$
			\item[b)] $\sum_{k=1}^{\infty}(c\*a_k)$ konvergiert gegen $c\*a$
		\end{itemize}
	\subsection{Konvergenz-/Divergenzkriterien für Reihen}
	\begin{itemize}
		\item[\fbox{1}] Ist $S_n$ mit $S_n=\sum_{k=1}^{\infty}a_k$ beschränkt und $a_k\geq 0\quad\forall k\in\N$, so ist $\sum_{k=1}^{\infty}a_k$ konvergent (folgt aus Satz 2.19/monotone Konvergenz).
		\item[\fbox{2}] \underline{Cauchy-Kriterium}\\
		$\sum_{k=1}^{\infty}a_k$ konvergiert $\Leftrightarrow$ $\forall\epsilon>0\quad\exists N\in\N$, so dass $\forall m>n\geq N$ gilt: $|a_{n+1}+...+a_m|=|\sum_{k=n+1}^{m}a_k|<\epsilon$\\
		$|S_m-S_n|$\\
		(folgt aus 2.31/Cauchykriterium für Folgen)\\
		Daraus ergibt sich:\\
		Ist $\sum_{k=1}^{\infty}a_k$ konvergent, so ist $(a_n)_n$ Nullfolge (wähle $m=n+1$, dann $|a_{n+1}|<\epsilon$, d.h. $a_n\rightarrow0$).\\
		$\Rightarrow$[3]
		\item[\fbox{3}] \underline{Divergenzkriterium}\\
		Ist $(a_n)_n$ keine Nullfolge, so ist $\sum_{k=1}^{\infty}a_k$ divergent.\\
		Bsp: $\sum_{k=1}^{\infty}\underbrace{(1+\frac{1}{k})}_{\rightarrow1\textrm{ für }k\rightarrow\infty\textrm{, keine Nullfolge!}}$ divergiert
		\item[\fbox{4}] \underline{Majorantenkriterium}\\
		Seien $(a_n),(b_n)$ Folgen mit $|a_n|\leq b_n\quad\forall n\in\N$ (für fast alle $n$, d.h. für alle bis auf endlich viele)\\
		Dann gilt:\\
		Ist $\underbrace{\sum_{k=1}^{\infty}b_k}_{\textrm{Majorante}}$ konvergent, dann auch $\sum_{k=1}^{\infty}a_k$ und $\sum_{k=1}^{\infty}|a_k|$\\
		\\
		Beweis:
		\begin{align*}
			|\sum_{k=n+1}^{m}a_k|&\leq\sum_{k=n+1}^{m}|a_k|\\
			&\leq\sum_{k=n+1}^{m}b_k\\
			&\leq|\sum_{k=n+1}^{m}b_k|<\epsilon\textrm{ , da }\sum_{k=1}^{\infty}b_k\textrm{ konvergent,}
		\end{align*}
		also ist Cauchykriterium [2] für $\sum_{k=1}^{\infty}a_k$ erfüllt, $\sum_{k=1}^{\infty}a_k$ konvergiert.\\
		Ähnlich: \underline{Minorantenkriterium} für Divergenz, s. Blatt 5.
		\item[\fbox{5}] \underline{Leibnitzkriterium für alternierende Reihen}\\
		Sei $(a_n)_n$ reelle, monoton fallende Nullfolge mit $a_n\geq0\quad\forall n$.\\
		Dann konvergiert die alternierende Reihe $\sum_{k=0}^{\infty}(-1)^k\*a_k$\\
		\\
		Beweis: Intervallschachtelungsprinzip
		\begin{align*}
			A_n&\coloneqq\sum_{k=0}^{2n-1}(-1)^k\*a_k\\
			B_n&\coloneqq\sum_{k=0}^{2n}(-1)^k\*a_k
		\end{align*}
	\begin{itemize}
		\item $A_n\nearrow$, denn
		\begin{align*}
			A_{n1}-A_n & =\overbrace{\sum_{k=0}^{2(n+1)-1}}^{2n+1}(-1)^k\*a_k-\sum_{k=0}^{2n-1}(-1)^k\*a_k &  \\
			           & =(-1)^{2n+1}a_{2n+1}+(-1)^{2n}a_{2n}                                              & =-a_{2n+1}+a_{2n}\geq 0
		\end{align*}
		(da $(a_n)\searrow$)
		\item ähnlich für $B_n\searrow$
		\item $B_n-A_n=(-1)^{2n}a_{2n}=a_{2n}\geq0\quad\longrightarrow0$ für $\infn$ (weil $(a_n)_n$ Nullfolge nach Voraussetzung)\\
		$\Rightarrow\exists\lim\limits_{\infn}A_n=\lim\limits_{\infn}B_n$, also konvergiert $\sum_{k=0}^{\infty}(-1)^ka_k$
	\end{itemize}
	Bsp:
	\begin{itemize}
		\item[a)] Leibnitz-Reihe:
		\begin{align*}
			&1-\frac{1}{3}+\frac{1}{5}-\frac{1}{7}+...-...\\
			=&\sum_{k=0}^{\infty}(-1)^k\frac{1}{2k+1}
		\end{align*}
		konvergiert gegen $\frac{\pi}{4}$
		\item[b)] Die alternierende harmonische Reihe
		\begin{align*}
			&1-\frac{1}{2}+\frac{1}{3}-\frac{1}{4}+\frac{1}{5}-...+...\\
			=&\sum_{k=0}^{\infty}(-1)^k\frac{1}{k+1}
		\end{align*}
		konvergiert gegen $\ln2$
	\end{itemize}
	\item[\fbox{6}] \underline{Absolute Konvergenz}
	\subsubsection*{Definition}
	Eine Reihe $\sum_{k=0}^{\infty}a_k$ heißt \underline{absolut konvergent}, falls die Betragsreihe $\sum_{k=0}^{\infty}|a_k|$ konvergiert.\\
	\subsubsection*{Beispiel}
	\begin{itemize}
		\item[a)] $\sum_{k=1}^{\infty}(-1)^k\frac{1}{k^2}$ konvergiert absolut, da $\sum_{k=1}^{\infty}|(-1)^k\frac{1}{k^2}|=\underbrace{\sum_{k=1}^{\infty}\frac{1}{k^2}}_{\textrm{s. \fbox{6a}}}$ konvergiert
		\item[b)] $\sum_{k=1}^{\infty}(-1)^k\frac{1}{k}$ konvergiert nicht absolut (aber konvergiert, s. Leibnitzkriterium), da $\sum_{k=1}^{\infty}|(-1)^k\frac{1}{k}|=\sum_{k=1}^{\infty}\frac{1}{k}$ (harmonische Reihe, konvergiert nicht)
	\end{itemize}
	Es gilt: $\overset{\textrm{(Majorantenkriterium)}}{\textrm{Reihe konvergiert absolut}}$ $\Rightarrow$ Reihe konvergiert\\
	(aber nicht umgekehrt, s. Beispiel b))
	\item[\fbox{6a}] \underline{Wurzelkriterium}\\
	Für $a_k\in\R$ gilt:
	\begin{itemize}
		\item falls $\limsup{\infn}\sqrt[n]{|a_n|}<1\Rightarrow\sum_{k=0}^{\infty}|a_k|$ konvergiert (d.h. $\sum_{k=0}^{\infty}a_k$ konvergiert absolut)
		\item falls $\limsup{\infn}\sqrt[n]{|a_n|}>1\Rightarrow\sum_{k=0}^{\infty}a_k$ divergiert
		\item für $\limsup{\infn}\sqrt[n]{|a_k|}=1$ ist keine allgemeine Aussage möglich
	\end{itemize}
	Beweis:\\
	Sei $s\coloneqq\limsup{\infn}\sqrt[n]{|a_n|}$
	\begin{itemize}
		\item falls $s<1$: Wähle kleines $\epsilon>0$, so dass $s+\epsilon<1$\\
		$\Rightarrow\sqrt[n]{|a_n|}\leq s+\epsilon$ für fast alle $n$\\
		$\Rightarrow|a_n|leq(s+\epsilon)^n$\\
		Die Reihe $\sum_{k=0}^{\infty}{\underbrace{(s+\epsilon)}_{<1}}^n$ ist geometrische Reihe und konvergiert, und ist Majorante für die Reihe $\sum_{k=0}^{\infty}|a_k|$
		\item falls $s>1$, dann ist $\sqrt[n]{|a_n|}>1$ für unendlich viele $n$, also $a_n\rightarrow0$, $\sum_{k=0}^{\infty}a_n$ divergent nach \fbox{3}
		\item z.B. $\sum_{k=1}^{\infty}\frac{1}{k^\alpha}$ (allgemeine harmonische Reihe) mit $\alpha\geq1$ liefert $\limsup{\infn}\sqrt[n]{|a_n|}=1$, aber es gilt (Mitteilung):\\
		für $\alpha=1$ ist Reihe divergent (für $0<\alpha<1$ ebenso, Blatt 5 Aufgabe 2);\\
		für $\alpha>1$ ist Reihe konvergent\\
		Das Wurzelkriterium kann diese Fälle nicht unterscheiden.
	\end{itemize}
	\item[\fbox{6b}] \underline{Quotientenkriterium}\\
	Sei $a_n\neq0$ für fast alle $k$ (d.h. für alle bis auf endlich viele)
	\begin{itemize}
		\item falls $\limsup{\infn}|\frac{a_{n+1}}{a_n}|<1\Rightarrow\sum_{k=0}^{\infty}|a_k|$ konvergiert
		\item falls $\liminf{\infn}|\frac{a_{n+1}}{a_n}|>1\Rightarrow\sum_{k=0}^{\infty}a_k$ divergiert
		\item falls $\limsup{\infn}|\frac{a_{n+1}}{a_n}|\geq1$ und $\liminf{\infn}|\frac{a_{n+1}}{a_n}|\leq1$, so ist keine allgemeine Aussage möglich (wie bei \fbox{6a}, dritter Punkt)
	\end{itemize}
	Beweis: ähnlich wie \fbox{6a}
 \end{itemize}%\fbox{numbers}
 \subsection{Bemerkung}
 Umordnung einer Reihe, Konvergenzverhalten\\
 $\rightarrow$ s. Folien 11.05.2016
	
	
	
	
	
	
	
	
	
	
	
	
	
	
	
\end{document}
