\documentclass[12pt, titlepage]{article}

\usepackage[ngerman]{babel}
\usepackage[utf8]{inputenc}
\usepackage{color}
\usepackage{amssymb}
\usepackage{amsmath}
\usepackage{dsfont}
\usepackage{mathtools}
\usepackage[colorlinks=true, linkcolor=black, citecolor=black filecolor=green, urlcolor=blue]{hyperref}
\usepackage[a4paper]{geometry}
\usepackage{tikz}

\begin{titlepage}
	\title{Mathematik II}
	\date{20.07.2016}
\end{titlepage}

\newcommand{\R}{\mathds{R}}
\newcommand{\N}{\mathds{N}}
\newcommand{\e}{\textrm{e}}
\newcommand{\infn}{n\rightarrow\infty}
\newcommand{\bmark}[1]{\textcolor{blue}{#1}}
\newcommand{\gmark}[1]{\textcolor{lightgray}{#1}}
\newcommand{\limsuperior}[1]{\lim\limits_{#1}\sup}
\newcommand{\liminferior}[1]{\lim\limits_{#1}\inf}
\renewcommand{\vec}[1]{\left(\begin{array}{c}#1
	\end{array}\right)}
\renewcommand{\O}{\mathcal{O}}
\renewenvironment{rcases}{% 
	\left.\renewcommand*\lbrace.% 
	\begin{cases}}% 
	{\end{cases}\right\rbrace}

\renewcommand{\>}{\rightarrow}
\renewcommand{\*}{\cdot}
\renewcommand{\epsilon}{\varepsilon}
\renewcommand{\limsup}[1]{\underset{#1}{\overline{\lim}}}
\renewcommand{\liminf}[1]{\underset{#1}{\underline{\lim}}}
\renewcommand{\dfrac}[2]{\frac{\boldsymbol{d}#1}{\boldsymbol{d}#2}}

\begin{document}
	\maketitle
	\tableofcontents
	\newpage
	\section{Reelle Funktionen}
	\subsection{Wiederholung Mathe 1: Funktionen}
	\subsubsection*{Definition}
		Eine \underline{Funktion/Abbildung} $f\colon A\> B$ besteht aus
		\begin{itemize}
			\item zwei Mengen:
			\begin{itemize}
				\item $A$: \underline{Definitionsbereich} von $f$
				\item $B$: \underline{Bildbereich} von $f$
			\end{itemize}
			\item und einer \underline{Zuordnungsvorschrift}, die jedem Element $a\in A$ genau ein Element $b\in B$ zuordnet.
		\end{itemize}
		Wir schreiben dann $b=f(a)$, nennen $b$ das \underline{Bild}/den \underline{Funktionswert} von $a$ (unter $f$) sowie $a$ (ein) \underline{Urbild} von $b$ (unter $f$).
	\subsubsection*{Notation}
		\vspace{-1cm}\begin{align*}
			f\colon A&\> B\\
			a&\mapsto f(a)
		\end{align*}
	\subsubsection*{Beispiel}
		$\>$ Folien 11.04.2016
	\subsection{Reelle Funktionen}
	\subsubsection*{Definition}
	Eine \underline{reelle Funktion} einer \underline{Veränderlichen} ist eine Abbildung $f\colon D\> \R$, wobei $D\subseteq \R$ (oft ist $D$ endliche Vereinigung von Intervallen, z.B.
	\begin{itemize}
		\item $ D=(-\infty,a]=\{x\in \R|x\leq a\} $
		\item $ D=\R^+_0=[0,\infty)=\{x\in\R|x\geq 0\}$
		\item $ D=(-\infty,\infty)=\R $
		\item $ D=\R\setminus\{0\}=(-\infty,0)\cup(0,\infty) $\hfill).
	\end{itemize}
	\subsection{Neue Funktionen aus Alten, Kompositionen}
	\subsubsection*{Definition}
	Seien $f,g\colon D\> \R$ reelle Funktionen.
	\begin{itemize}
		\item[a)] $(f\pm g)(x)\coloneqq f(x)\pm g(x)\qquad \forall x\in D$\\
				  \underline{Summe/Differenz} von $f$ und $g$\\
				  (genauer:\vspace{-0.5cm} \begin{align*}
				  	f\pm g\colon D&\> \R\\
				  	x&\mapsto (f\pm g)(x)=f(x)\pm g(x) \textrm{\qquad )}\end{align*}
		\item[b)] $(f\* g)(x)\coloneqq f(x)\* g(x)\qquad \forall x\in D$\\
		\underline{Produkt} von $f$ und $g$
		\item[c)] falls $g(x)\neq 0\quad \forall x\in D$, dann\\
		$(\frac{f}{g})(x)\coloneqq\frac{f(x)}{g(x)}\qquad \forall x\in D$\\
		\underline{Quotient} von $f$ und $g$
		\item[d)] \underline{Komposition/Hintereinanderausführung}\\
		$f\colon D_f\> \R,\quad g\colon D_g\>\R\textrm{, wobei } f(D_f)\subseteq D_g$\\
		\vspace{-0.5cm}\begin{align*}
			g\circ f\colon D_f&\>\R\\
			x&\mapsto g(f(x))
		\end{align*}
	\end{itemize}
	\subsection{Beispiel}
	\begin{align*}
		f,g&\colon \R\>\R\\
		f(x)&=x^2\\
		g(x)&=x-1\\
		\\
		(f+g)(x)&=x^2+x-1\\
		(f\* g)(x)&=x^2\* (x-1)=x^3-x^2\\
		(\frac{f}{g})(x)&=\frac{x^2}{x-1}\quad\textrm{für }x\neq 1\quad (D_g=\R\setminus \{1\})\\
		(g\circ f)(x)&=g(f(x))=g(x^2)=x^2-1\\
		(f\circ g)(x)&=f(g(x))=f(x-1)=(x-1)^2=x^2-2x+1\\
		\\
		\Rightarrow (g\circ f)(x)&\neq (f\circ g)(x)
	\end{align*}
	\subsection{Wiederholung Mathe 1: Injektivität, Surjektivität, Bijektivität; Umkehrfunktion}
	$\>$ Folien 13.04.2016	
	\subsection{Elementare Funktionen (naive Einführung)}
	\begin{itemize}
		\item[a)] \underline{Konstante Funktionen}\\
		für $c\in\R$ (fest):\\
		\begin{minipage}[c]{0.5\textwidth}
			\begin{align*}
				f\colon \R&\>\R\\
				x&\mapsto c
			\end{align*}
		\end{minipage}
		\begin{minipage}[c]{0.5\textwidth}
			\begin{tikzpicture}[scale=0.5,domain=0:4]
				\draw[->]
					(-0.2,0) -- (4.2,0) node[right] {$x$};
				\draw[->]
					(0,-0.2) -- (0,4.2) node[above] {$f(x)$};
				\draw[color=blue]
					plot(\x,1.5) node[right]{$f(x)=c$};
			\end{tikzpicture}
		\end{minipage}
		\item[b)] \underline{Die identische Funktion (Identität)}\\
		\begin{minipage}[c]{0.5\textwidth}
			\begin{align*}
				f\colon \R&\> \R\\
				x&\mapsto x
			\end{align*}
		\end{minipage}
		\begin{minipage}[c]{0.5\textwidth}
			\begin{tikzpicture}[scale=0.5,domain=0:4]
			\draw[->]
				(-0.2,0) -- (4.2,0) node[right] {$x$};
			\draw[->]
				(0,-0.2) -- (0,4.2) node[above] {$f(x)$};
			\draw[color=blue]
				plot(\x,\x) node[right]{$f(x)=x$};
			\end{tikzpicture}
		\end{minipage}
	\end{itemize}
	Durch mehrfache Anwendung von 1.3 entstehen aus a) und b) viele weitere Funktionen.
	\begin{itemize}
		\item[c)] \underline{Potenzen (Monome)}\\
		für $n\in\N_0$ (fest):
		\begin{align*}
			f\colon \R&\>\R\\
			x&\mapsto x^n
		\end{align*}
		\begin{itemize}
			\item $n=0$: die konstante 1-Funktion
			\begin{align*}
				f\colon \R&\>\R\\
				x&\mapsto x^0=1
			\end{align*}\\
			\item $n$ ungerade:\\
			\begin{minipage}[c]{0.5\textwidth}
				$f$ punktsymmetrisch zum Ursprung $(0|0)$, bijektiv
			\end{minipage}
			\begin{minipage}[c]{0.5\textwidth}
				\begin{tikzpicture}[scale=0.5]
				\draw[->]
				(-4.2,0) -- (4.2,0) node[right] {$x$};
				\draw[->]
				(0,-4.2) -- (0,4.2) node[above] {$f(x)$};
				\draw[color=blue, domain=-2:2]
				plot(\x,\x*\x*\x*0.5) node[right]{$f(x)=x^n$};
				\end{tikzpicture}
			\end{minipage}
			\item $n$ gerade:\\
			\begin{minipage}[c]{0.5\textwidth}
				$f$ achsensymmetrisch zur $y$-Achse, nicht bijektiv\\ $f(x)\geq 0\quad\forall x\in\R$
			\end{minipage}
			\begin{minipage}[c]{0.5\textwidth}
				\begin{tikzpicture}[scale=0.5]
				\draw[->]
				(-4.2,0) -- (4.2,0) node[right] {$x$};
				\draw[->]
				(0,-0.2) -- (0,4.2) node[above] {$f(x)$};
				\draw[color=blue, domain=-3:3]
				plot(\x,\x*\x*0.5) node[right]{$f(x)=x^n$};
				\end{tikzpicture}
			\end{minipage}
		\end{itemize}
		\item[d)] \underline{Wurzelfunktionen}\\
		Wurzelfunktionen sind die Umkehrfunktionen der Monome. Dazu musss die Gleichung $f(x)=x^n=y$ ($y\in\R$ gegeben) gelöst werden.
		\begin{itemize}
			\item $n$ ungerade:\\
			\\
			\begin{minipage}[c]{0.5\textwidth}
				$f$ ist bijektiv, dann gibt es zu jedem \\
				$y\in\R$ genau ein $x\in\R$ mit $x^n=y$. Dieses wird die $n$-te Wurzel aus $y$ genannt:\\ $x=\sqrt[n]{y}$.\\
				\vspace{-0.5cm}
				\begin{align*}
					\sqrt[n]{\quad}\colon \R&\>\R\\
					x&\mapsto\sqrt[n]{x}
				\end{align*}
			\end{minipage}
			\begin{minipage}[c]{0.5\textwidth}
				\begin{tikzpicture}[scale=0.5]
				\draw[->]
				(-4.2,0) -- (4.2,0) node[right] {$x$};
				\draw[->]
				(0,-4.2) -- (0,4.2) node[above] {$f(x)$};
				\draw[color=gray, domain=-2:2,very thin]
				plot(\x,\x*\x*\x*0.5) node[right]{$f(x)=x^n$};
				\draw[color=blue,domain=-1.7:1.7]
				plot(\x*\x*\x,\x) node[right]{$f(x)=\sqrt[n]{x}$};
				\end{tikzpicture}
			\end{minipage}
			\item $n$ gerade: Dann hat die Gleichung $x^n=y$ in $\R$
			\begin{itemize}
				\item keine Lösung, falls $y<0$
				\item genau eine Lösung, falls $y=0$ (nämlich $x=0$)
				\item zwei Lösungen, falls $y>0$:
				\begin{align*}
					x_1&=\sqrt[n]{y}\quad(>0)\\
					x_2&=-\sqrt[n]{y}\quad(<0)
				\end{align*}
				\begin{minipage}[c]{0.5\textwidth}
					Die \underline{positive} Lösung wird hier dann als $n$-te Wurzel bezeichnet:
					\begin{align*}
					\sqrt[n]{\quad}\colon\R_0^+&\>\R_0^+\\
					x&\mapsto\sqrt[n]{x}
					\end{align*}
				\end{minipage}
				\begin{minipage}[c]{0.5\textwidth}
					\begin{tikzpicture}[scale=0.5]
					\draw[->]
					(-4.2,0) -- (4.2,0) node[right] {$x$};
					\draw[->]
					(0,-4.2) -- (0,4.2) node[above] {$f(x)$};
					\draw[color=gray, domain=-3:3,very thin]
					plot(\x,\x*\x*0.5) node[right]{$f(x)=x^n$};
					\draw[color=blue,domain=0:3]
					plot(\x*\x*0.5,\x) node[right]{$f(x)=\sqrt[n]{x}$};
					\draw[color=gray,domain=-3:3,dotted]
					plot(\x*\x*0.5,-\x);
					\end{tikzpicture}
				\end{minipage}
			\end{itemize}
		\end{itemize}
		\item[e)] \underline{Polynome}\\
		$a_0,\quad...\quad,a_n\in\R$ (Koeffizienten)\\
		Ein Polynom ist eine Funktion $p$ mit 
		\begin{align*}
			p\colon \R&\>\R\\
			x&\mapsto a_nx^n+a_{n-1}x^{n-1}+...+a_1x^1+a_0x^0=\sum_{k=0}^{n}{a_kx^k}
		\end{align*}
		Falls $a_n\neq 0$ ist, heißt $n$ \underline{Grad} des Polynoms.
		\item[f)] \underline{Rationale Funktionen}\\
		Rationale Funktionen sind Quotienten von Polynomen (mit $p,q$...Polynome):
		\begin{align*}
			f\colon D&\>\R\\
			x&\mapsto\frac{p(x)}{q(x)}
		\end{align*}
		mit $D=\{x\in\R|q(x)\neq 0\}$
		\item[g)] \underline{Exponentialfunktionen}\\
		Exponentialfunktionen sind Funktionen
		\begin{align*}
			f\colon\R&\>\R^+\\
			x&\mapsto q^x
		\end{align*} wobei die Basis $\R\ni q>0$, $q\neq 1$ vorgegeben ist.
		\begin{align*}
			q>1&\colon\textrm{$f$ steigt}\\
			0<q<1&\colon\textrm{$f$ fällt}
		\end{align*}
		Bekannte Rechenregeln:
		\begin{itemize}
			\item $q^x\* q^y=q^{x+y}$
			\item $\frac{q^x}{q^y}=q^{x-y}$
			\item $(q^x)^y=q^{x\* y}$
			\item $(p\* q)^x=p^x\* q^x$
			\item $(\frac{p}{q})^x=\frac{p^x}{q^x}$
		\end{itemize}
		Zur Beschreibung von Exponentialfunktionen genügt es, \underline{eine} bestimmte Basis zu benutzen (man kann $g(x)=p^x$ durch $f(x)=q^x$ ausdrücken, siehe Teil h).\\
		Früher: Basis 10\\
		Heute: Basis $\e\approx 2.781828...$ (Eulersche Zahl)\\
		Informatik: oft Basis 2\\
		\begin{minipage}[c]{0.5\textwidth}
			\begin{align*}
			\textrm{exp}\colon\R&\>\R^+\\
			x&\mapsto \e^x\\
			\\
			\textrm{exp}(0)&=\e^0=1\\
			\textrm{exp}(1)&=\e^1=2.781828...
			\end{align*}
		\end{minipage}
		\begin{minipage}[c]{0.5\textwidth}
			\begin{tikzpicture}[scale=0.5]
			\draw[->]
				(-4.2,0) -- (4.2,0) node[right] {$x$};
			\draw[->]
			 	(0,-0.2) -- (0,4.2) node[above] {$f(x)$};
			\draw[color=blue,domain=-3:1.5]
				plot(\x,{exp(\x)}) node[right]{$f(x)=\textrm{exp}(x)$};;
			\end{tikzpicture}
		\end{minipage}
		\item[h)] \underline{Logarithmen}\\
		Die Exponentialfunktion
		\begin{align*}
			\textrm{exp}(x)\colon\R&\>\R^+\\
			x&\mapsto \e^x
		\end{align*} ist bijektiv.\\
		Um sie umzukehren, muss zu gegebenem $y\in\R^+$ die Gleichung $\e^x=y$ gelöst werden.\\
		Die Lösung ist für $y>0$ in $\R$ eindeutig und wird als der \underline{natürliche Logarithmus} von $y$ bezeichnet: $x=\ln y$.\\
		In $\R$ ist die Gleichung für $y\leq 0$ unlösbar.\\
		\begin{minipage}[c]{0.5\textwidth}
			\begin{align*}
			\ln\colon\R^+&\>\R\\
			x&\mapsto\ln x
			\end{align*}
		\end{minipage}
		\begin{minipage}[c]{0.5\textwidth}
			\begin{tikzpicture}[scale=0.5]
			\draw[->]
			(-4.2,0) -- (4.2,0) node[right] {$x$};
			\draw[->]
			(0,-4.2) -- (0,4.2) node[above] {$f(x)$};
			\draw[color=gray, domain=-3:1.5,very thin]
			plot(\x,{exp(\x)}) node[right]{$f(x)=\textrm{exp}(x)$};
			\draw[color=blue,domain=-3:1.5]
			plot({exp(\x)},\x) node[right]{$f(x)=\ln x$};;
			\end{tikzpicture}
		\end{minipage}
		Analoges gilt für andere Exponentialfunktionen.
		\begin{align*}
			f:\R&\>\R^+\\
			x&\mapsto q^x\quad (q>0,q\neq 1)
		\end{align*}
		Es gilt: $q^x=y\Leftrightarrow x=\log_qy$ (Logarithmus zur Basis $q$).\\
		\\
		Es genügt wieder, \underline{eine} feste Basis zu betrachten, z.B. $\e$, denn $q^x=(\e^{\ln q})^x=\e^{x\*\ln q}$. Es gilt:
		\begin{align*}
			q^x=y&\Leftrightarrow\e^{x\*\ln q}=y\\
			&\Leftrightarrow\ln(\e^{x\*\ln q})=\ln y\\
			&\Leftrightarrow x\*\ln q=\ln y\\
			&\Leftrightarrow x=\frac{\ln y}{\ln q}\quad\textrm{,}
		\end{align*}
	also gilt $\log_qy=\frac{\ln y}{\ln q}$.\\
	\\
	\underline{Rechenregeln} für den Logarihmus lassen sich aus den Regeln für die Exponentialfunktion herleiten:\\
	\\
	Sei $u\coloneqq\ln x$, $v\coloneqq\ln y$, dann ist $x=\e^u$ und $y=\e^v$, daraus folgt 
	\begin{align*}
	x\* y=\e^u\*\e^v=\e^{u+v}\quad\textrm{,}
	\end{align*} also ist 
	\begin{align*}
	\ln(x\* y)=\ln(\e^{u+v})=u+v=\ln x+\ln y\quad\textrm{.}
	\end{align*}
	Genauso kann man mit beliebiger Basis $q>0$, $q\neq1$ verfahren, wir erhalten für jede Logarithmusfunktion $\log\colon\R^+\>\R$:
	\begin{itemize}
		\item $\log(x\*y)=\log x+\log y\quad\forall x,y>0$
		\item $\log(\frac{x}{y})=\log x-\log y\quad\forall x,y>0$
		\item $\log(x^\alpha)=\alpha\*\log x\quad\forall x>0,\alpha\in\R$
	\end{itemize}
	\item[i)] \underline{Trigonometrische Funktionen}\\
	Wir betrachten einen Punkt $P$ auf dem Einheitskreis (Kreis um $O$, Radius 1).\\
	\\
	\begin{minipage}[c]{0.5\textwidth}
		Der Winkel, der von der positiven $x_1$-Achse und der Geraden durch $O$ und $P$ eingeschlossen wird, sei $x$.
	\end{minipage}
	\begin{minipage}[c]{0.5\textwidth}
		\begin{tikzpicture}[scale=0.5]
		\draw[->]
		(-4.2,0) -- (4.2,0) node[right] {$x_1$};
		\draw[->]
		(0,-4.2) -- (0,4.2) node[above] {$x_2$};
		\draw (0,0) circle (100pt);
		\draw[color=gray, domain=-0.5:3,very thin]
		plot(\x,\x) node[right]{};
		\draw[color=blue, domain=-1:2.45, style=dotted]
		plot(2.45,\x);
		\draw[color=blue, domain=0:2.45, line width=2pt]
		plot(\x,-1) node[below left]{$\cos x$};
		\draw[color=red, domain=-1:2.45, style=dotted]
		plot(\x,2.45);
		\draw[color=red, domain=0:2.45, line width=2pt]
		plot(-1,\x) node[below left]{$\sin x$};
		\draw[fill=black](2.45,2.45) circle(3pt) node[above right]{$P$};
		\draw[fill=black](0,0) circle(3pt) node[below left]{$O$};
		\draw[fill=black](3.5,0) circle(3pt) node[below right]{1};
		\draw[very thin, color=gray, ->]
		(2, 0) node[above left]{$x$} arc (0:45:2);
		\end{tikzpicture}
	\end{minipage}\\
	Dann heißt die $x_1$-Koordinate von $P$ der \underline{Kosinus} von $x$ ($\cos x$), die $x_2$-Koordinate heißt der \underline{Sinus} von $x$ ($\sin x$).\\
	\\
	Der Winkel $x$ kann im Gradmaß oder im Bogenmaß (Länge des Bogens von $(1|0)$ bis $P$) gemessen werden, es gilt:
	\begin{align*}
		\frac{\textrm{Gradmaß}}{360^\circ}=\frac{\textrm{Bogenmaß}}{2\pi}
	\end{align*}
	So lassen sich die Funktionen $\cos$ und $\sin$ definieren:
	\begin{align*}
		\cos\colon\R&\>[-1;1]\\
		x&\mapsto\cos x\\
		\\
		\sin\colon\R&\>[-1;1]\\
		x&\mapsto\sin x
	\end{align*}
	und weiter
	\begin{align*}
		\tan x&\coloneqq\frac{\sin x}{\cos x}\qquad\textrm{(Tangens)}\qquad
		\textrm{und}\\
		\\
		\cot x&\coloneqq\frac{\cos x}{\sin x}\qquad\textrm{(Kotangens)}
	\end{align*}
	(Tangens und Kotangens sind jeweils nur dort definiert, wo der Nenner $\neq 0$ ist!)\\
	\begin{minipage}[c]{0.5\textwidth}
		\begin{tikzpicture}
			\draw[->]
			(-0.2,0) -- (4.2,0) node[right] {$x_1$};
			\draw[->]
			(0,-0.2) -- (0,4.2) node[above] {$x_2$};
			\draw
			(4,0) arc(0:90:4);
			\draw[domain=-0.1:4.5, very thin]
			plot(\x,0.75*\x);
			\draw[color=gray, very thin, ->]
			(2,0) node[above left]{$x$} arc(0:37:2);
			\draw[fill=black](3.2,2.4) circle(1.5pt);
			\draw[color=gray, very thin, domain=0:4, <->]
			(-1.5,0) -- (-1.5,2) node[left]{1} -- (-1.5,4);
			\draw[color=gray, very thin, domain=0:4, <->]
			(0,-1) -- (2,-1) node[below]{1} -- (4,-1);		
			\draw[style=dotted]
			(4,-0.5) -- (4,4);
			\draw[color=green, line width=2pt]
			(4,0) -- (4,1.5) node[right]{$\tan x$} -- (4,3);
			\draw[color=blue]
			(0,-0.3) -- (1.6,-0.3) node[below]{$\cos x$} -- (3.2,-0.3);
			\draw[color=red]
			(-0.3,0) -- (-0.3,1) node[left]{$\sin x$} -- (-0.3,2.4);
		\end{tikzpicture}
	\end{minipage}
	\begin{minipage}[c]{0.5\textwidth}
		Strahlensatz: $\frac{\sin x}{\cos x}=\frac{\tan x}{1}$\\
		\\
		Wertetabelle: s. PÜ 02
	\end{minipage}
	Graphen:\\
	\begin{tikzpicture}
		\draw[->]
		(-6,0) -- (7,0) node[right] {$x$};
		\draw[->]
		(0,-2) -- (0,2) node[above] {$f(x)$};
		\draw
		(0,1) node[left]{$1$}
		(0,-1) node[left]{$-1$}
		(-6.283,0) node[below]{$-2\pi$}
		(-4.65,0) node[below]{$-\frac{3}{2}\pi$}
		(-3.1415,0) node[below]{$-\pi$}
		(-1.57,0) node[below]{$-\frac{1}{2}\pi$}
		(1.57,0) node[below]{$\frac{1}{2}\pi$}
		(3.1415,0) node[below]{$\pi$}
		(4.65,0) node[below]{$\frac{3}{2}\pi$}
		(6.283,0) node[below]{$2\pi$};
		\draw[color=red, domain=-6:6, samples=50]
		plot(\x,{sin(\x r)}) node[below left]{$\sin x$};
		\draw[color=blue, domain=-6:6, samples=50]
		plot(\x,{cos(\x r)}) node[above right]{$\cos x$};
		\draw[color=green, domain=-1.2:1.2]
		plot(\x,{tan(\x r)});
		\draw[style=dotted, color=green]
		(-1.57,-3)--(-1.57,3);
		\draw[color=green, domain=1.95:4.35]
		plot(\x,{tan(\x r)}) node[right]{$\tan x$};
		\draw[style=dotted, color=green]
		(1.57,-3)--(1.57,3);
		\draw[color=green, domain=-4.35:-1.95]
		plot(\x,{tan(\x r)});
	\end{tikzpicture}\\
	Additionstheoreme:
	\begin{align*}
		\sin (x+y)&=\sin x\*\cos y+\cos x\* \sin y\\
		\cos(x+y)&=\cos x\*\cos y-\sin x\*\sin y\\
		(\sin x)^2+(\cos x)^2&=\sin^2x+\cos^2x=1\qquad\textrm{(Satz des Pythagoras)}
	\end{align*}
	\\
	Es gilt: $\cos x=\sin (x+\frac{\pi}{2})$ (Verschiebung um $\frac{\pi}{2}$).\\
	\\
	$\sin$ und $\cos$ sind $2\pi$-periodisch, d.h.
	\begin{align*}
		\sin x&=\sin(x+2\pi)\qquad\forall x\\
		\cos x&=\cos (x+2\pi)\qquad\forall x
	\end{align*}
	$\tan$ ist $\pi$-periodisch:
	\begin{align*}
		\tan x&=\tan(x+\pi)\qquad\forall x\textrm{ aus Definitionsbereich}
	\end{align*}
	\end{itemize}
	\newpage
	\section{Folgen}
	\subsection{Definition: Folge}
	\subsubsection*{Definition}
	Eine \underline{Folge} $(a_n)_{n\in\N}$ ist eine Abbildung von der Menge der natürlichen Zahlen $\N$ in eine Menge $M$ (oft $M\subset \R$).\\
	Die $a_n$ ($n=1,2,3,...$) heißen \underline{Glieder} der Folge, $n$ heißt \underline{Index}.\\
	(Bemerkung: Das 1. Glied der Folge muss nicht $a_1$ sein. durch Umbenennung, z.B. $b_1\coloneqq a_7, b_2\coloneqq a_8$, ist auch $(a_7, a_8, a_9, ...)$ eine Folge im sinne der Definition 2.1)
	\subsubsection*{Schreibweisen}
	\begin{align*}
		&(a_n)_{n\in\N}\\
		&(a_n)_{n\geq n_0}\qquad \textrm{(z.B. $(a_n)_{n\geq 7}$) oder nur}\\
		&(a_n)
	\end{align*}
	\subsection{Beispiel}
	\begin{itemize}
		\item[a)] $a_n=c\qquad\forall n\geq 1, c\in\R\textrm{ konstant}$\\
		$(a_n)_{n\in\N}=(c)_n\qquad (c,c,c,c,...)$
		\item[b)] $a_n=n\qquad (1, 2, 3,4,...)$
		\item[c)] $a_n=(-1)^n\qquad (-1,1,-1,1,-1,...)$
		\item[d)] $a_n=\frac{1}{n}\qquad(1,\frac{1}{2},\frac{1}{3},\frac{1}{4},...)$
		\item[e)] $a_n=[0,\frac{2}{n})\qquad$ Folge von Intervallen
		\item[f)] $a_n$ rekursiv definiert:\\
		\begin{align*}
			a_1&\coloneqq 1\\
			a_{n+1}&\coloneqq(n+1)a_n\qquad(n\geq 1)\\
			a_2&=2\*a_1=2\\
			a_3&=3\*a_2=6\\
			a_4&=4\*a_3=24
		\end{align*}
	\end{itemize}
	\subsection{Definition: Eigenschaften von Folgen}
	Eine Folge $(a_n)_{n\in\N}$ reeller Zahlen heißt
	\begin{itemize}
		\item[a)] \underline{beschränkt}, wenn die Menge der Folgenglieder beschränkt ist (s. Mathe 1), d.h. wenn es eine Zahl $K\geq 0$ gibt mit $|a_n|\leq K\quad\forall n\in\N$ (d.h. alle Folgenglieder liegen im Intervall $[-K,K]\quad\forall n;\quad(-K\leq a_n\leq K)$).
		\item[b)] \underline{alternierend}, falls ihre Glieder abwechselnd positiv und negativ sind.
	\end{itemize}
	\subsection{Beispiel}
	Beispiele aus 2.2:\\
	beschränkt: a), c), d) [für c) und d) z.B. K=1] \\
	alternierend: c)
	\subsection{Definition: Konvergenz}
	\begin{itemize}
		\item[a)] Eine Folge $(a_n)_{n\in\N}$ reeller Zahlen heißt \underline{konvergent gegen $a\in\R$}, wenn es zu jeder positiven Zahl $\epsilon>0$ ein $N\in\N$ gibt (das von $\epsilon$ abhängen darf), so dass gilt: $|a_n-a|<\epsilon$ für alle $n\geq N$.\\
		(kurz: $\forall\epsilon>0\quad\exists N\in\N\quad\forall n\geq N\colon|a_n-a|<\epsilon$)
		\item[b)] Die Zahl a heißt dann \underline{Grenzwert} oder \underline{Limes} der Folge, wir schreiben:\\
		 $\lim\limits_{\infn}a_n=a$ oder\\
		 $a_n\> a$ für $\infn$ ($a_n$ strebt gegen $a$)
		 \item[c)] Eine Folge, die gegen $0$ konvergiert, heißt \underline{Nullfolge}.
		 \item[d)] Eine Folge, die nicht konvergiert, heißt \underline{divergent} (die Folge divergiert).
	\end{itemize}
	\subsection{Bemerkung}
	$\>$ Folien 20.04.16\\
	\subsection{Beispiel}
	\begin{itemize}
		\item[a)] $a_n=\frac{1}{n}$ ist Nulfolge, d.h. $\lim\limits_{\infn}\frac{1}{n}=a=0$, denn:\\
		Sei $\epsilon>0$ beliebig. Dann wähle $N$ als $N>\frac{1}{\epsilon}$, denn damit gilt für alle $a_n$ mit $n\geq N$:\\
		$|a_n-0|=|\frac{1}{n}-0|=\frac{1}{n}\leq\frac{1}{N}$, da $n\geq N$ und $\frac{1}{N}<\frac{1}{\frac{1}{\epsilon}}=\epsilon\Rightarrow|a_n-0|<\epsilon$.\\
		(z.B. falls $\epsilon=\frac{1}{10}$, wähle $N>10$, z.B. $N=11$; ab $a_{11}$ haben alle Folgenglieder einen Abstand $<\frac{1}{10}$ von 0)
		\item[b)] $(a_n)$ mit $a_n=\frac{n+1}{3n}$. Behauptung: $a=\frac{1}{3}$.\\
		Beweis: Sei $\epsilon>0$ beliebig. Dann wähle $N>\frac{1}{3\epsilon}$. Für alle $a_n$ mit $n\geq N$ gilt dann:\\
		$|a_n-a|=|\frac{n+1}{3n}-\frac{1}{3}|=|\frac{n+1-n}{3n}|=\frac{1}{3n}<\frac{1}{3N}<\epsilon$. $\frac{1}{3N}<\epsilon$ genau dann, wenn $N>\frac{1}{3\epsilon}$.
		\item[c)] $(a_n)_{n\in\N}$ mit $a_n=c\quad\forall n$.\\
		$\lim\limits_{\infn}a_n=c$\\
		Sei $\epsilon>0$ beliebig. Dann ist\\
		$|a_n-c|=|c-c|=0<\epsilon\quad\forall n\geq1$, hier ist also $N=1$, hängt nicht von $\epsilon$ ab, untypisch.
	\end{itemize}
	\subsection{Bemerkung}
	$N$ muss nicht optimal gewählt werden.\\
	Beispiel: $\lim\limits_{\infn}\frac{1}{n^3+n+5}=0$, [...]\\
	$|\frac{1}{n^3+n+5}-0|=\frac{1}{n^3+n^+5}\leq\frac{1}{N^3+N+5}\overset{!}{<}\epsilon$. Für optimales $N$: $\frac{1}{N^3+N+5}<\epsilon$ nach $N$ auflösen, schwer.\\
	Deshalb grob abschätzen, z.B. so:\\
	$\frac{1}{N^3+N+5}<\frac{1}{N}<\epsilon$, also wähle $N>\frac{1}{\epsilon}$.
	\subsection{Satz: Beschränktheit von Folgen}
	Jede konvergente folge ist beschränkt.\\
	\\
	Beweis: (zu zeigen: $(a_n)$ konvergente Folge: $\exists K\in\N$, so dass $|a_n|\leq K\quad\forall n\in\N$)\\
	Sei $(a_n)_{n\in\N}$ konvergent gegen $a$.\\
	dann existiert für alle $\epsilon>0$, also auch speziell für $\epsilon=1$, ein $N\in\N$ mit $|a_n-a|<1\quad\forall b\geq N$.\\
	Also gilt für alle $n\geq N$:\\
	\begin{align*}
		|a_n|&=|a_n+a-a|&\leq |a_n-a|+|a|\\
		&\textrm{'Einschiebetrick'} &\textrm{Dreiecksungleichung}\\
		|a_n|&&<1+|a|
	\end{align*}
	(also für $n\geq N$ sind die $|a_n|<1+|a|$; aber für $n=1,2,3,..., N-1$?)\\
	Definiere $K$ als $K\coloneqq\max\{|a_1|,|a_2|,|a_3|,...,|a_{N-1}|,1+|a|\}$\\
	Dann gilt $|a_n|\leq K\quad\forall n$.\\
	(Anmerkung: Durch den vorletzten Schritt ist meist $K\in\R^+$.)\\
	\subsection{Bemerkung}
	Nach 2.9 gilt:\\
	 $(a_n)$ konvergiert $\Rightarrow$ $(a_n)$ ist beschränkt\\
	Das ist äquivalent zu:\\
	$(a_n)$ ist nicht beschränkt $\Rightarrow$ $(a_n)$ konvergiert nicht\\
	(Kontraposition). Unbeschränkte Folgen sind also immer divergent.\\
	Bsp. $(a_n)$ mit $a_n=n$
	\subsection{Wichtiges Beispiel (geometrische Folgen)}
	Für $q\in\R$ gilt: 
	$\lim\limits_{\infn}q^n=\begin{cases}
	0\textrm{, falls }|q|<1\\
	1\textrm{, falls }|q|=1
	\end{cases}$\\
	Die Folge $(q^n)_n\in\N$ divergiert, falls $q=-1$ oder $|q|>1$.\\
	Beweis: \begin{itemize}
		\item[1. Fall] $|q|<1$ (zu zeigen $q^n\> 0$ für $\infn$)\\
		Sei $\epsilon>0$ beliebig. Dann ist
		\begin{align*}
			|q^n-0|=|q^n|=|q|^n&<\epsilon\\
			\Leftrightarrow n\*\ln|q|&<\ln\epsilon\\
			\Leftrightarrow n \qquad &\mathclap{\overset{\textrm{da}|q|<1}{\geq}}\qquad \frac{\ln\epsilon}{\ln|q|}
		\end{align*}
		Wähle $\N\ni N>\frac{\ln\epsilon}{\ln|q|}$, dann ist also $|q|^n<\epsilon\quad\forall n\geq N$.
		\item[2. Fall] $q=1$ $\>$ konstante 1-Folge, konvergiert, s. 2.7 c)
		\item[3. Fall] $|q|\geq 1, q\neq 1$\\
		Für $|q|>1$ ist $(q^n)$ unbeschränkt, also divergent (s. 2,9/2.10).\\
		Für $q=-1$: können wir erst später beweisen ($\>$ Cauchy-Folgen)
	\end{itemize}
	\subsection{Beispiel}
	Nach 2.11 sind die Folgen $((\frac{1}{2})^n)_{n\in\N}=(\frac{1}{2^n})_{n\in\N},\quad ((-\frac{7}{8})^n)_n\in\N$ Nullfolgen.
	\subsection{Satz: Rechenregeln für konvergente Folgen}
	Seien $(a_n), (b_n)$ reelle Folgen mit $\lim\limits_{\infn}a_n=a$ und $\lim\limits_{\infn}b_n=b$. Dann gilt:
	\begin{itemize}
		\item[a)] Die Folge $(c\*a_n)$ konvergiert gegen $c\*a, c\in\R$.
		\item[b)] Die Folge $(a_n\pm b_n)$ konvergiert gegen $a\pm b$.
		\item[c)] Die Folge $(a_n\*b_n)$ konvergiert gegen $a\*b$.
		\item[d)] Die Folge $(\frac{a_n}{b_n})$ konvergiert gegen $\frac{a}{b}$, falls $b_n, b\neq0$ und $|a_n|\>|a|$.
	\end{itemize}
	Seien weiter $(d_n), (e_n)$ reelle Folgen mit $\lim\limits_{\infn}d_n=0$, dann gilt:
	\begin{itemize}
		\item[e)] Ist $(e_n)$ beschränkt, dann ist $(d_n\*e_n)$ auch eine Nullfolge.
		\item[f)] Gilt $|e_n|\leq d_n\quad\forall n$, so ist $(e_n)$ auch eine Nullfolge.
	\end{itemize}
	Beweis [exemplarisch für a) und b), Rest s. Moodle]:\\
	\begin{itemize}
		\item[a)] Falls $c=0$: klar, konstante 0-Folge.\\
		Falls $c\neq 0$: Sei $\epsilon>0$ beliebig. Dann existiert $N\in\N$, so dass $|a_n-a|<\frac{\epsilon}{|c|}\quad\forall n\in\N$ (denn $a_n\> a$)\\
		Dann ist aber $|c\*a_n-c\*a|=|c\*(a_n-a)|=|c|\*\overbrace{|a_n-a|}^{<\frac{\epsilon}{|c|}}<\epsilon\quad\forall n\geq N$, also $c\*a_n\> c\*a$
		\item[b)] Sei $\epsilon>0$ beliebig.\\
		Dann $\exists N_1\in\N$, so dass $|a_n-a|<\frac{\epsilon}{2}\quad\forall n\geq N_1$ (denn $a_n\> a$)\\
		und $\exists N_2\in\N$, so dass $|b_n-b|<\frac{\epsilon}{2}\quad\forall n\geq N_2$ (denn $b_n\> b$).\\
		Dann gilt:
		\begin{align*}
			|(a_n+b_n)-(a+b)&|=|\overbrace{(a_n-a)}^{<\frac{\epsilon}{2}}+\overbrace{(b_n-b)}^{<\frac{\epsilon}{2}}|\overset{\triangle\textrm{-Ungleichung}}{\leq}|a_n-a|+|b_n-b|\\
			&<\frac{\epsilon}{2}+\frac{\epsilon}{2}=\epsilon\quad\forall n\geq N_1\textrm{ und }N_2
		\end{align*}
		(also z.B. für $n\geq N\coloneqq\max\{N_1,N_2\}$).\\
		\\
		Also gilt $(a_n+b_n)\> a+b$.\hfill$\square$
	\end{itemize}
	\subsection{Beispiel}
	\begin{itemize}
		\item[a)] $\frac{(-1)^n+5}{n}\>0$ für $\infn$, denn $\frac{1}{n}\>0$ für $\infn$ und $(-1)^n+5$ ist beschränkt: $|(-1)^n+5|\leq6\quad\forall n\in\N$ (nach 2.13 d)
		\item[b)] $\frac{3n^2-2n+1}{-n^2+n}\>-3$ für $\infn$, denn\\ $\frac{3n^2-2n+1}{-n^2+n}=\frac{n^2\*(3-\frac{2}{n}+\frac{1}{n^2})}{n^2\*(-1+\frac{1}{n})}=\frac{3-\bmark{\frac{2}{n}}+\bmark{\frac{1}{n^2}}}{-1+\bmark{\frac{1}{n}}}\quad\gmark{\frac{\>3\textrm{ für }\infn}{\>-1\textrm{ für }\infn}}\longrightarrow\frac{3}{-1}$ für $\infn$ (nach 2.13 b,d)\begin{tiny}
			[\bmark{Nullfolgen}]
		\end{tiny}
		\item[c)] \underline{Wichtiges Beispiel}\\
		Sei $x\in\R$ mit $|x|<1$, d.h. $|x|=\frac{1}{1+t}$ mit $t>0$.\\
		Sei $k\in\N_0$. Dann ist $\lim\limits_{\infn}(n^k\*x^n)=0$, denn
		\begin{align*}
			(1+t)^n\quad&\qquad\mathclap{\overset{\overset{\textrm{Mathe 1: 7.17}}{\textrm{bin. Lehrsatz}}}{=}}\qquad\quad\sum_{j=0}^{n}[\binom{n}{j}\*1^{n-j}\*t^j]\\
			&\qquad\mathclap{=}\overbrace{1}^{j=0}+\overbrace{nt}^{j=1}+\overbrace{\frac{n\*(n-1)}{2!}t^2}^{j=2}+\overbrace{\frac{n\*(n-1)\*(n-2)}{3!}t^3}^{j=3}+...\\
			&\qquad\mathclap{\overset{\overset{\textrm{nur Term}}{j=k+1}}{\geq}}\quad\frac{n\*(n-1)\*(n-2)\*...\*(n-k)}{(k+1)!}t^{k+1}=\binom{n}{k+1}t^{k+1}
		\end{align*}
		Damit gilt:
		\begin{align*}
			|n^k\*x^n|=|\frac{n^k}{(1+t)^n}|\leq\frac{n^k}{\binom{n}{k+1}t^{k+1}}=\frac{n^k}{n^{k+1}+...}\> 0
		\end{align*}
		für $\infn$.\\
		Es gilt also z.B. ($k=10000, x=\frac{1}{2}$): $\frac{n^{10000}}{2^{n}} \> 0$ für $\infn$\\
		$\Rightarrow \overset{\textrm{Exponentialfkt.}}{(1+t)^n}$ wächst schneller als jede Potenz $\overset{Polynom}{n^k}$ !
	\end{itemize}
	\subsection{Anmerkung (Landau-Symbole, $\mathcal{O}$-Notation)}
	(Informatik, VL Algorithmen)\\
	Sei $(a_n)$ eine \underline{strikt positive} Folge, d.h. $a_n>0\quad\forall n\in\N$. Dann ist
	\begin{itemize}
		\item[a)] $\mathcal{O}(a_n)=\mathcal{O}((a_n))=\{(b_n)|(\frac{b_n}{a_n})\textrm{ ist beschränkt }\}$ ("Menge aller Folgen, für die ... gilt")
		\item[b)] $o(a_n)=\{(b_n)|\frac{b_n}{a_n}\textrm{ ist Nullfolge }\}$ ($(a_n)$ wächst schneller als $(b_n)$)
	\end{itemize}
	$\mathcal{O},o\colon $ Landau-Symbole
	\begin{itemize}
		\item[c)] $(a_n)\sim(b_n)$, falls $\lim\limits_{\infn}(\frac{a_n}{b_n})_n=1$
	\end{itemize}
	Beispiel:
	\begin{itemize}
		\item $(2n^2+5n+1)_n\in\mathcal{O}(n^2)$, denn\\
		$(\frac{2n^2+5n+1}{n^2})=\frac{n^2\*(2+\frac{5}{n}+\frac{1}{n^2})}{n^2}\>2$ für $\infn$, beschränkt
		\item $(n^2)\in o(n^3)$
		\item $(n^3)\in o(2^n)$
		\item $(n^3-3)\sim(n^3)$, denn $(\frac{n^3}{n^3-3})=(\frac{n^3\*(1)}{n^3\*(1-\frac{3}{n^3})})\>1$ für $\infn$
		\item häufig auch laxe Schreibweise
		\begin{align*}
			2n^2+5n+1&=\mathcal{O}(n^2)\\
			n^2&=o(n^3)
		\end{align*}
	\end{itemize}
	Außerdem:
	\begin{align*}
		\mathcal{O}(1)&=\textrm{ Menge der beschränkten Folgen}\\
		o(1)&=\textrm{ Menge der Nullfolgen}
	\end{align*}
	Wichtige Formel: \underline{Stirling}: $(n!)\sim(\sqrt{2\pi n}(\frac{n}{\e})^n)$\\
	\\
	Problem: Wie zeigt man die Konvergenz einer Folge, wenn man den Grenzwert nicht kennt?
	\subsection{Definition}
	Eine Folge reeller Zahle $(a_n)_n$ heißt
	\begin{itemize}
		\item[a)] \underline{(\bmark{streng}) monoton steigend/wachsend}, falls $a_{n+1}\overset{\bmark{>}}{\geq}a_n\quad\forall n\in\N$, Schreibweise: $(a_n)\nearrow$
		\item[b)] \underline{(\bmark{streng}) monoton fallend} $(a_n)\searrow$, falls $a_{n+1}\overset{\bmark{<}}{\leq}a_n\quad\forall n\in\N$
		\item[c)] \underline{monoton}, falls a) oder b) gilt (oder beides)
	\end{itemize}	
	\subsection{Beispiel}
	\begin{itemize}
		\item $(a_n)=(\frac{1}{n})$ ist streng monoton fallend
		\item $(a_n)=(1)$ ist monoton fallend und monoton steigend
		\item $(a_n)=((-1)^n)$ ist nicht monoton
	\end{itemize}
	\subsection{Bemerkung}
	$(a_n)\nearrow$ zeigt man so:
	\begin{align*}
		a_{n-1}-a_n&\geq0\quad\textrm{ oder }\\
		\frac{a_{n+1}}{a_n}&\geq1 
	\end{align*}
	\subsection{Satz (Monotone Konvergenz)}
	Jede beschränkte, monotone Folge reeller Zahlen $(a_n)_n$ konvergiert, und zwar gegen
	\begin{itemize}
		\item $\sup\{a_n\colon n\in\N\}$, falls $(a_n)$ monoton steigend oder gegen
		\item $\inf\{a_n\colon n\in\N\}$, falls $(a_n)$ monoton fallend ist.
	\end{itemize}
	Beweis:\\
	\\
	Sei $(a_n)\nearrow$ und beschränkt.
	\begin{align*}
		\Rightarrow \{a_n\colon n\in\N\}&\subseteq\R\quad\textrm{ ist beschränkt}\\
		\overset{\overset{\textrm{Vollst.-Axiom}}{\textrm{Mathe 1, 8.16}}}{\Rightarrow}S&\coloneqq\sup\{a_n\colon n\in\N\}\quad\textrm{ existiert.}
	\end{align*}
	Wir zeigen: $a_n\> S$ für $\infn$.\\
	Sei $\epsilon>0$ beliebig. Zu zeigen ist $\exists N\in\N$ mit $|a_n-S|<\epsilon\quad\forall n\geq N$.\\
	Es gilt $a_n\leq S\quad\forall n\in\N$, also zu zeigen: $S-a_n<\epsilon\quad\forall n\geq N$.\\
	$S$ ist \underline{kleinste} obere Schranke, d.h. $S-\epsilon$ ist \underline{keine} obere Schranke\\
	\begin{align*}
		\Rightarrow\exists N\in\N\quad\textrm{ mit }\quad a_n&>S-\epsilon\quad\forall n\geq N\\
		\Rightarrow S-a_n&<\epsilon\qquad\forall n\geq N
	\end{align*}
	$(a_n)\searrow$ analog\hfill$\square$
	\subsection{Beispiel}
	\begin{itemize}
		\item[a)] $x\in\R^+$, dann $(x^n)\in o(n!)\qquad(x^n=o(n!))$, d.h. $a_n=\frac{x^n}{n!}\> 0$ für $\infn$
		\begin{itemize}
			\item $a_n>0$
			\item $\frac{a_{n+1}}{a_n}=\frac{x^{n+1}\*n!}{(n+1)!\*x^n}=\frac{x}{n+1}\leq 1$ für $n+1\geq x$, also gilt $a_{n+1}\leq a_n$, d.h. $(a_n)\searrow$ und $(a_n)$ ist beschränkt
			\item $\inf\{a_n\colon n\in\N\}=0$
		\end{itemize}
		\item[b)] \underline{wichtige Folge}\\
		\begin{align*}
			(a_n)_{n\in\N}&=((a+\frac{1}{n})^n)_{n\in\N}\\
			\lim\limits_{\infn}(a_n)&=\e\qquad\textrm{ (Eulersche Zahl, $\e=2,71828...$)}
		\end{align*}
		Warum existiert dieser Limes?\\
		Zeige: $(a_n)\nearrow$ und $(a_n)$ beschränkt, benutze Satz 2.19
		\begin{itemize}
			\item $(a_n)\nearrow$
			\begin{align*}
				\frac{a_n}{a_{n+1}}&=(\frac{1+n}{n})^n\*(\frac{n-1}{n})^{n-1}=(\frac{n+1}{n})^n\*(\frac{n-1}{n})^n\*(\frac{n-1}{n})^{-1}\geq 1\\
				&=(\frac{n^2-1}{n^2})^n\*\frac{n}{n-1}\geq1\\
				&=(1-\frac{1}{n^2})^n\*\frac{n}{n-1}\geq 1
			\end{align*}
			Benutze die \underline{Bernoulli-Ungleichung}, für $h\in\R,\quad n\in\N$ gilt $(1+h)^n\geq 1+nh$ für $h\geq -1$ (hier: $h=-\frac{1}{n^2}$)
			\begin{align*}
				\frac{a_n}{a_{n-1}}&=(1-\frac{1}{n^2})^n\*\frac{n}{n-1}\geq (1-n\*\frac{1}{n^2})\*\frac{n}{n-1}\\
				&=(1-\frac{1}{n})\*\frac{n}{n-1}=1\qquad\textrm{ , }
			\end{align*}
			also $(a_n)\nearrow$
			\item $(a_n)$ beschränkt: Übung, benutze wieder Bernoulli
		\end{itemize}
	\end{itemize}
	\subsection{Satz (Intervallschachtelungsprinzip)}
	Seien $(a_n),\quad(b_n)$ reelle Folgen mit
	\begin{itemize}
		\item $(a_n)\nearrow$ (= linke Intervallgrenze)
		\item $(b_n)\searrow$ (= rechte Intervallgrenze)
		\item $a_n\leq b_n\quad\forall n\in\N$
		\item $b_n-a_n\> 0$ für $\infn$
	\end{itemize}
	Dann sind beide Folgen konvergent und besitzen denselben Limes.\\
	\\
	Beweis:\\
	\\
	$(a_n),\quad(b_n)$ konvergent nach Satz 2.19, denn
	\begin{itemize}
		\item $(a_n)\nearrow$; $(a_n)$ beschränkt, da $a_n\leq b_n\quad\forall n\in\N$, also gilt auch $a_n\leq b$ (alle anderen $b_n$ sind noch kleiner)
		\item $(b_n)\searrow$; $(b_n)$ beschränkt, da $b_n\geq a_n\quad\forall n\in\N$, also $b_n\geq a_n\geq a_1$
		\item Da $(b_n)-(a_n)$ Nulfolge ist, sind auch die Grenzwerte gleich.
	\end{itemize}\hfill$\square$
	\subsection{Beispiel (vgl. Beispiel 2.20 b))}
	$a_n=(1+\frac{1}{n})^n$, $b_n=(1+\frac{1}{n})^{n+1}$\\
	Man kann zeigen: $(a_n)\nearrow$, $(b_n)\searrow$\\
	$a_n\leq b_n$, $b_n-a_n\> 0$, also $\exists\lim\limits_{\infn}(1+\frac{1}{n})^n=\lim\limits_{\infn}(1+\frac{1}{n})^{n+1}$\\
	\\
	Ähnlich zeigt man $\lim\limits_{\infn}(1+\frac{x}{n})^n$ existiert $\forall x\in\R$\\
	So definiert man $\e^x\coloneqq\lim\limits_{\infn}(1+\frac{x}{n})^n$\\
	\\
	\underline{Bisher:}\\
	$(a_n)$ konvergiert $\Rightarrow (a_n)$ beschränkt, Umkehrung gilt nicht; z.B. $((-1)^n)$\\
	Allerdings besitzt diese Folge zwei konvergente Teilfolgen mit $\lim=+1$ und $\lim=-1$.
	\subsection{Definition}
	Sei $(a_n)_{n\in\N}$ eine Folge und $(n_k)_{k\in\N}$ ($n_1,n_2,...$) eine streng monoton steigende Folge von Indizes (d.h. $n_1<n_2<n_3<...$).\\
	Dann heißt die Folge $(a_{n_k})_{k\in\N}$ \underline{Teilfolge} von $(a_n)_{n\in\N}$ ("Teilfolgen entstehen durch Streichung von Gliedern").
	\subsection{Beispiel}
	\begin{align*}
		(a_n)&=((-1)^n)\\
		n_k&\coloneqq 2n\quad\textrm{ ergibt  }(n_1=2;\ n_2=4\ n_3=6)\\
		&a_n=1\quad\forall n\in\N\quad\textrm{ (Teilfolge 1,1,1,1...)}\\
		n_k&\coloneqq 2n-1\quad\textrm{ ergibt (Teilfolge -1,-1,-1,...)}\\
		&a_{2n-1}=-1\quad\forall n\in\N
	\end{align*}
	\subsection{Bemerkung}
	Es gilt: $(a_n)$ konvergiert gegen $a$ $\Rightarrow$ jede Teilfolge von $(a_n)$ konvergiert gegen $a$.
	\subsection{Definition}
	Sei $(a_n)$ eine reelle Folge.
	Eine Zahl $h\in\R$ heißt \underline{Häufungspunkt} von $(a_n)$, wenn es eine Teilfolge von $(a_N)$ gibt, die gegen $h$ konvergiert.
	\subsection{Beispiel}
	\begin{itemize}
		\item $(a_n)=((-1)^n+\frac{1}{n})$ besitzt zwei Häufungspunkte $-1$ und $1$
		\item $(a_n)=((-1)^n)$ besitzt die Häufungspunkte $-1$ und $1$
	\end{itemize}
	\subsection{Satz (Satz von Bolzano-Weierstraß)}
	Sei $(a_n)$ eine reelle Folge. Dann gilt:
	\begin{align*}
		(a_n)\textrm{  beschränkt  }\Rightarrow(a_n)\textrm{  besitzt eine konvergente Teilfolge  }
	\end{align*}
	Beweis: Intervallschachtelungsprinzip/Bisektionsverfahren\\
	(s. Folien/Blatt[$\leftarrow$s.u.])
	\\
	Wir verwenden das Intervallschachtelungsprinzip (Satz 2.21). Nach Voraussetzung ist $(a_n)_{n\in\N}$ beschränkt, es existiert also ein $K\in\N$, so dass alle Folgeglieder im Intervall $[-K,K]\eqqcolon[A_0,B_0]$ liegen. Halbiere dieses Intervall:
	\begin{itemize}
		\item Falls in der ersten Hälfte des Intervalls unendlich viele Folgenglieder liegen: wähle eines davon aus.
		\item Falls nicht (also falls nur endlich viele Folgenglieder in der ersten Hälfte des Intervalls liegen), dann liegen in der zweiten Hälfte unendlich viele Folgenglieder. Wähle davon eines aus.
	\end{itemize}
	Das ausgewählte Folgenglied nennen wir $a_{n1}$, die Intervallhälfte, aus der es stammt, nennen wir $[A_1,B_1]$. Fahre nun so fort: Halbiere $[A_1,B_1]$, wähle wie oben $a_{n2}$ aus, erhalte damit das Intervall $[A_2,B_2]$, usw. So erhalten wir eine Teilfolge $(a_{n_k})_{k\in\N}$. Für die Intervalgrenzen von $[A_k,B_k]$ gilt:
	\begin{itemize}
		\item $A_k\leq a_{n_k}\leq B_k$
		\item $(A_k)_{k\in\N}\nearrow,\qquad(B_k)_{k\in\N}\searrow$
		\item $A_k\leq B_k$
		\item $B_k-A_k\> 0$ für $k\>\infty$. 
	\end{itemize}
	Damit sind alle Voraussetzungen für Satz 2.21 (Intervallschachtelungsprinzip) erfüllt. Die Folgen $(A_k)_{k\in\N}$ und $(B_k)_{k\in\N}$ sind also konvergent und besitzen denselben Limes $a$. Damit gilt auch $a_{n_k}\> a$ für $k\>\infty$.\hfill$\square$
	\subsection{Bemerkung/Definition}
	Sei $(a_n)$ reell und beschränkt, dann gibt es einen größten und einen kleinsten Häufungspunkt, den
	\begin{itemize}
		\item \underline{Limes superior} von $(a_n)$: $\limsuperior{\infn}a_n$ oder $\limsup{\infn}a_n$ bzw. den
		\item \underline{Limes inferior} von $(a_n)$: $\liminferior{\infn}a_n$ oder $\liminf{\infn}a_n$.
	\end{itemize}
	Weiter setzt man
	\begin{itemize}
		\item $\limsup{\infn}a_n\coloneqq\begin{cases}
			\infty\textrm{, wenn }(a_n)\textrm{ nicht nach oben beschränkt ist}\\
			-\infty\textrm{, wenn }(a_n)\>-\infty\textrm{ gilt, d.h. }\forall K>0\quad\exists N\in\N\colon a_n\leq-K\quad\forall n\geq N 
		\end{cases}$
		\item $\liminf{\infn}a_n\coloneqq\begin{cases}
		-\infty\textrm{, wenn }(a_n)\textrm{ nicht nach unten beschränkt ist}\\
		\infty\textrm{, wenn }(a_n)\>\infty\textrm{ gilt, d.h. }\forall K>0\quad\exists N\in\N\colon a_n\geq K\quad\forall n\geq N
		\end{cases}$
	\end{itemize}
	\underline{Achtung:} $-\infty,\infty$ sind keine reellen Zahlen!\\
	Man erweitert hier $\R$ um zwei ideelle Elemente $-\infty,\infty$, setzt $\overline{\R}=\R\cup\{\infty,-\infty\}$ (Abschluss von $\R$) und erweitert die Ordnungsstruktur auf $\R$ durch\\ $-\infty<x<\infty\quad\forall x\in\R$.\\
	\\
	Mit dieser Festlegung besitzt \underline{jede} reelle Zahlenfolge sowohl $\limsuperior{}$ als auch $\liminferior{}$.\\
	Beispiel:
	\begin{itemize}
		\item[a)] $a_n=\frac{n+1}{n}\qquad\limsup{\infn}a_n=\liminf{\infn}a_n=1$
		\item[b)] $a_n=(-1)^n\qquad\limsup{\infn}a_n=1\qquad\liminf{\infn}a_n=-1$
		\item[c)] $a_n=(-1)^n\*n\qquad\limsup{\infn}a_n=\infty\qquad\liminf{\infn}a_n=-\infty$
		\item[d)] $a_n=n\*(1+(-1)^n)$ : Übung
	\end{itemize}
	\subsection{Definition (Cauchyfolge)}
	Eine Folge $(a_n)$ heißt \underline{Cauchyfolge}, falls es zu jedem $\epsilon>0$ ein $N\in\N$ gibt, so dass $|a_n-a_m|<\epsilon\quad\forall n,m\geq N$\\
	(kurz: $\forall\epsilon>0\quad\exists N\in\N\quad\forall n,m\geq N\colon|a_n-a_m|<\epsilon$) mit $|a_n-a_m|$... Abstand zweier Folgenglieder
	\subsection{Satz (Cauchykriterium)}
	Eine Folge konvergiert genau dann, wenn sie eine Cauchyfolge ist.
	\begin{align*}
		(a_n)\textrm{ konvergiert }\Leftrightarrow(a_n)\textrm{ ist eine Cauchyfolge }
	\end{align*}
	Beweisskizze (ausführlicher Beweis: s. Moodle):
	\begin{itemize}
		\item "$\Rightarrow$": Einschiebetrick, Dreiecksungleichung verwenden
		\item "$\Leftarrow$": Idee: $(a_n)$ ist Cauchyfolge (zu zeigen: konvergent)\\
		zeige: $(a_n)$ ist beschränkt\\
		$\Rightarrow$ 2.28 $\exists$ konvergente Teilfolge\\
		zeige: Limes der Teilfolge ist Limes der Folge
	\end{itemize}
	\subsection{Anwendung (Banachscher Fixpunktsatz)}
	Sei $f\colon[a,b]\>[a,b]$ eine Abbildung mit\\
	$\underbrace{|f(x)-f(y)|}_{\textrm{Abstand der Bildpunkte}}<\underbrace{|x-y|}_{\textrm{Abstand von 2 Punkten}}\qquad\forall x,y\in[a,b]$\\
	("$f$ ist strikte Kontraktion")\\
	Dann hat $f$ genau einen Fixpunkt, d.h.\\
	$\underbrace{\exists!}_{\textrm{es gibt genau ein...}}r\in[a,b]$ mit $f(r)=r$\\
	\\
	Beweisidee:\\
	Starte mit beliebigem $x_0\in[a,b]$.\\
	Berechne $x_1$ als $f(x_0)\qquad x_1\coloneqq f(x_0)$\\
	$x_2$ als $f(x_1)\qquad x_2\coloneqq f(x_1)$\\
	also $x_{n+1}\coloneqq f(x_n)$\\
	\\
	Zeige: Diese Folge konvergiert (Cauchyfolge), und zwar gegen $r=f(r)$; $r$ ist eindeutig (Annahme: es existieren 2 verschiedene $r$)
	%%%%%%%%%%%%%%%%%%%%%%%%%%%%%%%%%%%%%%%%%%%%%%%%%%%%%%%
	\newpage
	%%%%%%%%%%%%%%%%%%%%%%%%%%%%%%%%%%%%%%%%%%%%%%%%%%%%%%%
	\section{Reihen}
	\subsection{Definition}
	Sei $(a_n)_{n\in\N}$ eine Folge.\\
	Summiere die ersten $n$ Folgeglieder.
	\begin{align*}
		S\coloneqq\sum_{k=1}^{n}a_k\qquad\forall
		 n\in\N\qquad (=a_1+a_2+a_3+...+a_n)
	\end{align*}
	(\underline{$n$-te Partialsumme})
	\begin{align*}
		\underbrace{\underbrace{\underbrace{\underbrace{a_1}_{S_1}+a_2}_{S_2}+a_3}_{S_3}+...+a_n}_{S_n}
	\end{align*}
	Die Folge $(S_n)_{n\in\N}=(S_1,S_2,S_3,...)$ heißt \underline{unendliche Reihe}, schreibe $\sum_{k=1}^{\infty}a_k$\\
	Falls $(S_n)_{n\in\N}$ gegen $s\in\R$ konvergiert, heißt die Reihe \underline{konvergent gegen $s$} und ihr Grenzwert wird dann ebenfalls mit $\sum_{k=1}^{\infty}a_k$ bezeichnet.\\
	(Entsprechend kann man für eine Folge $(a_n)_{n\geq n_0}$ die Reihe $\sum_{k=n_0}^{\infty}a_k$ definieren)
	\subsection{Beispiel}
	\begin{itemize}
		\item[a)] $\sum_{k=1}^{\infty}k=1+2+3+...\qquad$ divergente Folge
		\item[b)] $\sum_{k=1}^{\infty}(-1)^k=(-1)+1+(-1)+...\qquad$ divergente Folge\\
		$S_n=\sum_{k=1}^{n}(-1)^k=\begin{cases}
		0\textrm{, falls }n\textrm{ gerade}\\
		-1\textrm{, falls }n\textrm{ ungerade}
		\end{cases}$
		\item[c)] \underline{Die harmonische Reihe}
		\begin{align*}
			\sum_{k=1}^{\infty}\frac{1}{k}&=1+\frac{1}{2}+\frac{1}{3}+\frac{1}{4}+\frac{1}{5}+...\qquad\textrm{divergiert}\\
			S_n&=1+\frac{1}{2}+\underbrace{\frac{1}{3}+\frac{1}{4}}_{>2\*\frac{1}{4}=\frac{1}{2}}+\underbrace{\frac{1}{5}+...+\frac{1}{8}}_{>4\*\frac{1}{8}=\frac{1}{2}}+\underbrace{\frac{1}{9}+...+\frac{1}{16}}_{>8\*\frac{1}{16}=\frac{1}{2}}+\underbrace{...+\frac{1}{n}}_{\textrm{usw.}}\\
			&>1+\frac{1}{2}+\quad\frac{1}{2}\quad+\qquad\frac{1}{2}\qquad+\qquad\frac{1}{2}\qquad+\quad...
		\end{align*}
		$\Rightarrow$ divergent (per Induktion: $S_{2^m}\geq1+\frac{m}{2}$)
		\item[d)] $\sum_{k=0}^{\infty}\frac{1}{2^k}=1+\frac{1}{2}+\frac{1}{4}+\frac{1}{8}+\frac{1}{16}+...\qquad$ ist konvergent gegen den Grenzwert $\sum_{k=0}^{\infty}\frac{1}{2^k}=2$
		\item[e)] wichtiges Beispiel: \underline{Geometrische Reihe}\\
		Für $q\in\R$ mit $|q|<1$ gilt
		\begin{align*}
			\sum_{k=0}^{\infty}q^k&=\frac{1}{1-q}\qquad\textrm{, denn:}\\
			S_n=\sum_{k=0}^{n}q^k&=\frac{1-q^{n+1}}{1-q}\qquad\textrm{(Übung: geom. Summe, Induktion)}
		\end{align*}
		Aus 2.11:
		\begin{align*}
			\lim\limits_{\infn}q^n=0\qquad\textrm{, falls }|q|<1
		\end{align*}
		Geometrische Folge. Also gilt:
		\begin{align*}
			&S_n\>\frac{1-0}{1-q}=\frac{1}{1-q}\qquad\textrm{ für }\infn\\
			&\sum_{k=0}^{\infty}q^k\qquad\textrm{ divergiert für }|q|\geq 1
		\end{align*}
		Nochmal Beispiel d)\\
		$\sum_{k=0}^{\infty}\frac{1}{2^k}=\sum_{k=0}^{\infty}(\frac{1}{2})^k$, also geometrische Reihe mit $q=\frac{1}{2}\qquad1>|q|$, konvergiert gegen $\frac{1}{1-q}=\frac{1}{1-\frac{1}{2}}=\frac{1}{\frac{1}{2}}=2$\\
		\\
		Weitere Beispiele:
		\begin{itemize}
			\item $\sum_{k=0}^{\infty}\frac{(-1)^k}{2^k}=\sum_{k=0}^{\infty}(-\frac{1}{2})^k=\frac{2}{3}$
			\item $\sum_{k=3}^{\infty}q^k=\sum_{k=0}^{\infty}q^{k+3}=q^3\*\sum_{k=0}^{\infty}q^k=\frac{q^3}{1-q}\qquad$ (falls $|q|<1$)
		\end{itemize}
	\end{itemize}
		\subsection{Rechenregeln für Reihen}
		folgen aus den Rechenregeln für Folgen. Sei
		\begin{itemize}
			\item $\sum_{k=1}^{\infty}a_k$ konvergiert gegen $a$,
			\item $\sum_{k=1}^{\infty}b_k$ konvergiert gegen $b$.
		\end{itemize}
		Dann gilt mit $c\in\R$:
		\begin{itemize}
			\item[a)] $\sum_{k=1}^{\infty}(a_k+b_k)$ konvergiert gegen $a+b$
			\item[b)] $\sum_{k=1}^{\infty}(c\*a_k)$ konvergiert gegen $c\*a$
		\end{itemize}
	\subsection{Konvergenz-/Divergenzkriterien für Reihen}
	\begin{itemize}
		\item[\fbox{1}] Ist $S_n$ mit $S_n=\sum_{k=1}^{\infty}a_k$ beschränkt und $a_k\geq 0\quad\forall k\in\N$, so ist $\sum_{k=1}^{\infty}a_k$ konvergent (folgt aus Satz 2.19/monotone Konvergenz).
		\item[\fbox{2}] \underline{Cauchy-Kriterium}\\
		$\sum_{k=1}^{\infty}a_k$ konvergiert $\Leftrightarrow$ $\forall\epsilon>0\quad\exists N\in\N$, so dass $\forall m>n\geq N$ gilt: $|a_{n+1}+...+a_m|=|\sum_{k=n+1}^{m}a_k|<\epsilon$\\
		$|S_m-S_n|$\\
		(folgt aus 2.31/Cauchykriterium für Folgen)\\
		Daraus ergibt sich:\\
		Ist $\sum_{k=1}^{\infty}a_k$ konvergent, so ist $(a_n)_n$ Nullfolge (wähle $m=n+1$, dann $|a_{n+1}|<\epsilon$, d.h. $a_n\>0$).\\
		$\Rightarrow$[3]
		\item[\fbox{3}] \underline{Divergenzkriterium}\\
		Ist $(a_n)_n$ keine Nullfolge, so ist $\sum_{k=1}^{\infty}a_k$ divergent.\\
		Bsp: $\sum_{k=1}^{\infty}\underbrace{(1+\frac{1}{k})}_{\>1\textrm{ für }k\>\infty\textrm{, keine Nullfolge!}}$ divergiert
		\item[\fbox{4}] \underline{Majorantenkriterium}\\
		Seien $(a_n),(b_n)$ Folgen mit $|a_n|\leq b_n\quad\forall n\in\N$ (für fast alle $n$, d.h. für alle bis auf endlich viele)\\
		Dann gilt:\\
		Ist $\underbrace{\sum_{k=1}^{\infty}b_k}_{\textrm{Majorante}}$ konvergent, dann auch $\sum_{k=1}^{\infty}a_k$ und $\sum_{k=1}^{\infty}|a_k|$\\
		\\
		Beweis:
		\begin{align*}
			|\sum_{k=n+1}^{m}a_k|&\leq\sum_{k=n+1}^{m}|a_k|\\
			&\leq\sum_{k=n+1}^{m}b_k\\
			&\leq|\sum_{k=n+1}^{m}b_k|<\epsilon\textrm{ , da }\sum_{k=1}^{\infty}b_k\textrm{ konvergent,}
		\end{align*}
		also ist Cauchykriterium [2] für $\sum_{k=1}^{\infty}a_k$ erfüllt, $\sum_{k=1}^{\infty}a_k$ konvergiert.\\
		Ähnlich: \underline{Minorantenkriterium} für Divergenz, s. Blatt 5.
		\item[\fbox{5}] \underline{Leibnitzkriterium für alternierende Reihen}\\
		Sei $(a_n)_n$ reelle, monoton fallende Nullfolge mit $a_n\geq0\quad\forall n$.\\
		Dann konvergiert die alternierende Reihe $\sum_{k=0}^{\infty}(-1)^k\*a_k$\\
		\\
		Beweis: Intervallschachtelungsprinzip
		\begin{align*}
			A_n&\coloneqq\sum_{k=0}^{2n-1}(-1)^k\*a_k\\
			B_n&\coloneqq\sum_{k=0}^{2n}(-1)^k\*a_k
		\end{align*}
	\begin{itemize}
		\item $A_n\nearrow$, denn
		\begin{align*}
			A_{n1}-A_n & =\overbrace{\sum_{k=0}^{2(n+1)-1}}^{2n+1}(-1)^k\*a_k-\sum_{k=0}^{2n-1}(-1)^k\*a_k &  \\
			           & =(-1)^{2n+1}a_{2n+1}+(-1)^{2n}a_{2n}                                              & =-a_{2n+1}+a_{2n}\geq 0
		\end{align*}
		(da $(a_n)\searrow$)
		\item ähnlich für $B_n\searrow$
		\item $B_n-A_n=(-1)^{2n}a_{2n}=a_{2n}\geq0\quad\longrightarrow0$ für $\infn$ (weil $(a_n)_n$ Nullfolge nach Voraussetzung)\\
		$\Rightarrow\exists\lim\limits_{\infn}A_n=\lim\limits_{\infn}B_n$, also konvergiert $\sum_{k=0}^{\infty}(-1)^ka_k$
	\end{itemize}
	Bsp:
	\begin{itemize}
		\item[a)] Leibnitz-Reihe:
		\begin{align*}
			&1-\frac{1}{3}+\frac{1}{5}-\frac{1}{7}+...-...\\
			=&\sum_{k=0}^{\infty}(-1)^k\frac{1}{2k+1}
		\end{align*}
		konvergiert gegen $\frac{\pi}{4}$
		\item[b)] Die alternierende harmonische Reihe
		\begin{align*}
			&1-\frac{1}{2}+\frac{1}{3}-\frac{1}{4}+\frac{1}{5}-...+...\\
			=&\sum_{k=0}^{\infty}(-1)^k\frac{1}{k+1}
		\end{align*}
		konvergiert gegen $\ln2$
	\end{itemize}
	\item[\fbox{6}] \underline{Absolute Konvergenz}
	\subsubsection*{Definition}
	Eine Reihe $\sum_{k=0}^{\infty}a_k$ heißt \underline{absolut konvergent}, falls die Betragsreihe $\sum_{k=0}^{\infty}|a_k|$ konvergiert.\\
	\subsubsection*{Beispiel}
	\begin{itemize}
		\item[a)] $\sum_{k=1}^{\infty}(-1)^k\frac{1}{k^2}$ konvergiert absolut, da $\sum_{k=1}^{\infty}|(-1)^k\frac{1}{k^2}|=\underbrace{\sum_{k=1}^{\infty}\frac{1}{k^2}}_{\textrm{s. \fbox{6a}}}$ konvergiert
		\item[b)] $\sum_{k=1}^{\infty}(-1)^k\frac{1}{k}$ konvergiert nicht absolut (aber konvergiert, s. Leibnitzkriterium), da $\sum_{k=1}^{\infty}|(-1)^k\frac{1}{k}|=\sum_{k=1}^{\infty}\frac{1}{k}$ (harmonische Reihe, konvergiert nicht)
	\end{itemize}
	Es gilt: $\overset{\textrm{(Majorantenkriterium)}}{\textrm{Reihe konvergiert absolut}}$ $\Rightarrow$ Reihe konvergiert\\
	(aber nicht umgekehrt, s. Beispiel b))
	\item[\fbox{6a}] \underline{Wurzelkriterium}\\
	Für $a_k\in\R$ gilt:
	\begin{itemize}
		\item falls $\limsup{\infn}\sqrt[n]{|a_n|}<1\Rightarrow\sum_{k=0}^{\infty}|a_k|$ konvergiert (d.h. $\sum_{k=0}^{\infty}a_k$ konvergiert absolut)
		\item falls $\limsup{\infn}\sqrt[n]{|a_n|}>1\Rightarrow\sum_{k=0}^{\infty}a_k$ divergiert
		\item für $\limsup{\infn}\sqrt[n]{|a_k|}=1$ ist keine allgemeine Aussage möglich
	\end{itemize}
	Beweis:\\
	Sei $s\coloneqq\limsup{\infn}\sqrt[n]{|a_n|}$
	\begin{itemize}
		\item falls $s<1$: Wähle kleines $\epsilon>0$, so dass $s+\epsilon<1$\\
		$\Rightarrow\sqrt[n]{|a_n|}\leq s+\epsilon$ für fast alle $n$\\
		$\Rightarrow|a_n|\leq(s+\epsilon)^n$\\
		Die Reihe $\sum_{k=0}^{\infty}{\underbrace{(s+\epsilon)}_{<1}}^n$ ist geometrische Reihe und konvergiert, und ist Majorante für die Reihe $\sum_{k=0}^{\infty}|a_k|$
		\item falls $s>1$, dann ist $\sqrt[n]{|a_n|}>1$ für unendlich viele $n$, also $a_n\nrightarrow0$, $\sum_{k=0}^{\infty}a_n$ divergent nach \fbox{3}
		\item z.B. $\sum_{k=1}^{\infty}\frac{1}{k^\alpha}$ (allgemeine harmonische Reihe) mit $\alpha\geq1$ liefert $\limsup{\infn}\sqrt[n]{|a_n|}=1$, aber es gilt (Mitteilung):\\
		für $\alpha=1$ ist Reihe divergent (für $0<\alpha<1$ ebenso, Blatt 5 Aufgabe 2);\\
		für $\alpha>1$ ist Reihe konvergent\\
		Das Wurzelkriterium kann diese Fälle nicht unterscheiden.
	\end{itemize}
	\item[\fbox{6b}] \underline{Quotientenkriterium}\\
	Sei $a_n\neq0$ für fast alle $k$ (d.h. für alle bis auf endlich viele)
	\begin{itemize}
		\item falls $\limsup{\infn}|\frac{a_{n+1}}{a_n}|<1\Rightarrow\sum_{k=0}^{\infty}|a_k|$ konvergiert
		\item falls $\liminf{\infn}|\frac{a_{n+1}}{a_n}|>1\Rightarrow\sum_{k=0}^{\infty}a_k$ divergiert
		\item falls $\limsup{\infn}|\frac{a_{n+1}}{a_n}|\geq1$ und $\liminf{\infn}|\frac{a_{n+1}}{a_n}|\leq1$, so ist keine allgemeine Aussage möglich (wie bei \fbox{6a}, dritter Punkt)
	\end{itemize}
	Beweis: ähnlich wie \fbox{6a}
 \end{itemize}%\fbox{numbers}
 \subsection{Bemerkung}
 Umordnung einer Reihe, Konvergenzverhalten\\
 $\>$ s. Folien 11.05.2016
 \newpage
 \section{Potenzreihen}
 \subsection{Definition}
 Sei $(a_k)_k$ eine reelle Folge, $x\in\R$. Dann heißt die Reihe
 \begin{align*}
 	P(x)=\sum_{k=0}^{\infty}a_k\*x^k
 \end{align*}
 \underline{Potenzreihe} mit Koeffizientenfolge $(a_k)_k$ (oft auch $P(x)=\sum_{k=0}^{\infty}a_k\*(x-b)^k$, $b\in\R$ heißt \underline{Entwicklungspunkt}).\\
 Falls $a_k\neq0$ für nur endlich viele (d.h. $a_k=0$ für fast alle $k$), dann heißt $P(x)$ \underline{Polynom}.\\
 (Unterschied zu bisherigen Reihen: abhängig von $x$. Für welche Werte von $x$ konvergiert $P(x)$? Klar: für $x=0$, dann $P(0)=a_0\*0^0=a_0$, auch für andere $x$? Das hängt von der Folge $a_k$ ab)
 \subsection{Beispiel}
 \begin{itemize}
 	\item[a)] $\sum_{k=0}^{\infty}x^k=\sum_{k=0}^{\infty}1\*x^k\qquad(a_k=1\quad\forall k,\quad b=0)$ konvergiert für alle $x$ mit $|x|<1$ (geometrische Reihe!), also für $x\in\underbrace{(-1,1)}_{\textrm{\underline{Konvergenzintervall}}}$, sonst divergiert sie (z.B. für $x=2,\quad x=3,\quad...$)
 	\item[b)] $\sum_{k=0}^{\infty}2^k\*x^k$ (d.h. $a_k=2^k\quad\forall k$) $=\sum_{k=0}^{\infty}(2x)^k$ wie in a), konvergiert für alle $x$ mit $|2x|<1$, also $|x|<\frac{1}{2}$; also für $x\in(-\frac{1}{2},\frac{1}{2})$
 \end{itemize}
 \subsection{definition (Formel von Cauchy-Hadamard)}
 Sei $P(x)=\sum_{k=0}^{\infty}a_kx^k$ eine Potenzreihe.
 \begin{align*}
 	\rho\coloneqq\frac{1}{\limsup{\infn}\sqrt[n]{|a_n|}}
 \end{align*}
 heißt der \underline{Konvergenzradius} von $P(x)$ (dabei sei $\frac{1}{0}\coloneqq+\infty$ und $\frac{1}{\infty}\coloneqq0$ gesetzt, es ist also $\rho\in\R\cup\{\infty\}$). Zur Bedeutung von $\rho=\infty$ siehe 4.4.\\
 Oft einfacher: Formel von Euler:
 \begin{align*}
 	\rho=\lim\limits_{\infty}|\frac{a_n}{a_{n+1}}|\qquad\textrm{ , }
 \end{align*}
	falls $(|\frac{a_n}{a_{n+1}}|)$ konvergente Folge ist oder "bestimmt gegen $\infty$ divergiert", d.h. falls $\forall K>0\quad\exists N$ mit $|\frac{a_n}{a_{n+1}}|\geq K\quad\forall n\geq N$, dann setze $\rho=+\infty$.\\
	(Achtung: z.B. für $a_n=\begin{cases}
	1\textrm{ , n gerade}\\
	\frac{1}{n}\textrm{ , n ungerade}
	\end{cases}$ ist diese Formel nicht anwendbar!)
	\subsection{Satz (Konvergenz von Potenzreihen)}
	Sei $P(x)=\sum_{k=0}^{\infty}a_kx^k$ eine Potenzreihe mit Konvergenzradius $\rho$. Dann gilt:
	\begin{itemize}
		\item[a)] Für alle $x\in\R$ mit $|x|<\rho$ konvergiert $P(x)$ absolut (d.h. Reihe konvergiert für alle $x\in\R$, die im Konvergenzintervall $(-\rho,\rho)$ liegen.\\
		Ist $\rho=\infty$, so heißt das für \underline{alle} $x\in\R$!).
		\item[b)] Für alle $x$ mit $|x|>\rho$ divergiert $P(x)$.
		\item[c)] Für $|x|=\rho$ ist keine allgemeine Aussage möglich (Konvergenzintervall kann also $(-\rho,\rho)$, $[-\rho,\rho]$, $[-\rho,\rho)$, $(-\rho,\rho]$ sein).
	\end{itemize}
	\underline{Beweis:}
	\begin{itemize}
		\item[a)] Nach dem Wurzelkriterium 3.4 \fbox{6a} ist $\sum_{k=0}^{\infty}|a_kx^k|$ konvergent, falls $\limsup{\infn}\sqrt[n]{|a_nx^n|}<1$ gilt. Das ist äquivalent zu
		\begin{align*}
			&|x|\*\limsup{\infn}\sqrt[n]{|a_n|}&<1\\
			\Leftrightarrow&|x|&<\frac{1}{\limsup{\infn}\sqrt[n]{|a_n|}}=\rho
		\end{align*}
		\item[b)] analog
		\item[c)] Übung (Beispiel suchen)\hfill$\square$
	\end{itemize}
	Mit dem Quotientenkriterium 3.4 \fbox{6b} lässt sich der Satz auch für die Formel von Euler für $\rho$ beweisen.	
	\subsection{Bemerkung}
	Ist $\rho$ Konvergenzradius von $\sum_{k=0}^{\infty}a_kx^k$, so konvergiert die Reihe absolut für $|x|<\rho$ (für $x\in(-\rho,\rho)$), divergiert für $|x|>\rho$.\\
	Die Reihe $\sum_{k=0}^{\infty}a_k(x-b)^k$ konvergiert dann absolut für $|x-b|<\rho$ (für $x\in(b-\rho,b+\rho)$); divergiert für $|x-b|>\rho$. Für $x=b-\rho$, $x=b+\rho$ ist keine allgemeine Aussage möglich.\\
	(Also: Falls $b$ dabei: erst alles ohne $b$ rechnen (4.3,4.4), dann Bemerkung 4.5 verwenden)
	\subsection{Beispiel}
	\begin{itemize}
		\item[a)] Bsp. 4.2 mit der Formel für $\rho$ nachrechnen (Präsenzübungsblatt 6)
		\item[b)] wichtiges Beispiel: die Exponentialreihe
		\begin{align*}
			\sum_{k=0}^{\infty}\frac{x^k}{x!}&\qquad(a_k=\frac{1}{k!},\quad b=0)\\
			&=1+\frac{x}{1!}+\frac{x^2}{2!}+\frac{x^3}{3!}+...
		\end{align*}
		hat Konvergenzradius $\rho=\infty$ nach Euler, d.h. Reihe konvergiert für \underline{alle} $x\in\R$, deshalb kann man für $x\in\R$ die folgende Funktion definieren (Exponentialfunktion):
		\begin{align*}
			\exp\colon&\R\>\R\\
			x\mapsto \exp(x)=\sum_{k=0}^{\infty}\frac{x^k}{k!}
		\end{align*}
		Es gilt (vgl. Präsenzübungsblatt 6) (Cauchyprodukt):\\
		$\exp(x+y)=\exp(x)\*\exp(y)\qquad x,y\in\R$\\
		Für $x=1$ erhält man\\
		$\exp(1)=\frac{1}{0!}+\frac{1}{1!}+\frac{1}{2!}+\frac{1}{3!}+...$\\
		Man kann zeigen: Dies ist $\e$ (Eulersche Zahl) $\lim\limits_{\infn}(1+\frac{1}{n})^n$.\\
		Allgemein gilt: Die Folge $((1+\frac{x}{n})^n)_n$ konvergiert für $\infn$ gegen $\exp(x)=\sum_{k=0}^{\infn}\frac{x^k}{k!}$\\
		Daher schreibt man auch\\
		$\e^x=\exp(x)$ für $x\in\R$
	\end{itemize}
	\newpage
	\section{Funktionsgrenzwerte und Stetigkeit}
	\subsection{Definition}
	Sei $D\subseteq\R$, $f\colon D\>\R$ eine Funktion, $x_0,a\in\R$.
	\begin{itemize}
		\item[a)] $f$ heißt \underline{konvergent gegen $a$ für $x$ gegen $x_0$}, wenn gilt:\\
		Für \underline{alle} Folgen $\underbrace{(x_n)_n\textrm{ aus }D\setminus\{x_0\}}_{\textrm{d.h. Glieder der Folge sind alle aus } D\setminus\{x_0\}}$, die gegen $x_0$ konvergieren, konvergieren die Funktionswerte $f(x_n)\> a$, also $f(x_n)\> a$ für $x_n\> x_0$ (Schreibweise: $\lim\limits_{x\> x_0}f(x)=a$).
		\item[b)] Analog lässt sich der Grenzwert für $x\>\infty$ oder für $x\>-\infty$ definieren:\\
		$f$ konvergiert gegen $a$ für
		$\begin{array}{l}
		x\>\infty\\
		x\>-\infty
		\end{array}$ falls für alle Folgen $(x_n)_n$ mit $\underbrace{\begin{array}{l}
		x_n\>\infty\\
		x_n\>-\infty
		\end{array}}_{\textrm{d.h. }\forall k\in\N\exists N\in\N\colon x_n>k\forall n\geq N}$ gilt: $f(x_n)\> a$ ($\lim\limits_{x\>\infty}=a$ bzw. $\lim\limits_{x\>-\infty}f(x)=a$)
	\end{itemize}
	\subsection{Beispiel}
	\begin{itemize}
		\item[zu a)] \begin{align*}
			f\colon D=&\R\>\R\qquad\quad x_0\in\R\\
			&x\> x^2
		\end{align*}
		Was ist $\lim\limits_{x\> x_0}f(x)$?\\
		Sei $(x_n)_n$ Folge mit $\lim\limits_{\infn}x_n=x_0$. Nach den Rechenregeln für Folgen (Satz 2.13) gilt:\\
		$f(x_n)=x^2_n\> x^2_0$ (Voraussetzung $x_n\> x_0$, aus Rechenregeln $x^2_n\> x^2_0$), also ist $\lim\limits_{x\> x_0}f(x)=x^2_0\eqqcolon a$\\
		(Bemerkung: allgemein gilt: $\lim\limits_{x\> x_0}f(x)=f(x_0)$ für alle Polynome)
		\item[zu b)] \begin{align*}
			f\colon (0,\infty)=&\R^+\>\R\\&x\mapsto\frac{1}{x}
		\end{align*}
		Was ist $\lim\limits_{x\>\infty}f(x)$?\\
		Für alle $(x_n)_n$ mit $x_n\>\infty$ für $\infn$ gilt:\\
		$f(x_n)=\frac{1}{x_n}\>0$, also $\lim\limits_{x\>\infty}f(x)=0$
	\end{itemize}
	\subsection{Bemerkung/Definition}
	Definition 5.1 ist nur interessant für die Punkte $x_0\in\R$, für die es Folgen $(x_n)_n$ aus $D\setminus\{x_0\}$ gibt, die gegen $x_0$ konvergieren.\\
	Solche Punkte nennt man \underline{Häufungsstellen (HS)} von $D$.\\
	$\overline{D}\coloneqq D\cup\{x|x\textrm{ ist HS von }D\}$ heißt \underline{Abschluss} von $D$.\\
	Beispiel:
	\begin{itemize}
		\item[a)] $D=(0,\infty)=\R^+$\\
		-1 keine HS von $D$\\
		1 ist HS von $D$\\
		0 ist HS von $D$
		\item[b)] $D=\R^+\cup\{-2\}$\\
		-2 keine HS von $D$, aber $-2\in D$, $-2\in\overline{D}$
		\item[c)] $D=(a,b)\subseteq\R$, dann sind $a$ und $b$ HS;\\
		$\overline{D}=[a,b]$
		\item[d)] Es gilt: $\overline{\mathds{Q}}=\R$ 
	\end{itemize}
	\subsection{Bemerkung/Definition}
	In Definition 5.1 muss $f(x_n)\> a$ für \underline{alle} Folgen $(x_n)_n$ aus $D\setminus\{x_0\}$, die gegen $x_0$ konvergieren, gelten; also insbesondere für Folgen, die \underline{von links} ($x_n\> x_0$ mit $x_n<x_0\forall n$) und für Folgen, die \underline{von rechts} ($x_n\> x_0$ mit $x_n>x_0\forall n$) gegen $x_0$ konvergieren (falls möglich).\\
	\\
	Man spricht vom \underline{links- bzw. rechtsseitigen Grenzwert}. Schreibweise:\\
	\\
	\begin{minipage}[c]{0.5\textwidth}
		links:\\
		\\
		$\lim\limits_{x\nearrow x_0}f(x)=a$\\
		\\
		oder\\
		$\lim\limits_{x\> x_0-}$
	\end{minipage}
	\begin{minipage}[c]{0.5\textwidth}
		rechts:\\
		\\
		$\lim\limits_{x\searrow x_0}f(x)=a$\\
		\\
		oder\\
		$\lim\limits_{x\> x_0+}$
	\end{minipage}
	\subsection{Beispiel}
	\begin{itemize}
		\item[a)] \begin{itemize}
			\item $D=\R$, $x_0=0$\\
			Folge aus $D\setminus\{x_0\}$, die von rechts [links] gegen $x_0$ konvergiert ist z.B. $(x_n)_n=(\frac{1}{n})_n\qquad[(x_n)_n=(\frac{-1}{n})_n]$
		\item $D=[0,\infty]$, $x_0=0$\\
		Nur Folgen, die von rechts gegen $x_0$ konvergieren, sind möglich (sonst nicht in  $D$)
		\end{itemize}
		\item[b)] Heavisidefunktion (Schwellenwertfunktion)\\
		\begin{minipage}[c]{0.5\textwidth}
			\begin{align*}
				f\colon\R&\>\R\\
				f(x)&=\begin{cases}
				0\textrm{, }x<0\\
				1\textrm{, }x\geq0
				\end{cases}
			\end{align*}
		\end{minipage}
		\begin{minipage}[c]{0.5\textwidth}
			\begin{tikzpicture}[scale=0.5]
			\draw[->]
			(-4.2,0) -- (4.2,0) node[right] {$x$};
			\draw[->]
			(0,-0.2) -- (0,4.2) node[above] {$f(x)$};
			\draw[color=blue, domain=-3:0, ultra thick]
			plot(\x,0) node[right]{};
			\draw[color=blue, domain=0:3, ultra thick]
			plot(\x,3) node[left]{};
			\draw[fill, color=blue] (0,3) circle [radius=5pt] node[right] {};
			\draw[color=blue] (0,0) circle [radius=5pt] node[right] {};
			\end{tikzpicture}
		\end{minipage}\\
		$\lim\limits_{x\>0}f(x)=$?
		\begin{itemize}
			\item für $(x_n)_n=(\frac{1}{n})_n$ gilt $f(x_n)=f(\frac{1}{n})=1\>1$
			\item für $(x_n)_n=(-\frac{1}{n})_n$ gilt $f(x_n)=f(-\frac{1}{n})=0\>0$
		\end{itemize}
		Also gibt es kein $a$, so dass \underline{alle} Folgen  mit $x_n\> x_0$ die Bedingung $f(x_n)\> a$ erfüllen.\\
		$\lim\limits_{x\> 0}f(x)$ existiert hier nicht!
	\end{itemize}
	\subsection{Bemerkung/Definition}
	Sei $f\colon D\>\R,\quad x_0\in\overline{D}$.\\
	Falls für alle Folgen $(x_n)_n\in D\setminus\{x_0\}$ mit $x_n\> x_0$ gilt:
	\begin{align*}
		f(x_n)\> \pm\infty\quad\textrm{ , }
	\end{align*}
	so sagt man, \underline{$f$ divergiert bestimmt gegen $\pm\infty$} für $x\> x_0$ (analog für $x\>\pm\infty$).\\
	\\
	\underline{Beispiel:}
	\begin{align*}
		f\colon D&\>\R\\
		x&\mapsto\frac{1}{x}
	\end{align*}
	\begin{itemize}
		\item falls $D=(0,\infty)$:\\
		$f(x)$ divergiert bestimmt gegen $\infty$ für $x\> 0$ 
		\item falls $D=(-\infty,0)$:\\
		$f(x)$ divergiert bestimmt gegen $-\infty$ für $x\> 0$
		\item falls $D=\R\setminus\{0\}$:\\
		dann existiert $\lim\limits_{x\>0}f(x)$ nicht und $f(x)$ divergiert auch nicht bestimmt.
	\end{itemize}
	\subsection{Definition (Stetigkeit)}
	Sei $f(x)$ eine reelle Funktion $f\colon D\>\R$.\\
	Die Funktion heißt \underline{stetig an der Stelle $x_0\in D$}, falls $\lim\limits_{x\> x_0}f(x)=f(x_0)$ gilt.\\
	Die Funktion heißt (überall) \underline{stetig in D}, falls $f(x)$ in jedem Punkt $x_0\in D$ stetig ist.
	\subsection{Bemerkung}
	Sei $f\colon D\>\R$.
	\begin{itemize}
		\item[a)] Äquivalente Definition der Stetigkeit: $\epsilon$-$\delta$-Kriterium, siehe z.B. WHK 6.17:
		\begin{align*}
			\forall\epsilon>0\quad\exists\delta>0\quad\forall x\in D\colon|x-x_0|\leq\delta\Rightarrow|f(x)-f(x_0)|\leq\epsilon
		\end{align*}
		\item[b)] Gibt es eine Konstante $K$ mit
		\begin{align*}
			|f(x)-f(x_0)|\leq K\*|x-x_0|\quad\forall x\in D\qquad\textrm{, }
		\end{align*}
		dann ist die Funktion stetig.
	\end{itemize}
	\subsection{Beispiel}
	\begin{itemize}
		\item[a)] \begin{align*}
			f\colon\R&\>\R\\
			x&\mapsto x^2
		\end{align*}
		$f(x)$ ist stetig auf $\R$ ($\lim\limits_{x\> x_0}f(x)$ existiert [vgl. Bsp. 5.2 a)] und ist gleich $f(x_0)\quad\forall x_0\in\R$)
		\item[b)]
		\begin{itemize}
			\item[(1)] Bild:\\
			\begin{minipage}[c]{0.5\textwidth}
				$f(x)$ ist stetig in $x_0$
			\end{minipage}
			\begin{minipage}[c]{0.5\textwidth}
				\begin{tikzpicture}[scale=0.5]
				\draw[->]
				(-0.2,0) -- (4.2,0) node[right] {$x$};
				\draw[->]
				(0,-0.2) -- (0,4.2) node[above] {$f(x)$};
				\draw[color=blue]
				(0,2.75) node[left]{$\lim\limits_{x\> x_0}f(x)=f(x_0)$}
				(2.3,0) node[below]{$x_0$};
				\draw[style=dotted, color=blue]
				(0,2.75)--(2.3,2.75);
				\draw[style=dotted, color=blue]
				(2.3,2.75)--(2.3,0);
				\draw[color=blue, domain=0.5:3.7, ultra thick]
				plot(\x,0.875*\x*\x*\x-6.625*\x*\x+14.75*\x-6.5) node[right]{$f$};
				\end{tikzpicture}
			\end{minipage}
			\item[(2)] Bild:\\
			\begin{minipage}[c]{0.5\textwidth}
				$f(x)$ ist nicht stetig in $x_0$:\\
				$\lim\limits_{x\> x_0}$ existiert nicht!
			\end{minipage}
			\begin{minipage}[c]{0.5\textwidth}
				\begin{tikzpicture}[scale=0.5]
				\draw[->]
				(-0.2,0) -- (4.2,0) node[right] {$x$};
				\draw[->]
				(0,-0.2) -- (0,4.2) node[above] {$f(x)$};
				\draw[color=blue]
				(0,3.6) node[left]{$f(x_0)$}
				(2,0) node[below]{$x_0$};
				\draw[style=dotted, color=blue]
				(2,0)--(2,3.6);
				\draw[color=blue, style=dotted]
				(0,3.6)--(2,3.6);
				\draw[color=blue, domain=0.5:2, ultra thick]
				plot(\x,\x);
				\draw[color=blue, domain=2:4, ultra thick]
				plot(\x,0.875*\x*\x*\x-6.625*\x*\x+14.75*\x-6.5) node[right]{$f$};
				\draw[color=blue] (2,2) circle [radius=5pt] node[right] {};
				\draw[fill, color=blue] (2,3.6) circle [radius=5pt] node[right] {};
				\end{tikzpicture}
			\end{minipage}
			\item[(3)] Bild:\\
			\begin{minipage}[c]{0.5\textwidth}
				$f(x)$ ist nicht stetig in $x_0$:\\
				$\lim\limits_{x\> x_0}$ existiert, ist aber ungleich $f(x_0)$.
			\end{minipage}
			\begin{minipage}[c]{0.5\textwidth}
				\begin{tikzpicture}[scale=0.5]
				\draw[->]
				(-0.2,0) -- (4.2,0) node[right] {$x$};
				\draw[->]
				(0,-0.2) -- (0,4.2) node[above] {$f(x)$};
				\draw[color=blue]
				(0,3.6) node[left]{$f(x_0)$}
				(2,0) node[below]{$x_0$}
				(0,2) node[left]{$\lim\limits_{x\> x_0}f(x)$};
				\draw[style=dotted, color=blue]
				(2,0)--(2,3.6);
				\draw[color=blue, style=dotted]
				(0,3.6)--(2,3.6);
				\draw[color=blue, style=dotted]
				(0,2)--(2,2);
				\draw[color=blue, domain=0.5:1.9, ultra thick]
				plot(\x,-\x*\x+4*\x-2) node[right]{};
				\draw[color=blue, domain=2.1:3.6, ultra thick]
				plot(\x,-\x*\x+4*\x-2) node[right]{$f$};
				\draw[color=blue] (2,2) circle [radius=5pt] node[right] {};
				\draw[fill, color=blue] (2,3.6) circle [radius=5pt] node[right] {};
				\end{tikzpicture}
			\end{minipage}
		\end{itemize}
		\item[c)]
		\begin{align*}
			f\colon\R&\>\R\\
			x&\mapsto|x|
		\end{align*}
		ist stetig auf $\R$
	\end{itemize}
	\subsection{Bemerkung}
	Eine Funktion $f(x)$ ist stetig, falls der Graph von $f$ keine 'Sprungstelle' hat bzw. "man $f$ ohne abzusetzen zeichnen kann".\\
	$\Rightarrow$ in Ordnung für Intuition, aber unpräzise\\
	\\
	\underline{Beispiel dazu:}
	\begin{itemize}
		\item[a)]
		\begin{align*}
			f\colon D=\R\setminus\{0\}&\>\R\\x&\mapsto\frac{1}{x}
		\end{align*}
		Die Funktion ist stetig auf $D$, weil die $0$ ausgenommen wurde.
		\item[b)] Dirichlet-Funktion
		\begin{align*}
			f\colon\R&\>\R\\
			x&\mapsto\begin{cases}1\quad&\textrm{falls }x\in\mathds{Q}\\
			0\quad&\textrm{falls }x\in\R\setminus\mathds{Q}
			\end{cases}
		\end{align*}
		Die Funktion ist unstetig in \underline{jedem} Punkt $x_0\in\R$.
		\item[c)] Thomaesche Funktion\\
		Die Funktion ist stetig in jedem $x_0\in\R\setminus\mathds{Q}$ in $[0,1]$\\
		und unstetig in jedem $x_0\in\mathds{Q}$ in $[0,1]$.
	\end{itemize}
	\subsection{Satz (Rechenregeln für stetige Funktionen)}
	\begin{itemize}
		\item[a)] Seien $f, g\colon D\>\R$ stetig in $x_0$, $c\in\R$. Dann sind auch
		\begin{itemize}
			\item $c\* f$
			\item $f+g$
			\item $f-g$
			\item $f\* g$
			\item $\frac{f}{g}$ (für $g(x)\neq 0\quad\forall x\in D$)
		\end{itemize}
		stetig in $x_0$.
		\item[b)] Die Komposition zweier stetiger Funktionen ist stetig.
		\begin{align*}
			D,D'&\subseteq\R\\
			f\colon D&\>\R\\
			g\colon D'&\>\R\\
			f(D)&\subseteq D'\\
			f,g&\textrm{ stetig}\\
			&\Rightarrow\\
			(g\circ f)\colon D\>\R&\textrm{ ist stetig}
		\end{align*}
		\underline{Beweis} folgt direkt aus Def. 5.1, 5.7 und den Rechenregeln für Folgen 2.13.
	\end{itemize}
	\subsection{Bemerkung}
	Es gilt: $D\subseteq\R$ Intervall, $f\colon D\> f(D)$ bijektiv, stetig.\\
	Dann ist auch die Umkehrfunktion $f^{-1}\colon f(D)\> D$ stetig.
	\subsection{Bemerkung}
	Man kann zeigen (u.a. mit 5.11), dass
	\begin{itemize}
		\item[a)] Potenzreihen mit Konvergenzradius $\rho$ sind stetig für alle $x$ mit $|x|<\rho$.
		\item[b)] Polynome, Exponentialfunktionen, Logarithmen, Wurzelfunktionen sind stetig auf ihrem gesamten Definitionsbereich.
		\item[c)] $\sin(x), \cos(x), \tan(x), \cot(x)$ ebenso (vgl. PÜ 7*).
	\end{itemize}
	\subsection{Bemerkung/Definition (Rationale Funktionen)}
	\begin{align*}
		f\colon D&\>\R\\x&\mapsto\frac{p(x)}{q(x)}
	\end{align*}
	sei rationale Funktion mit $D=\R\setminus\{x\in\R|q(x)=0\}$.\\
	Dann ist $f$ stetig auf ganz D.\\
	Lässt sich $f$ auf ganz $\R$ definieren ("fortsetzen"), so dass man eine auf ganz $\R$ stetige Funktion erhält?\\
	\underline{Beispiel:}\\
	\begin{itemize}
		\item[a)] \begin{align*}
			f\colon D=\R\setminus\{1\}&\> \R\\
			f(x)&=\frac{x^2-1}{x-1}
		\end{align*}
		ist stetig auf D.\\
		Setze $f$ auf $\R$ fort.\\
		\begin{minipage}[c]{0.5\textwidth}
			\begin{align*}
			\tilde{f}\colon \R&\>\R\\
			\tilde{f}(x)&=\begin{cases}\frac{x^2-1}{x-1}&\textrm{falls }x\neq1\\
			b&\textrm{falls }x=1
			\end{cases}
			\end{align*}
		\end{minipage}
		\begin{minipage}[c]{0.5\textwidth}
			\begin{tikzpicture}[scale=0.5]
			\draw[->]
			(-0.2,0) -- (4.2,0) node[right] {$x$};
			\draw[->]
			(0,-0.2) -- (0,4.2) node[above] {$f(x)$};
			\draw[color=blue]
			(0,2) node[left]{$b$}
			(1,0) node[below]{$1$};
			\draw[color=blue] (1,2) circle [radius=5pt] node[right] {};
			\draw[color=blue, domain=-0.5:0.9, ultra thick]
			plot(\x,\x+1) node[right]{};
			\draw[color=blue, domain=1.1:3.5, ultra thick]
			plot(\x,\x+1) node[right]{$f(x)$};
			\end{tikzpicture}
		\end{minipage}\\
		$\tilde{f}$ ist stetig in $x_0=1$ genau dann, wenn $b=2$ gewählt wird,\\ denn $\lim\limits_{x\>1}f(x)=2$.\\
		$x_0=1$ ist eine \underline{(stetig) hebbare Definitionslücke} von $f$\\
		Allgemein:\\
		Sei $f\colon\R\setminus\{x_0\}\>\R$, es existiert $\lim\limits_{x\> x_0}f(x)\eqqcolon r\quad r\in\R$, dann ist $x_0$ stetig hebbare Definitionslücke von $f$, die Funktion
		\begin{align*}
			\tilde{f}\colon \R&\>\R\\
			\tilde{f}(x)&=\begin{cases}f(x)&\textrm{ für }x\neq x_0\\
			r&\textrm{ für }x=x_0
			\end{cases}
		\end{align*}
		ist dann die stetige Fortsetzung von $f$ auf $\R$.
		\item[b)] $f\colon\R\setminus\{1\}\>\R,\quad f(x)=\frac{(x^2-1)(x-1)}{(x-1)}$\\
		Definitionslücke $x_0=1$ hebbar durch $0=\lim\limits_{x\>1}f(x)$,\\
		$\tilde{f}\colon\R\>\R\quad\tilde{f}(x)=\begin{cases}f(x)&x\neq 1\\0&x=1
		\end{cases}$ stetig.
		\item[c)] $f\colon\R\setminus\{1\}\>\R,\quad f(x)=\frac{x^2}{(x-1)^2}=\frac{x+1}{x-1}$\\
		Definitionslücke $x_0=1$ nicht hebbar: $\lim\limits_{x\>1}f(x)$ existiert nicht
		\item[d)] Gilt für die Nullstelle $x_0$ des Nenners einer rationalen Funktion $f(x)\>\pm\infty$ für $x\> x_0\mp$, so nennt man $x_0$ \underline{Polstelle}.\\
		Z.B. $f,g\colon\R\setminus\{0\}\>\R,\quad f(x)=\frac{1}{x},\quad g(x)=\frac{1}{x^2}$\\
		$x_0=0$\\
		\begin{itemize}
			\item $f(x)\>\infty$ für $x\>0+$
			\item $f(x)\>-\infty$ für $x\>0-$
			\item $g(x)\>\infty$ für $x\>0+$ und $x\>0-$
		\end{itemize}
		\item[e)] $f\colon\R\setminus\{0\}\>\R,\quad f(x)=\sin(\frac{1}{x})$ hat bei $x_0=0$ \underline{Oszillationsstelle}:
		\begin{itemize}
			\item für $(x_n)_n=(\frac{1}{n\pi})_n$ ist $f(x_n)=\sin(n\pi)=0\>0$
			\item für $(x_n)_n=(\frac{1}{\frac{\pi}{2}+2\pi n})_n$ ist $f(x_n)=\sin(\frac{\pi}{2}+2\pi n)=1\>1$
		\end{itemize}
		d.h. $\lim\limits_{x\>0}f(x)$ existiert nicht
		\item[f)] Es gilt aber:\\
		$f(x)=x\*\sin(\frac{1}{x})\>0$ für $x\>0$ (vgl. ÜB 7)\\
		stetig hebbare Definitionslücke, hebbar durch 0
		\item[g)] \underline{Wichtig:} (ohne Beweis hier, später)\\
		$f(x)=\frac{\sin(x)}{x}\>1$ für $x\>0$\\
		stetig hebbare Definitionslücke, hebbar durch 1
	\end{itemize}
	Stetige Funktionen auf abgeschlossenen Intervallen  ($f\colon[a,b]\>\R$) besitzen wichtige Eigenschaften, u.a.:
	\begin{itemize}
		\item Zwischenwerteigenschaft
		\item Existenz von Minimum und Maximum
	\end{itemize}
	\subsection{Satz (Zwischenwertsatz von Bolzano, Nullstellensatz, ZWS, IVT [Intermediate Value Theorem])}
	$f\colon[a,b]\>\R$ stetig, $f(a)<0$ und $f(b)>0$\\
	(genauso: $f(a)>0$ und $f(b)<0$ bzw. "$f(a)-f(b)<0$")\\
	\\
	Dann existiert ein $c\in[a,b]$ mit $f(c)=0$, d.h. $f$ hat Nullstelle in $[a,b]$.\\
	Anschaulich klar:\\
	\begin{minipage}[c]{0.5\textwidth}
		$f$ stetig ('ohne absetzen'), keine Sprünge möglich
	\end{minipage}
	\begin{minipage}[c]{0.5\textwidth}
		\begin{tikzpicture}[scale=0.5]
		\draw[->]
		(-0.2,0) -- (4.2,0) node[right] {$x$};
		\draw[->]
		(0,-2.2) -- (0,4.2) node[above] {$f(x)$};
		\draw[color=blue]
		(0.5,0) node[above]{$a$}
		(3,0) node[below]{$b$}
		(2,0) node[below, color=purple]{$c$}
		(0,-1.5) node[left]{$f(a)$}
		(0,1) node[left]{$f(b)$};
		\draw[color=blue, style=dotted]
		(0,1)--(3,1)--(3,0);
		\draw[color=blue, style=dotted]
		(0,-1.5)--(0.5,-1.5)--(0.5,0);
		\draw[color=blue, domain=0.1:3.9, ultra thick]
		plot(\x,\x-2) node[right]{$f(x)$};
		\end{tikzpicture}
	\end{minipage}
	\underline{Beweis:} mittels Bisektionsverfahren\\
	\subsubsection*{Start:}
	$[a,b]\eqqcolon[a_1,b_1]$\\
	\\
	\begin{minipage}[c]{0.5\textwidth}
	\subsubsection*{Schritt 1:}
	halbiere das Intervall\\
	berechne $y_1=f(\frac{a_1+b_1}{2})$
	\begin{itemize}
		\item $y_1<0$\\
		neues Intervall: $[a_2,b_2]\coloneqq[\frac{a_1+b_1}{2},b_1]$
		\item $y_1>0$\\
		neues Intervall:$[a_2,b_2]\coloneqq[a_1,\frac{a_1+b_1}{2}]$
		\item $y_1=0$\\
		habe $c$ schon gefunden, $c=\frac{a_1+b_1}{2}$
	\end{itemize}
	\end{minipage}
	\begin{minipage}[c]{0.5\textwidth}
		\begin{tikzpicture}[scale=0.5]
		\draw[->]
		(-0.2,0) -- (8,0) node[right] {$x$};
		\draw[->]
		(0,-5) -- (0,5) node[above] {$f(x)$};
		\draw[color=blue]
		(6,0)node[above, color=black]{$c$}
		(0.5,0)node[above]{$a_1$}
		(0,-1)node[left]{$f(a_1)$}
		(7.8,0) node[below]{$b_1$}
		(7.8,0) node[above left, color=red]{$b_2$}
		(0,3.5)node[left]{$f(b_1)$}
		(4,0)node[above, color=red]{$a_2$}
		(0,-2.5)node[left, color=blue]{$y_1$};
		\draw[color=red, style=dotted]
		(4,0)--(4,-2.5)--(0,-2.5);
		\draw[color=blue, style=dotted]
		(0,3.5)--(7.8,3.5)--(7.8,0);
		\draw[color=blue, style=dotted]
		(0.5,0)--(0.5,-1)--(0,-1);
		\draw[color=blue, domain=0.1:7.9, ultra thick]
		plot(\x,0.25*\x*\x-1.5*\x-0.5) node[right]{$f(x)$};
		\draw[color=blue, domain=0.1:7.9]
		plot(\x,-3) node[right]{Intervall Schritt 1};
		\draw[color=red, domain=4:7.9]
		plot(\x,-5) node[right]{Intervall Schritt 2};
		\end{tikzpicture}
	\end{minipage}\\
	\\
	Nach Schritt 1 gilt:\\
	$[a_2,b_2]$ (neues Intervall) ist halb so groß wie $[a_1,b_1]$, und (hier)\\ $f(a_2)<0,\quad f(b_2)>0$
	\subsubsection*{Schritt 2:}
	halbiere Intervall, berechne $y_2=f(\frac{a_2+b_2}{2})$
	\begin{itemize}
		\item $y_2<0\Rightarrow[a_3,b_3]\coloneqq[\frac{a_2+b_2}{2},b_2]$
		\item $y_2>0\Rightarrow[a_3,b_3]\coloneqq ...$
		\item $y_2=0\Rightarrow ...$
	\end{itemize}
	usw.\\
	\\
	Für die Folge $([a_n,b_n])_n$ gilt:\\
	\\
	$\begin{rcases}
	(a_n)\nearrow\quad\textrm{(monoton steigend)}\\
	(b_n)\searrow\quad\textrm{(monoton fallend)}\\
	(b_n-a_n)\>0\textrm{ für }\infn
	\end{rcases}\Rightarrow$ (Intervallschachtelungsprinzip)\\
	\\
	$\lim\limits_{\infn}a_n=\lim\limits_{\infn}b_n\eqqcolon c$\\
	\\
	Es ist $f(a_n)\leq0,\qquad f(b_n)\geq0$\\
	Da $f$ stetig ist, gilt:\\
	\\
	$\begin{rcases}
	\underbrace{\lim\limits_{\infn}f(a_n)}_{\leq0}\overset{\textrm{Def. Stetigkeit}}{=}f(c)\leq 0\\
	\underbrace{\lim\limits_{\infn}f(b_n)}_{\geq0}\overset{\textrm{Def. Stetigkeit}}{=}f(c)\geq0
	\end{rcases}\Rightarrow f(c)=0$, also hat $f$ Nullstelle bei $c$\\
	\hspace*{\fill}$\square$\\
	\\
	Dieses Verfahren wird auch zur Berechnung von Nullstellen verwendet.\\
	Folgerung: 5.16
	\subsection{Satz (ZWS allgemein)}
	Sei $f\colon[a,b]\>\R$ stetig, $y$ eine Zahl zwischen $f(a)$ und $f(b)$.\\
	Dann gibt es ein $\overline{x}\in[a,b]$ mit $f(\overline{x})=y$ ($f$ nimmt auf dem Intervall $[a,b]$ jeden zwischen $f(a)$ und $f(b)$ liegenden Wert an).\\
	\subsubsection*{Beweis}
	OBdA sei $f(a)\leq y\leq f(b)$.\\
	Definiere Hilfsfunktion
	\begin{align*}
		g\colon[a,b]&\>\R\\
		x&\mapsto f(x)-y
	\end{align*}\\
	Dann ist\\
	$\begin{rcases}
	\underbrace{g(a)}_{f(a)-y,\quad f(a)\leq y}\leq0\\
	\\
	\underbrace{g(b)}_{f(b)-y,\quad f(b)\geq y}\geq0\\
	\\
	g\textrm{ ist stetig}\\
	\textrm{  (als Verknüpfung stetiger Funktionen)}
	\end{rcases}\overset{5.15\textrm{/ZWS}}{\Rightarrow}\exists c\in[a,b]$ mit $g(c)=0$\\
	\\
	$g(c)=f(c)-y=0$, dann ist $f(c)=g(c)+y=0+y=y$\\
	\hspace*{\fill}$\square$
	\subsection{Anwendung}
	\begin{itemize}
		\item[a)] Existenz von Nullstellen, Bsp $\pi$ (s. PÜ)
		\item[b)] Existenz von Lösungen einer Gleichung (PÜ 8)
		\item[c)] Kamel, Antipoden, Käsebrot, Tisch (s. Folien 06.06.2016)
		\item[d)] Der ZWS liefert auch ein Kriterium zur Existenz von stetigen Umkehrfunktionen.
	\end{itemize}
	\subsection{Definition}
	$f\colon D\>\R$ heißt \underline{(streng) monoton fallend [wachsend]}, falls gilt:\\
	Sind $x,y\in D,\quad x<y$, dann ist $f(x)\overset{(<)}{\leq} f(y)\quad[f(x)\overset{(>)}{\geq}f(y)]$.\\
	Wenn $f$ (streng) monoton wachsend oder fallend: $f$ ist (streng) monoton.
	\subsection{Satz}
	$D$ Intervall, $f\colon D\>\R$ stetig.\\
	Dann gilt: $f$ ist injektiv auf $D\Leftrightarrow\quad f$ streng monoton auf $D$.\\
	\subsubsection*{Beweis}
	\begin{itemize}
		\item["$\Leftarrow$":] Sei $x\neq y$, also etwa $x<y$.\\
		$f$ streng monoton $\Rightarrow\quad f(x)<f(y)$ bzw. $f(x)>f(y)$, also $f(x)\neq f(y)$; also $f$ injektiv.
		\item["$\Rightarrow$":](Achtung: falls $f$ nicht stetig: injektiv $\nRightarrow$ streng monoton, z.B. \begin{minipage}{0.15\textwidth}
			\begin{tikzpicture}[scale=0.2]
			\draw[->]
			(-0.2,0) -- (5,0) node[right] {};
			\draw[->]
			(0,-0.5) -- (0,5) node[above] {};
			\draw[color=blue, domain=0:2]
			plot(\x,\x) node[]{};
			\draw[color=blue, domain=2:4]
			plot(\x,6-\x) node[]{};
			\draw[color=blue] (2,4) circle [radius=8pt] node[right] {};
			\filldraw[color=blue] (2,2) circle [radius=5pt] node[right] {};
			\end{tikzpicture}
		\end{minipage})\\
		Wir zeigen (Kontraposition):\\
		$f$ nicht stetig$\Rightarrow$ $f$ nicht injektiv\\
		Sei $f$ nicht streng monoton, dann gilt für ein $y$ mit $x<y<z$ aus $D$:\\
		\begin{minipage}{0.25\textwidth}
			\begin{tikzpicture}[scale=0.25]
				\draw[->]
				(-0.2,0) -- (5,0) node[right] {};
				\draw[->]
				(0,-0.5) -- (0,5) node[above] {};
				\draw[color=blue, domain=0.7:3.5]
				plot(\x,-2*\x*\x+7.6*\x-4) node[]{};
				\draw[color=blue]
				(0.74,0) node[below]{$x$}
				(2,0) node[below]	{$y$}
				(3.25,0)node[below right]{$z$};
			\end{tikzpicture}
		\end{minipage} $f(x)\underset{\geq}{\leq} f(y),\quad f(y)\underset{\leq}{\geq}f(z)$\\
		Nach ZWS: $f$ nimmt in $[x,y]$ jeden Wert zwischen $f(x)$ und $f(y)$ an, $f$ nimmt in $[y,z]$ jeden Wert zwischen $f(y)$ und $f(z)$ an.\\
		$\Rightarrow$ mind. ein Wert wird doppelt angenommen\\
		$\Rightarrow$ $f$ ist nicht injektiv
		\hfill$\square$
	\end{itemize}
	\subsection{Satz (Minimax-Theorem von Weierstraß)}
	Jede stetige Funktion $f\colon[a,b]\>\R$
	\begin{itemize}
		\item[(i)] ist beschränkt (d.h. $\exists K\in\N$, so dass $f(x)\in[-K,K]\quad\forall x\in[a,b]$)
		\item[(ii)] besitzt sowohl Minimum als auch Maximum, d.h. $\exists x_*,x^*\in[a,b]$ mit\\
		 $f(x_*)\leq f(x)\leq f(x^*)\quad\forall x\in[a,b]$\\
		($x_*$ heißt dann $\overset{\textrm{Minimum-}}{\textrm{Minimal}}$stelle, $x^*$ heißt $\overset{\textrm{Maximum-}}{\textrm{Maximal}}$stelle)
	\end{itemize}
	\underline{Beweis} nutzt wieder Bisektionsverfahren, s. WHK Theorem 6.24\hfill$\square$\\
	(Achtung: Aussage über \underline{globale Maxima/Minima} nicht eindeutig)
	\subsection{Beispiel/Gegenbeispiel}
	\begin{itemize}
		\item[a)] \begin{minipage}[c]{0.15\textwidth}
			\begin{tikzpicture}[scale=0.25]
			\draw[->]
			(-0.2,0) -- (8,0) node[right] {$x$};
			\draw[->]
			(0,-5) -- (0,5) node[above] {$f(x)$};
			\draw[color=blue, domain=1:7.9, ultra thick]
			plot(\x,0.25*\x*\x-1.5*\x-0.5) node[right]{};
			\filldraw[color=blue] (3,-2.75) circle [radius=5pt] node[below] {Min};
			\filldraw[color=blue] (8,3.5) circle [radius=5pt] node[left] {Max};
			\draw[color=blue]
			(1,0)node[below]{$a$}
			(8,0)node[below]{$b$};
			\end{tikzpicture}
		\end{minipage}
		\item[b)]\begin{minipage}[c]{0.25\textwidth}
			\begin{tikzpicture}[scale=0.25]
			\draw[->]
			(-0.2,0) -- (7,0) node[right] {$x$};
			\draw[->]
			(0,-0.5) -- (0,5) node[above] {$f(x)$};
			\draw[color=blue, domain=2:5, ultra thick]
			plot(\x,2.5) node[right]{};
			\draw[color=blue]
			(2,0)node[below]{$a$}
			(5,0)node[below]{$b$};
			\end{tikzpicture}
		\end{minipage}
		Für $f(x)=c$: Alle $x\in[a,b]$ sind Maximal-/Minimalstellen!
		\item[c)]\begin{minipage}[c]{0.25\textwidth}
			\begin{tikzpicture}[scale=0.25]
			\draw[->]
			(-0.2,0) -- (7,0) node[right] {$x$};
			\draw[->]
			(0,-0.5) -- (0,5) node[above] {$f(x)$};
			\draw[color=blue, domain=-0.2:7, samples=50, ultra thick]
			plot(\x,{sin(2*\x r)+2}) node[]{};
			\draw[color=blue]
			(2,0)node[below]{$a$}
			(5,0)node[below]{$b$};
			\end{tikzpicture}
		\end{minipage}
		$f(x)=\sin (x)+c$, $D=[0,n\*\pi]$: mehrere Maximal-/Mininmalstellen
		\item[d)] Falls nicht alle Voraussetzungen erfüllt sind, gilt der Satz i.A. nicht, z.B.
		\begin{itemize}
			\item $f$ nicht stetig:\\
			\begin{minipage}[c]{0.5\textwidth}
				\begin{tikzpicture}[scale=0.25]
				\draw[->]
				(-0.2,0) -- (8,0) node[right] {$x$};
				\draw[->]
				(0,-0.5) -- (0,5) node[above] {$f(x)$};
				\draw[color=blue, domain=0:4, ultra thick]
				plot(\x,0.25*\x*\x) node[]{};
				\draw[color=blue, domain=4:8, ultra thick]
				plot(\x,0) node[]{};
				\filldraw[color=blue] (4,0) circle [radius=8pt] node[left] {};
				\draw[color=blue] (4,4) circle [radius=8pt] node[left] {};
				\end{tikzpicture}
			\end{minipage}
			\begin{minipage}[c]{0.35\textwidth}
				$f(x)=\begin{cases}
				x^2&x<1\\
				0&x\geq1
				\end{cases}$ auf $[0,2]$\\
				$f$ hat kein Maximum auf $[0,2]$
			\end{minipage}
			\item Definitionsbereich von $f$ nicht abgeschlossen\\
				\begin{minipage}[c]{0.5\textwidth}
					\begin{tikzpicture}[scale=0.25]
					\draw[->]
					(-0.2,0) -- (8,0) node[right] {$x$};
					\draw[->]
					(0,-0.5) -- (0,5) node[above] {$f(x)$};
					\draw[color=blue, domain=0:4, ultra thick]
					plot(\x,\x) node[]{};
					\filldraw[color=blue] (0,0) circle [radius=8pt] node[left] {};
					\draw[color=blue] (4,4) circle [radius=8pt] node[left] {};
					\end{tikzpicture}
				\end{minipage}
				\begin{minipage}[c]{0.35\textwidth}
					\begin{align*}
						f\colon[0,3)&\>\R\\
						x&\mapsto x
					\end{align*}
					kein Maximum
				\end{minipage}
				\item Definitionsbereich von $f$ nicht beschränkt:\\
				\begin{minipage}[c]{0.5\textwidth}
					\begin{tikzpicture}[scale=0.25]
					\draw[->]
					(-0.2,0) -- (8,0) node[right] {$x$};
					\draw[->]
					(0,-0.5) -- (0,5) node[above] {$f(x)$};
					\draw[color=blue, domain=0.25:8, ultra thick]
					plot(\x,1/\x) node[]{};
					\end{tikzpicture}
				\end{minipage}
				\begin{minipage}[c]{0.35\textwidth}
					\begin{align*}
						f\colon(0,\infty)&\>(0,\infty)\\
						x&\mapsto \frac{1}{x}
					\end{align*}
					kein Maximum/Minimum
				\end{minipage}
		\end{itemize}
		Bemerkung: Es gibt aber auch nicht stetige Funktionen, die Maximum/Minimum besitzen! Z.B.\begin{minipage}{0.15\textwidth}
			\begin{tikzpicture}[scale=0.2]
			\draw[->]
			(-0.2,0) -- (7,0) node[right] {};
			\draw[->]
			(0,-0.5) -- (0,5) node[above] {};
			\draw[color=blue, domain=1:3.5]
			plot(\x,2) node[]{};
			\draw[color=blue, domain=3.5:6]
			plot(\x,4) node[]{};
			\draw[color=blue] (3.5,2) circle [radius=8pt] node[right] {};
			\filldraw[color=blue] (3.5,4) circle [radius=5pt] node[right] {};
			\end{tikzpicture}
		\end{minipage}
	\end{itemize}
	\subsection{Bemerkung}
	Satz 5.20 liefert nur die Existenz von Maxima/Minima, aber keine Aussage darüber, wie man Maximum-/Minimumstelle (insbesondere "lokale") finden kann!\\
	$\>$ 6 Differentialrechnung
	\newpage
	\section{Differenzierbare Funktionen}
	\subsection{Vorbemerkung}
	s. Folien\\
	Idee:\\
	\begin{tikzpicture}[scale=0.7]
	\draw[->]
	(-0.2,0) -- (7,0) node[right] {};
	\draw[->]
	(0,-0.2) -- (0,5) node[above] {};
	\draw[color=blue, domain=0:5.2]
	plot(\x,\x*\x*0.2) node[above]{f};
	\draw[color=red, domain=1:5.2]
	plot(\x,1.3*\x-1.5) node[right]{\textrm{Sekante}};
	\draw[color=black]
	(1.5,0) node[below]{$x_0$}
	(3,-1)node[below]{$x_0+$kleineres $h$}
	(5,0) node[below]{$x_0+h$};
	\draw[color=black, style=dotted]
	(1.5,0)--(1.5,0.45)
	(5,0)--(5,5);
	\draw[color=green, domain=1:5.2]
	plot(\x,0.9*\x-0.9)node[right]{\textrm{bessere Sekante}};
	\draw[color=green, style=dotted]
	(3,0)--(3,1.8);
	\end{tikzpicture}\\
	Die Gerade (Sekante) durch die Punkte $(x_0|f(x_0))$ und $(x_0+h|f(x_0+h))$ hat die Steigung $\frac{f(x_0+h)-f(x_0)}{x_0+h-x_0}$ ('Differenzenquotient').\\
	Je kleiner $h$, desto besser beschreibt die Sekante die 'Steigung von $f$ in $x_0$'.\\
	Für $h\>0$ erhält man (falls Grenzwert existiert) die \underline{Tangente} in $x_0$ mit Steigung $\lim\limits_{h\>0}\frac{f(x_0+h)-f(x_0)}{h}$.\\
	\\
	In diesem Kapitel ist $I\subseteq\R$ ein offenes Intervall.
	\subsection{Definition}
	$f\colon I\>\R,\qquad x_0\in I$
	\begin{itemize}
		\item[a)] $f$ heißt \underline{differenzierbar (diffbar)} in $x_0$ (an der Stelle $x_0$), falls $\lim\limits_{h\>0}\frac{f(x_0+h)-f(x_0)}{h}$ existiert. Dieser Grenzwert  heißt \underline{erste Ableitung von $f$ an der Stelle $x_0$} und wird mit $f^\prime(x_0)$ ['$f$ Strich'] oder $\dfrac{f}{x}(x_0)$ bezeichnet.
		\item[b)] Ist $f$ in jedem $x_0\in I$ diffbar, so heißt $f$ \underline{differenzierbar (auf $I$)} und man nennt die Funktion $f^\prime\colon I\>\R,\qquad x\mapsto f^\prime(x)$ die \underline{Ableitung(-sfunktion) von $f$}.
	\end{itemize}
	\subsection{Beispiel}
	\begin{itemize}
		\item[a)] \begin{align*}
			f\colon\R&\>\R\\
			x&\mapsto x^2,\qquad x_0=2\\
			&\\
			f^\prime(2)&=\lim\limits_{h\>0}\frac{f(2+h)-f(2)}{h}\\
			&=\lim\limits_{h\>0}\frac{(2+h)^2-2^2}{h}\\
			&=\lim\limits_{h\>0}\frac{4+4h+h^2-4}{h}\\
			&=\lim\limits_{h\>0}(4+h)\\
			&=4
		\end{align*}
		allgemein für $x^2$:
		\begin{align*}
			f^\prime(x)&=\lim\limits_{h\>0}\frac{f(x+h)-f(x)}{h}\\
			&=\lim\limits_{h\>0}\frac{(x+h)^2-x^2}{h}\\
			&=\lim\limits_{h\>0}\frac{x^2+2xh+h^2-x^2}{h}\\
			&=\lim\limits_{h\>0}(2x+h)\\
			&=2x\qquad\forall x\in\R
		\end{align*}
		\underline{Bemerkung:} '$\lim\limits_{h\>0}$' nicht weglassen/vergessen!
		\item[b)] konstante Fkt:\begin{align*}
			f(x)&=c\qquad\forall x\in\R\\
			&\\
			f^\prime(x)&=\lim\limits_{h\>0}\frac{f(x+h)-f(x)}{h}\\
			&=\lim\limits_{h\>0}\frac{c-c}{h}\\
			&=\lim\limits_{h\>0}\frac{0}{h}\\
			&=0
		\end{align*}
		\item[c)] $f(x)=x^n\qquad (n\in\N)$\\
		$\Rightarrow f^\prime(x)=n\*x^{n-1}$\\
		(Beweis durch vollst. Induktion)
		\item[d)] \begin{align*}
			f\colon\R^+&\>\R\\
			f(x)&=\frac{1}{x}\\
			&\\
			f^\prime(x)&=\lim\limits_{h\>0}\frac{f(x+h)-f(x)}{h}\\
			&=\lim\limits_{h\>0}\frac{\frac{1}{x+h}-\frac{1}{x}}{h}\\
			&=\lim\limits_{h\>0}\frac{x-(x+h)}{x\*(x+h)}\*\frac{1}{h}\\
			&=\lim\limits_{h\>0}\frac{-h}{x^2+xh}\*\frac{1}{h}\\
			&=\lim\limits_{h\>0}\frac{-1}{x\*(x+h)}\\
			&=-\frac{1}{x^2}
		\end{align*}
		\item[e)] \begin{align*}
			f(x)&=\sin x\\
			&\\
			f^\prime(x)&=\lim\limits_{h\>0}\frac{\sin(x+h)-\sin x}{h}\\
			&=\lim\limits_{h\>0}\frac{\overbrace{\sin x\*\cos x+\cos x\*\sin x}^{\textrm{Additionstheorem}}-\sin x}{h}\\
			&=\lim\limits_{h\>0}\sin x\*\underbrace{\frac{\cos (h)+1}{h}}_{\>0\textrm{, Mitteilung oder Übung}}+\cos x\*\underbrace{\frac{\sin h}{h}}_{\>1\textrm{, Bsp. 5.14 g}}\\
			&=\sin (x)\*0+\cos(x)\*1\\
			&=\cos x
		\end{align*}
		\item[f)] $f(x)=\cos x$\\
		$f^\prime(x)=-\sin x\qquad$(Übung)
	\end{itemize}
	\subsection{Satz}
	$f\colon I\>\R,\qquad x_0\in I$\\
	Dann sind äquivalent:
	\begin{itemize}
		\item[a)] $f$ ist in $x_0$ diffbar.
		\item[b)] Es gibt eine Funktion $R\colon I\>\R$ stetig in $x_0$, $R(x_0)=0$, und ein $m\in\R$, so dass gilt:\\
		$f(x)=\underbrace{\overbrace{f(x_0)}^{\textrm{Zahl}}+m(x-x_0)}_{\textrm{Gerade durch }(x_0|f(x_0)\textrm{ mit Steigung }m\textrm{ = Tangente})}+\underbrace{R(x)(x-x_0)}_{\textrm{wird 0 an der Stelle  }x_0\textrm{, wird klein in der Nähe von }x_0}$\\
		Gilt b), so ist $m=f^\prime(x_0)$ (d.h. die \underline{Ableitung ist die Steigung der Tangente}).\\
		b) besagt: $f$ lässt sich in der Nähe von $x_0$ gut durch eine (affin-)lineare Funktion (Gerade) approximieren.	
	\end{itemize}
	\subsubsection*{Beweis:}
	\begin{itemize}
		\item[a$\Rightarrow$b:] Setze $R(x)=\begin{cases}
		\frac{f(x)-f(x_0)}{x-x_0}-f^\prime(x_0)&\quad x\neq x_0\\
		0&\quad x=x_0
		\end{cases}$, dann ist nach Voraussetzung \begin{align*}
			\lim\limits_{x\>x_0}R(x)&=\lim\limits_{h\>0}R(x_0+h)\\
			&=\underbrace{\lim\limits_{h\>0}\frac{f(x_0+h)-f(x_0)}{x_0+h-x_0}}_{\textrm{existiert, da }f\textrm{ diffbar und }=f^\prime(x_0)}-f^\prime(x_0)\\
			&=0
		\end{align*} nach Definition 6.2, weil $f$ diffbar in $x_0$;\\
		also ist $R$ stetig mit $R(x_0)=0$.
		\item[b$\Rightarrow$a:] Sei $f(x)=f(x_0)+m(x-x_0)+R(x)(x-x_0)$ (Voraussetzung)\\
		$\Rightarrow\quad f(x_0+h)=f(x_0)+m(x_0+h-x_0)+R(x_0+h-x_0)$\\
		$\Leftrightarrow\quad \frac{f(x_0+h)-f(x_0)}{h}=m+R(x_0+h)$\\
		$\Rightarrow\quad \lim\limits_{h\>0}\frac{f(x_0+h)-f(x_0)}{h}=\lim\limits_{h\>0}m+R(x_0+h)$\\
		$\qquad\qquad=m+\underbrace{R(x_0)}_{=0\textrm{ (Voraussetzung)}}$\\
		$\qquad\qquad=m$\\
		D.h. $\lim$ existiert, $f$ ist diffbar in $x_0$, $f^\prime(x_0)=m$
	\end{itemize}
	\hfill$\square$
	\subsection{Korollar (Folgerung)}
	Ist $f\colon I\>\R$ diffbar in $x_0\in I$, dann ist $f$ auch stetig in $x_0$ (Beweis folgt aus 6.4 b).
	\subsection{Bemerkung/Beispiel}
	Umkehrung von 6.5 gilt nicht!\\
	\\
	Die Betragsfunktion $f\colon \R\>\R,\quad x\mapsto |x|$ ist bei $x_0=0$ stetig, aber nicht diffbar!\\
	$f(x)=\begin{cases}x&\qquad x\geq0\\
	-x&\qquad x<0
	\end{cases}$\\
	Denn: $\lim\limits_{h\>0}\frac{f(x+h)-f(x)}{h}=\lim\limits_{h\>0}\frac{|x+h|-|x|}{h}$ existiert nicht für $x=0$!
	\begin{itemize}
		\item $\lim\limits_{h\>0+}\frac{|0+h|-0}{h}=\lim\limits_{h\>0+}\frac{h}{h}=1$
		\item $\lim\limits_{h\>0-}\frac{|0+h|-0}{h}=\lim\limits_{h\>0-}\frac{-h}{h}=-1$
		\item $\Rightarrow 1\neq-1$
	\end{itemize}
	Also: stetig $\nRightarrow$ differenzierbar\\
	${}^{}\qquad\qquad{}^{}\quad\Leftarrow$
	\subsection{Satz (Ableitungsregeln)}
	$f,g\colon I\>\R$ diffbar in $x\in I$. Dann sind auch
	\begin{itemize}
		\item $c\*f$, $c\in\R$
		\item $f\pm g$
		\item $f\*g$ und
		\item $\frac{f}{g}$ (falls $g(x)\neq0$)
	\end{itemize}
	diffbar in $x$ mit
	\begin{itemize}
		\item[a)] $(c\*f)'(x)=c\*f^\prime(x)$
		\item[b)] $(f\pm g)'(x)=f^\prime(x)\pm g^\prime(x)$
		\item[c)] $(f\*g^\prime)(x)=f^\prime(x)\*g(x)+f(x)\*g^\prime(x)$ [Produktregel]
		\item[d)] $(\frac{f}{g})'(x)=\frac{f^\prime(x)\*g(x)-f(x)\*g^\prime(x)}{g^2(x)}$ [Quotientenregel]
	\end{itemize}
	\subsubsection*{Beweis:}
	(nur a,c; Rest ähnlich)
	\begin{itemize}
		\item[a)] $\frac{c\*f(x+h)-c\*f(x)}{h}=c\*\underbrace{\frac{f(x+h)-f(x)}{h}}_{\>f^\prime(x)\textrm{ für }h\>0}$\\
		${}^{}\qquad\qquad{}^{}\quad\overset{h\>0}{\>}c\*f^\prime(x)$
		\item[c)] \begin{align*}
			\frac{(f\*g)(x+h)-(f\*g)(x)}{h}\quad&\overset{\textrm{Regeln für Fkt.}}{=}\quad\frac{f(x+h)\*g(x+h)-f(x)\*g(x)}{h}\\
			\>\textrm{Einschiebetrick: }&-f(x)\*g(x+h)+f(x)\*g(x+h)\\
			&=\underbrace{\frac{f(x+h)-f(x)}{h}}_{\overset{h\>0}{\>f^\prime(x)}}\*\underbrace{g(x+h)}_{\overset{h\>0}{\>}g(x)\textrm{ (da stetig)}}\\
			&\qquad+\underbrace{\frac{g(x+h)-g(x)}{h}}_{\overset{h\>0}{\>}g^\prime(x)}\*f(x)\\
			&\overset{h\>0}{\>}f^\prime(x)\*g(x)+f(x)\*g^\prime(x)
		\end{align*}
	\end{itemize}
	\hfill$\square$
	\subsection{Beispiel}
	\begin{itemize}
		\item[a)] Jedes Polynom ist diffbar (wegen a, b, c); jede rationale Funktion ist diffbar (a, b, d).
		\item[b)] $(4x^3+7x+5)^\prime=4\*3x^2+7+0=12x^2+7$
		\item[c)] $(\frac{3}{x})^\prime=\frac{0\*x-3\*1}{x^2}=-\frac{3}{x^2}$
		\item[d)] $(\frac{1}{x^2})^\prime=(\frac{1}{x}\*\frac{1}{x})^\prime=-\frac{1}{x^2}\*\frac{1}{x}+\frac{1}{x}\*(-\frac{1}{x^2})$\\
		${}^{}\qquad=-\frac{2}{x^3}$
		\item[e)] Allgemein:\\
		$f(x)=x^{-n}=\frac{1}{x^n},\qquad n\in\N$\\
		$\Rightarrow f^\prime(x)=-n\*x^{-n-1}=-\frac{n}{x^{n+1}}$
		\item[f)]$(\frac{\sin x}{x})^\prime=\frac{\cos(x)\*x-\sin(x)\*1}{x^2}$
		${}^{}\qquad\frac{x\*\cos(x)-\sin x}{x^2}$
		\item[g)] \begin{align*}
			(\tan x)^\prime&=(\frac{\sin x}{\cos x})^\prime\\
			&=\frac{\cos x\*\cos x-\sin x\*(-\sin x)}{(\cos x)^2}\\
			&\overset{\textrm{Additionstheoreme}}{=}\quad\frac{1}{(\cos x)^2}\\
			\textrm{oder (kürzen, aufspalten) }&=\frac{(\cos x)^2}{(\cos x)^2}+\frac{(\sin x)^2}{(\sin x)^2}\\
			&=1+(\tan x)^2
		\end{align*}
	\end{itemize}
	\subsection{Satz (Kettenregel)}
	Die Komposition $f\circ g$ zweier diffbarer Funktionen $f,g$ ist diffbar und es gilt:\\
	$(f\circ g)^\prime=(f^\prime\circ g)\*g^\prime$\\
	Beweis ähnlich wie Satz 6.7\hfill$\square$\\
	Zum Merken:\\
	$(f\circ g)(x)=\overset{\textrm{'äußere'}}{f}(\overset{'innere' Funktion}{g(x)})$\\
	$\>\qquad (f\circ g)^\prime(x)=f^\prime(g(x))\*g^\prime(x)$
	\subsection{Beispiel}
	\begin{itemize}
		\item[a)] $((3x^2-x)^5)^\prime=5\*(3x^2-x)^4\*(6x-1)$\\
		innere Fkt: $g(x)=3x^2-x$\\
		äußere Fkt: $f(x)=x^5$
		\item[b)] $(\sin(3x))^\prime=(\cos 3x)\*3$\\
		innere Fkt: $g(x)=3x$\\
		äußere Fkt: $f(x)=\sin x$
	\end{itemize}
	\subsection{Satz (Ableitung der Umkehrfunktion)}
	$I, J\subseteq\R$ Intervalle, $f\colon I\>J$ bijektiv, diffbar in $x_0\in I$ mit $f^\prime(x_0)\neq0$\\
	Dann ist auch die Umkehrfunktion $f^{-1}\colon J\>I$ diffbar in $y_0=f(x_0)$ und es gilt:\\
	$(f^{-1})^\prime(y_0)=\underbrace{\frac{1}{f^\prime(x_0)}}_{f^\prime(x_0)\neq0}=\frac{1}{f^\prime(f^{-1}(y_0))}$\\
	Beweis mittels Definition der Ableitung (Differenzenquotient), hier nur Merkregel.
	\begin{align*}
		&&y&=f(f^{-1}(y))\\
		&\mathclap{\overset{\textrm{nach }y\textrm{ ableiten mit 6.9}}{\Rightarrow}}&1&=f^\prime(f^{-1}(y))\*(f^{-1})^\prime(y)\\
		&\Leftrightarrow&(f^{-1})^\prime(y)&=\frac{1}{f^\prime(f^{-1}(y))}
	\end{align*}
	\subsection{Beispiel}
	\begin{itemize}
		\item[a)] $f\colon\R^+\>\R^+,\qquad f(x)=\sqrt{x}$\\
		Was ist $f^\prime(x)$?\\
		$f$ ist Umkehrfunktion $h^{-1}$ von $h\colon\R^+\>\R^+,\qquad h(x)=x^2$. $h^\prime(x)=2x$
		\begin{align*}
			(h^{-1})^\prime(y)&=\frac{1}{h^\prime(h^{-1}(y))}\\
			&=\frac{1}{2\*\sqrt{y}}\\
			&\\
			\Rightarrow f^\prime(x)&=\frac{1}{2\sqrt{x}}
		\end{align*}
		\item[b)] $f(x)=\sqrt[3]{x}=x^\frac{1}{3}$ ist $h^{-1}$ von $h(x)=x^3$.\\
		...\\
		$\Rightarrow f^\prime(x)=\frac{1}{3}x^{-\frac{2}{3}}$\\
		da:\\
		$h^\prime(x)=3x^2,\qquad (h^{-1})^\prime(y)=\frac{1}{3(\sqrt[3]{y})^2}=\frac{1}{3(y^\frac{1}{3})^2}$\\
		$h^\prime(0)=0\Rightarrow$ Ableitung von $h^{-1}$ existiert nicht in $x=0$\\
		$\Rightarrow \quad y\in\R\setminus\{0\}$
		\item[c)] \begin{align*}
			f\colon \R&\>\R_{>0}\qquad f(y)=\e^y\\
			f^{-1}\colon \R_{>0}&\>\R\qquad f^{-1}(x)=\ln x\\
			\Rightarrow\qquad (\ln x)^\prime&=\frac{1}{f^\prime(f^{-1}(x))}\\
			&=\frac{1}{\e^{\ln x}}=\frac{1}{x}\\
			\Rightarrow\qquad (\ln|x|)^\prime&=\begin{cases}\frac{1}{x}\qquad x>0\\
			\frac{1}{-x}\*(-1)=\frac{1}{x}\qquad x<0\textrm{(Kettenregel)}
			\end{cases}\\
			&=\frac{1}{x}\qquad x\neq 0
		\end{align*}
	\end{itemize}
	Wir nutzen 6.12 c, um Produkte von Funktionen abzuleiten
	\subsection{Bemerkung (Logarithmische Ableitung)}
	Für $f\colon\R\>\R_{>0}$, $f$ diffbar ist $(\ln f(x))^\prime=\frac{f^\prime(x)}{f(x)}$ (6.12c, Kettenregel).\\
	\subsubsection*{Beispiel}
	\begin{align*}
		f(x)&=\e^x(\sin(x)+2)\*x^5\qquad x>0\quad(\Rightarrow f(x)>0)\\
		\ln f(x)&=x+\ln(\sin(x)+2)+5\*\ln x\\
		(\ln f(x))^\prime&=1+\frac{\cos x}{\sin(x)+2}+5\*\frac{1}{x}=\frac{f^\prime(x)}{f(x)}\\
		f^\prime(x)&=(1+\frac{\cos x}{\sin(x)+2}+\frac{5}{x})(\e^x(\sin(x)+2)\*x^5)\\
		&=\e^x[(\sin(x)+2)\*x^5+\cos(x)\*x^5+(\sin(x)+2)\*5x^4]
	\end{align*}
	\subsubsection*{Bemerkung}
	Man kann zeigen, dass die logarithmische Ableitung auch auf Funktionen mit Werten in ganz $\R$ anwendbar ist (berechne $(\ln|f(x)|)^\prime$ und stetige Fortsetzung in $x$ mit $f(x)=0$).\\
	$\Rightarrow$ Beispiel gilt sogar $\forall x\in \R$
	\subsection{Satz (Ableitungen der elementaren Funktionen)}
	\begin{itemize}
		\item $(a^x)^\prime=(\ln a)a^x\qquad a\in\R_{>0}$
		\item $(x^\alpha)^\prime=\alpha x^{\alpha-1}\qquad\alpha\in\R,\quad x>0$
		\item $(x^x)^\prime=(1+\ln x)x^x\qquad x>0$
	\end{itemize}
	\subsubsection*{Beweis}
	\begin{itemize}
		\item $(a^x)^\prime\overset{\textrm{Logarithmus}}{=}(\e^{\ln(a^x)})^\prime=(\e^{x\*\ln a})^\prime\overset{\textrm{Kettenregel}}{=}(\ln a)a^x$
		\item Rest analog
	\end{itemize}
	Ab jetzt:
	\subsubsection*{Kurvendiskussion}
	\subsection{Definition}
	$f\colon D\>\R$ besitzt in $x_0\in D$ ein \underline{lokales} $\overset{\textrm{\underline{Extremum}}}{\textrm{\underline{Maximum/Minimum}}}$, wenn es ein Intervall $U=(x_0-s,x_0+s)\subseteq D, s>0$ gibt, so dass $f(x_0)\underset{\geq}{\leq}f(x)\quad\forall x\in U$ ($U$ heißt Umgebung von $x_0$).\\
	\begin{minipage}[c]{0.5\textwidth}
		$f$ besitzt ein\\
		 \underline{globales Maximum/Minimum}\\
		  in $x_0$, wenn $f(x_0)\underset{\geq}{\leq}f(x)\quad\forall x\in\R$
	\end{minipage}
	\begin{minipage}[c]{0.5\textwidth}
		\begin{tikzpicture}[scale=0.5]
		\draw[->]
		(-0.2,0) -- (4.2,0) node[right] {$x$};
		\draw[->]
		(0,-0.2) -- (0,4.2) node[above] {$f(x)$};
		\draw[color=blue]
		(1.17,4) node[above]{$U$}
		(1.15,2.5) node[above]{lokal}
		(3.6,4) node[above]{global};
		\draw[style=dotted, color=blue]
		(0.1,0)--(0.1,3.6)
		(2,0)--(2,3.6)
		(0,3.6)--(2,3.6);
		\draw[color=blue, domain=0.4:4, ultra thick]
		plot(\x,-0.7365967*\x*\x*\x*\x+6.82206682*\x*\x*\x-21.551282*\x*\x+27.3533411*\x-9.7867132) node[right]{$f$};
		\end{tikzpicture}
	\end{minipage}\\
	\subsection{Satz (notwendige Bedingungen für lokale Extrema)}
	Sei $f\colon D\>\R$ diffbar. Falls $f$ in $x_0\in D$ ein Extremum besitzt, dann ist $f^\prime(x_0)=0$.\\
	\subsubsection*{Beweis}
	Sei $U=(x_0-s,x_0+s),\quad s>0,\quad U\subseteq D$ und $\underbrace{f(x_0)\geq f(x)}_{\textrm{Maximum}}\qquad\forall x\in U$.\\
	$\underset{f\textrm{ diffbar}}{\Rightarrow}f^\prime(x_0)=\lim\limits_{h\>0}\frac{f(x_0+h)-f(x_0)}{h}$\\
	Da $f(x_0)\geq f(x_0+h)\quad\forall h<s$, ist $f(x_0+h)-f(x_0)\leq 0\quad\forall h<s$\\
	$\Rightarrow\lim\limits_{h\>0+}\frac{\overbrace{f(x_0+h)-f(x_0)}^{\leq0}}{\underbrace{h}_{\geq 0}}\leq0$ und $\lim\limits_{h\>0-}\frac{\overbrace{f(x_0+h)-f(x_0)}^{\leq0}}{\underbrace{h}_{\leq0}}\geq0$\\
	$\Rightarrow f^\prime(x_0)=0$ (für Minimum analog)\hfill$\square$
	\subsection{Bemerkung}
	$f^\prime(x_0)=0$ notwendige, aber nicht hinreichende Bedingung
	\subsubsection*{Beispiel}
	\begin{minipage}[c]{0.5\textwidth}
		$f(x)=x^3$ hat in $x=0$ Sattelpunkt mit Steigung 0
	\end{minipage}
	\begin{minipage}[c]{0.5\textwidth}
		\begin{tikzpicture}[scale=0.5]
		\draw[->]
		(-4.2,0) -- (4.2,0) node[right] {$x$};
		\draw[->]
		(0,-4.2) -- (0,4.2) node[above] {$f(x)$};
		\draw[color=blue, domain=-2:2]
		plot(\x,\x*\x*\x*0.5) node[right]{$f(x)=x^3$};
		\end{tikzpicture}
	\end{minipage}
	$f$ hat lokales Extremum in $x_0$ $\underset{\nLeftarrow}{\Rightarrow}f^\prime(x_0)=0$\\
	Für hinreichende Bedingung und weitere Ergebnisse der Kurvendiskussion: die zentralen Sätze der Differentialrechnung
	\subsection{Satz (Mittelwertsätze, Satz von Rolle)}
	Seien $f,g\colon[a,b]\>\R$ stetig und diffbar auf $(a,b)$; $g^\prime(x)\neq0\quad\forall x\in(a,b)$.
	\begin{itemize}
		\item[1)]
		\begin{minipage}[c]{0.5\textwidth}
			$\Rightarrow\quad\exists\xi\in(a,b)\colon\frac{f(b)-f(a)}{b-a}=f^\prime(\xi)$\\
			1. Mittelwertsatz
		\end{minipage}
		\begin{minipage}[c]{0.5\textwidth}
			\begin{tikzpicture}[scale=0.5]
			\draw[->]
			(-0.2,0) -- (4.2,0) node[right] {$x$};
			\draw[->]
			(0,-0.2) -- (0,4.2) node[above] {$f(x)$};
			\draw[color=blue]
			(0.6,0) node[below]{$a$}
			(4.2,0) node[below]{$b$};
			\draw[color=black]
			(4.2,2.7) node[right]{Tangente}
			(2.1,0) node[below]{$\xi_2$}
			(3.6,0) node[below]{$\xi_1$};
			\draw[color=black, style=dotted]
			(2.1,0)--(2.1,1.45)
			(3.6,0)--(3.6,3.95);
			\draw[style=dotted, color=blue, domain=0.5:4.4]
			plot(\x,0.25416666*\x+0.0915) node[right]{};
			\draw[style=dotted, color=black, domain=0.5:4.4]
			plot(\x,0.25416666*\x+0.0915+0.8);
			\draw[style=dotted, color=black, domain=0.5:4.4]
			plot(\x,0.25416666*\x+0.0915+2.9);
			\draw[color=blue, domain=0.6:4.2, ultra thick]
			plot(\x,-0.7365967*\x*\x*\x*\x+6.82206682*\x*\x*\x-21.551282*\x*\x+27.3533411*\x-9.7867132) node[right]{$f$};
			\end{tikzpicture}
		\end{minipage}\\
		\item[2)] 
		\begin{minipage}[c]{0.5\textwidth}
			$f(a)=f(b)\Rightarrow \quad\exists\xi\in(a,b)\colon f^\prime(\xi)=0$\\
			Satz von Rolle
		\end{minipage}
		\begin{minipage}[c]{0.5\textwidth}
			\begin{tikzpicture}[scale=0.5]
			\draw[->]
			(-0.2,0) -- (5,0) node[right] {$x$};
			\draw[->]
			(0,-0.5) -- (0,5) node[above] {$f(x)$};
			\draw[color=blue, domain=0.8:3, ultra thick]
			plot(\x,-2*\x*\x+7.6*\x-4) node[]{};
			\draw[style=dotted, color=black]
			(2,0)--(2,3.2);
			\draw[color=black, style=dotted, domain=0.8:3]
			plot(\x,3.2) node[right]{Tangente};
			\draw[color=black]
			(2,0) node[below]	{$\xi$};
			\draw[color=blue]
			(0.74,0) node[below]{$a$}
			(3,0)node[below right]{$b$};
			\draw[color=blue, style=dotted, domain=0.8:3]
			plot(\x,0.8) node[right]{};
			\end{tikzpicture}
		\end{minipage}
		\item[3)] $\Rightarrow \quad\exists\xi\in(a,b)\colon\frac{f(b)-f(a)}{g(b)-g(a)}=\frac{f^\prime(\xi)}{g^\prime(\xi)}$\\
		2. Mittelwertsatz
	\end{itemize}
	\subsubsection*{Beweis}
	\begin{itemize}
		\item[2)] $f$ stetig auf $[a,b]\Rightarrow\quad f$ besitzt Maximum/Minimum $M\in\R/m\in\R$ in $[a,b]$, d.h. $m\leq f(x)\leq M$
		\begin{itemize}
			\item[1. Fall] Beide Extreme werden auf Rand angenommen\\
			$f(a)=f(b)\Rightarrow m=M$\\
			$\Rightarrow\quad f$ konstant $\Rightarrow$ $f^\prime(\xi)=0\quad\forall\xi\in(a,b)$
			\item[2. Fall] Ein Extremum wird nicht auf Rand angenommen\\
			$\Rightarrow\quad\exists\xi\in(a,b)\colon\xi$ ist Extremstelle\\
			$\overset{\textrm{6.16}}{\Rightarrow}f^\prime(\xi)=0$
		\end{itemize} 
		\item[3)] Es ist $g(b)\neq g(a)$, denn sonst gäbe es $x\in(a,b)$ mit $g^\prime(x)=0$ (in Voraussetzung ausgeschlossen, Rolle).\\
		Hilfsfunktion $h(x)=f(x)-\frac{f(b)-f(a)}{g(b)-g(a)}\*g(x)$\\
		Man kann nachrechnen, dass $h(a)=h(b)$\\
		$h$ stetig auf $[a,b]$ und $h$ diffbar auf $(a,b)$\\
		$\underset{\textrm{Rolle}}{\Rightarrow}\exists\xi\in(a,b)\colon f^\prime(\xi)=0$\\
		$\Rightarrow \frac{f^\prime(\xi)}{g^\prime(\xi)}=\frac{f(b)-f(a)}{g(b)-g(a)}$
		\item[1)] folgt aus 3) für $g(x)=x$
	\end{itemize}
	\hfill$\square$
	\subsection{Satz (Monotoniekriterium)}
	Sei $f\colon[a,b]\>\R$ stetig und auf $(a,b)$ diffbar.
	\begin{itemize}
		\item[i)] $f^\prime(x)\underset{\leq}{\geq}0\quad\forall x\in(a,b)\Leftrightarrow f$ monoton $\underset{\textrm{fallend}}{\textrm{wachsend}}$ auf $[a,b]$
		\item[ii)] $f^\prime(x)\underset{<}{>}0\quad\forall x\in(a,b) \overset{\nLeftarrow}{\Rightarrow} f$ streng monoton $\underset{\textrm{fallend}}{\textrm{wachsend}}$ auf $[a,b]$
		\item[iii)] $f^\prime(x)=0\quad\forall x\in(a,b)\Leftrightarrow f$ konstant auf $[a,b]$		
	\end{itemize}
	\subsubsection*{Beweis}
	\begin{itemize}
		\item[i)] \begin{itemize}
			\item['$\Rightarrow$'] Sei $a\leq x_1<x_2\leq b$\\
			$\Rightarrow\exists\xi\in(x_1,x_2)\colon f(x_2)\*f(x_1)=\underbrace{\underbrace{f^\prime(\xi)}_{>0}\*\underbrace{(x_2-x_1)}_{>0}}_{>0}$\\
			$\Rightarrow f(x_2)\geq f(x_1)$
			\item['$\Leftarrow$'] Sei $f$ monoton wachsend auf $[a,b], f$ diffbar auf $(a,b)$\\
			$\Rightarrow f^\prime(x)=\lim\limits_{h\>0}\frac{f(x+h)-f(x)}{h}\quad\forall x\in(a,b)$\\
			Da $\frac{\overbrace{f(x+h)-f(x)}^{\geq0}}{\underbrace{h}_{>0}}\geq0$ für $h>0$ und\\
			$\frac{\overbrace{f(x+h)-f(x)}^{\leq0}}{\underbrace{h}_{<0}}\geq0$ für $h<0$, ist $f^\prime(x)\geq0\quad\forall x\in(a,b)$.
		\end{itemize}
		\item[ii), iii)] analog
	\end{itemize}
	\hfill$\square$
	\subsection{Satz (Hinreichende Bedingung für lokale Extrema I)}
	Sei $f\colon(a,b)\>\R$ diffbar auf $(a,b)$ und sei $x_0\in(a,b)$ mit $f(x_0)=0$
	\begin{itemize}
		\item[i)] $\quad$\\
		\begin{minipage}[c]{0.5\textwidth}
			 $f^\prime(x)\underset{\geq}{\leq}0\quad\forall x\in(x_0-s,x_0)$ und\\
			 $f^\prime(x)\underset{\leq}{\geq}0\quad\forall x\in(x_0,x_0+s)$, $s>0$\\
			 $\Rightarrow f$ hat lokales $\underset{\textrm{Maximum}}{\textrm{Minimum}}$ in $x_0$\\
			 Vorzeichenwechsel
		\end{minipage}
		\begin{minipage}[c]{0.5\textwidth}
			\begin{tikzpicture}[scale=0.5]
			\draw[->]
			(-0.2,0) -- (5,0) node[right] {$x$};
			\draw[->]
			(0,-0.5) -- (0,5) node[above] {$f(x)$};
			\draw[color=blue, domain=0.8:3, ultra thick]
			plot(\x,-2*\x*\x+7.6*\x-4) node[]{};
			\draw[color=red, domain=0.8:3, ultra thick]
			plot(\x,2*\x*\x-7.6*\x+7.8) node[]{};
			\draw[style=dotted, color=black]
			(1.9,0)--(1.9,3.2);
			\draw[color=blue, domain=0.8:2]
			plot(\x,4.7) node[below]{$\geq0\qquad$};
			\draw[color=blue, domain=2.4:3]
			plot(\x,4.7) node[below]{$\leq0$};
			\draw[color=red, domain=2.4:3]
			plot(\x,4.8) node[above]{$\geq0$};
			\draw[color=red, domain=0.8:2]
			plot(\x,4.8) node[above]{$\leq0\qquad$};
			\draw[color=black]
			(2.2,0) node[below]	{$x_0$};
			\end{tikzpicture}
		\end{minipage}
		\item[ii)] $\quad$\\
		\begin{minipage}[c]{0.5\textwidth}
			$f^\prime(x)\underset{>}{<}0\quad\forall x\in(x_0-s,x_0)\cap(x_0,x_0+s)$, $s>0$\\
			$\Rightarrow f$ hat kein lokales Extremum in $x_0$
		\end{minipage}
		\begin{minipage}[c]{0.5\textwidth}
			\begin{tikzpicture}[scale=0.5]
			\draw[->]
			(-0.2,0) -- (5,0) node[right] {$x$};
			\draw[->]
			(0,-0.5) -- (0,5) node[above] {$f(x)$};
			\draw[color=blue, domain=0:4.5, ultra thick]
			plot(\x, 0.125*\x*\x*\x-0.5*\x*\x+0.5*\x-0.25*\x*\x+\x-1+2) node[]{};
			\draw[color=black, style=dotted]
			(2,2)--(2,0) node[below]{$x_0$};
			\draw[color=blue, domain=0.8:2]
			plot(\x,4.7) node[below]{$\geq0\qquad$};
			\draw[color=blue, domain=2.4:3]
			plot(\x,4.7) node[below]{$\geq0$};
			\end{tikzpicture}
		\end{minipage}
	\end{itemize}
	\subsubsection*{Beweis (für Minimum)}
	Zu zeigen: $f(x)\geq f(x_0)\quad\forall x\in(x_0-s,x_0+s)=U$\\
	Sei $x\in U\setminus\{x_0\}$. Wegen 1. MWS: $\exists\xi\in(x,x_0)$\\
	(*) $f(x)-f(x_0)=f^\prime(\xi)(x-x_0)$
	\begin{itemize}
		\item[1. Fall] $x\in(x_0-s,x_0)$\\
		$\Rightarrow x-x_0<0,\quad f^\prime(\xi)\leq0$ nach Voraussetzung\\
		$\overset{*}{\Rightarrow} f(x)-f(x_0)\geq0$\\
		$\Rightarrow f(x)\geq f(x_0)$
		\item[2. Fall] $x\in(x_0,x_0+s)$\\
		$\Rightarrow x-x_0>0,\quad f^\prime(\xi)\geq0$ nach Voraussetzung\\
		$\overset{*}{\Rightarrow} f(x)-f(x_0)\geq0$\\
		$\Rightarrow f(x)\geq f(x_0)$
	\end{itemize}
	Insgesamt also $f(x)\geq f(x_0)\quad\forall x\in U$\\
	Rest analog\hfill$\square$
	\subsection{Bemerkung}
	
		\begin{minipage}[c]{0.5\textwidth}
			Steigung von $f^\prime$ ist positiv in $x_0$, d.h. $f^{\prime\prime}(x_0)>0$\\
			Wenn $f^{\prime\prime}(x_0)=0$, ist keine Aussage über Vorzeichenwechsel möglich, z.b. s. Skizze, jeweils $f^{\prime\prime}(x_0)=0$
		\end{minipage}
		\begin{minipage}[c]{0.25\textwidth}
			\begin{tikzpicture}[scale=0.5]
			\draw[->]
			(-0.2,0) -- (5,0) node[right] {$x$};
			\draw[->]
			(0,-0.5) -- (0,5) node[above] {$f(x)$};
			\draw[color=blue, domain=0:4.5, ultra thick]
			plot(\x, 0.125*\x*\x*\x-0.5*\x*\x+0.5*\x-0.25*\x*\x+\x-1) node[right]{$f^\prime$};
			\draw[color=black]
			(2,0) node[below]{$x_0$};
			\end{tikzpicture}
		\end{minipage}
		\begin{minipage}[c]{0.25\textwidth}
			\begin{tikzpicture}[scale=0.5]
			\draw[->]
			(-0.2,0) -- (5,0) node[right] {$x$};
			\draw[->]
			(0,-0.5) -- (0,5) node[above] {$f(x)$};
			\draw[color=blue, domain=0.6:3.7, ultra thick]
			plot(\x,\x*\x-4*\x+4) node[right]{$f^\prime$};
			\draw[color=black]
			(2,0) node[below]{$x_0$};
			\end{tikzpicture}
		\end{minipage}
	\subsection{Satz (Hinreichende Bedingung für lokale Extrema II)}
	Sei $f\colon(a,b)\>\R$ diffbar auf $(a,b)$ und in $x_0\in(a,b)$ 2-mal diffbar\\
	$[f^\prime(x_0)=0\textrm{ und }f^{\prime\prime}(x_0)\underset{<}{>}]\Rightarrow f$ besitzt in $x_0$ lokales $\underset{\textrm{Maximum}}{\textrm{Minimum}}$
	\subsubsection*{Beweis (für Minimum)}
	Es ist $\lim\limits_{h\>0}\frac{f^\prime(x_0+h)-\overbrace{f(x_0)}^{=0}}{h}=\lim\limits_{h\>0}\frac{f^\prime(x_0+h)}{h}\overset{h\>0}{\>}f^{\prime\prime}(x_0)>0$\\
	$\Rightarrow \exists s>0\colon\frac{\overbrace{f^\prime(x_0+h)}^{*}}{h}>0\forall|h|<s,h\neq0$
	\begin{itemize}
		\item[1. Fall] $-s<h<0$\\
		$\overset{*}{\Rightarrow}f^\prime(x_0+h)<0$
		\item[2. Fall] $0<h<s$\\
		$\overset{*}{\Rightarrow}f^\prime(x_0+h)>0$
	\end{itemize}
	$\underset{\textrm{6.20}}{\Rightarrow}$ lokales Minimum in $x_0$\\
	(Anm.: Für Randpunkte: Monotonieargumente, notwendige Bedingung gilt nicht)
	\subsubsection*{Die regeln von l'Hospital}
	Problem: Grenzwerte vom Typ $\frac{0}{0},\frac{\infty}{\infty},0\*\infty$ usw.
	\subsubsection*{Beispiel}
	\begin{itemize}
		\item	$\frac{\sin x}{x}\underset{x\>0}{\>}?$\\
	$f(x)=\sin x$ und $g(x)=x$ haben in $x=0$ dieselbe Tangente ($t(x)=x$)\\
	$\Rightarrow f,g$ konvergieren mit derselben Geschwindigkeit gegen 0 für $x\>0$\\
	$\Rightarrow \frac{\sin x}{x}\underset{x\>0}{\>}1$
	\item $\frac{\sin x}{x^2}\underset{x\>0}{\>}?$\\
	$f(x)=\sin x$, $g(x)=x^2$: $f,g$ haben unterschiedliche Tangenten in $x_0=0$, $x^2$ konvergiert schneller gegen 0 als $\sin x$\\
	$\Rightarrow \frac{\sin x}{x^2}\begin{cases}
	\underset{x\>0+}{\>}\infty\\
	\underset{x\>0-}{\>}-\infty
	\end{cases}$
	\end{itemize}
	\subsubsection*{Grundidee}
	Wenn $f(a)=g(a)=0$, $f,g$ diffbar mit $g^\prime(x)\neq0$, dann
	\begin{align*}
		\frac{f(a+h)}{g(a+h)}&=\frac{f(a+h)-\overbrace{f(a)}^{=0}}{g(a+h)-\underbrace{g(a)}_{=0}}=\frac{\frac{f(a+h)-f(a)}{h}}{\frac{g(a+h)-g(a)}{h}}\\
		&\underset{h\>0}{\>}\frac{f^\prime(a)}{g^\prime(a)}
	\end{align*}
	Im Beispiel: $\lim\limits_{x\>0}\frac{\sin x}{x}=\lim\limits_{x\>0}\frac{(\sin x)^\prime}{(x)^\prime}=\lim\limits_{x\>0}\frac{\cos x}{1}=1$
	\subsection{Satz (l'Hospital)}
	$f,g\colon(a,b)\>\R$ seien diffbar auf $(a,b)$ mit $a,b\in\R\cup\{-\infty,\infty\}$ und es sei $g^\prime(x)\neq0\quad\forall x\in(a,b)$ gilt:\\
	$\lim\limits_{x\>a+}f(x)=\lim\limits_{x\>a+}g(x)=\begin{cases}
	0\textrm{ oder}\\
	\infty\textrm{ oder}\\
	-\infty
	\end{cases}$ und weiter existiert $\lim\limits_{x\>a+}\frac{f^\prime(x)}{g^\prime(x)}$, so existiert auch\\
	$\lim\limits_{x\>a+}\frac{f(x)}{g(x)}$ und es gilt\\
	$\lim\limits_{x\>a+}\frac{f(x)}{g(x)}=\lim\limits_{x\>a+}\frac{f^\prime(x)}{g^\prime(x)}$\\
	Entsprechendes gilt für $x\>b-$	
	\subsection{Beispiel}
	\begin{itemize}
		\item[a)] $\lim\limits_{x\>0}\frac{\overbrace{\sin x}^{\>0}}{\underbrace{x}_{\>0}}\overset{\textrm{6.23}}{=}\lim\limits_{x\>0}\frac{(\sin x)^\prime}{(x)^\prime}=\lim\limits_{x\>0}\frac{\cos x}{1}=\frac{1}{1}=1$\\
		(vgl. 5.14g, hier ist jetzt der Beweis)
		\item[b)] $\lim\limits_{x\>0}\frac{\overbrace{1-\cos x}^{\>0}}{\underbrace{x^2}_{\>0}}\overset{\textrm{6.23}}{=}\underbracket{\lim\limits_{x\>0}}_{\textrm{nicht vergessen!}}\frac{\overbrace{\sin x}^{\>0}}{\underbrace{2x}_{\>0}}\overset{\textrm{6.23}}{=}\lim\limits_{x\>0}\frac{\cos x}{2}=\frac{1}{2}$
		\item[c)] $\lim\limits_{x\>0}\frac{\sin(x)-x}{x^3}=$...3x 6.23 anwenden...$=-\frac{1}{6}$
		\item[d)] $\lim\limits_{x\>\infty}\frac{\overbrace{\ln x}^{\>\infty}}{\underbrace{\sqrt{x}}_{\>\infty}}\overset{\textrm{6.23}}{=}\lim\limits_{x\>\infty}\frac{\frac{1}{x}}{\frac{1}{2\sqrt{x}}}=\lim\limits_{x\>\infty}\frac{2\sqrt{x}}{x}=\lim\limits_{x\>\infty}\frac{2}{\sqrt{x}}=0$
	\end{itemize}
	Allgemein kann man zeigen: $\lim\limits_{x\>\infty}\frac{\ln x}{x^\alpha}=0,\quad\alpha>0$, d.h. $\ln x$ geht langsamer gegen $\infty$ als jede Potenz von $x$
	\begin{itemize}
		\item[e)] $\lim\limits_{x\>\infty}\frac{\overbrace{x^2}^{\>\infty}}{\underbrace{\e^x}_{\>\infty}}\overset{\textrm{6.23}}{=}\lim\limits_{x\>\infty}\frac{2x}{\e^x}\overset{\textrm{6.23}}{=}\lim\limits_{x\>\infty}\frac{2}{\e^x}=0$
	\end{itemize}
	Allgemein: $\lim\limits_{x\>\infty}\frac{x^n}{\e^{ax}}=0$, d.h. jede Exponentialfunktion ($a>0$) geht schneller gegen $\infty$ als jede Potenz von $x$
	\begin{itemize}
		\item[f)] $\lim\limits_{x\>0+}\underbrace{x\*\ln x}_{\underbrace{\textrm{Informationstheorie}}_{\textrm{Begriff der Entropie}}}=\lim\limits_{x\>0+}\frac{\overbrace{\ln x}^{\>-\infty}}{\underbrace{-\frac{1}{x}}_{\>\infty}}\overset{\textrm{6.23}}{=}\lim\limits_{x\>0+}\frac{\frac{1}{x}}{-\frac{1}{x^2}}=\lim\limits_{x\>0+}\frac{-x^2}{x}=0$
		\item[g)] Achtung! $\lim\limits_{x\>1}\frac{(x-1)^2}{x^2-1}=0$\\
		mit falsche Anwendung von 6.23 erhält man z.B.\\
		$\lim\limits_{x\>1}\frac{(x-1)^2}{x^2-1}\overset{\textrm{6.23}}{=}\lim\limits_{x\>1}\frac{2(x-1)}{2x}\underset{\textrm{Vor. l'Hospital wird nicht erfüllt}}{\neq}\lim\limits_{x\>1}\frac{2}{2}=1$\\
		Wo steckt der Fehler?
	\end{itemize}
	\newpage
	\section{Integralrechnung}
	\subsubsection*{Berechnung von Flächen, das Bestimmte Integral}
	\subsection{Motivation/Herleitung}
	$\>$s. Folien/Blatt 27.06.2016
	\subsection{Definition (Riemannintegral)}
	Sei $f\colon[a,b]\>\R$.\\
	Konvergieren $U_n$ (Untersumme) und $O_n$ (Obersumme) für jede Folge von Zerlegungen $(Z_n)_n$ mit $\lim\limits_{\infn}\mu(Z_n)=0$ (Feinheit der Zerlegung) für $\infn$ gegen denselben Wert $A$, so heißt $f$ \underline{(Riemann-)integrierbar} über $[a,b]$. Man nennt diesen Grenzwert $A$ auch \underline{Integral von $f$ über $[a,b]$} und bezeichnet ihn mit
	\begin{align*}
		\int_{a}^{b}f(x)\textrm{d}x
	\end{align*}
	(Name der Integrationsvariablen spielt keine Rolle, also z.B. auch $\int_{a}^{b}f(t)\textrm{d}t$ oder $\int_{a}^{b}f(\xi)\textrm{d}\xi$ etc.)\\
	Wir legen fest:
	\begin{align*}
		\int_{a}^{a}f(x)\textrm{d}x&=0\\
		\int_{b}^{a}f(x)\textrm{d}x&=-\int_{a}^{b}f(x)\textrm{d}x
	\end{align*}
	\subsection{Beispiel}
	\begin{itemize}
		\item[a)] $f(x)=c\qquad\forall x \in[a,b]$, dann\\
		$y_i=c=Y_i\quad\forall i=1...n$\\
		$U_n=\sum_{i=1}^{n}c\*(x_i-x_{i-1})=c\*\sum_{i=1}^{n}(x_i-x_{i-1})$\\
		$\quad\overset{\textrm{Teleskopsumme}}{=}c\*(x_n-x_0)=c\*(b-a)$\\
		$O_n=\sum_{i=1}^{n}c\*(x_i-x_{i-1})=...=c\*(b-a)$\\
		Egal wie $x_i$ gewählt werden, $\lim\limits_{\infn}U_n=\lim\limits_{\infn}O_n=c\*(b-a)$
		\item[b)] Es gibt auch Funktionen, die nicht integrierbar sind:\\
		$f\colon[a,b]\>\R,\quad f(x)=\begin{cases}1&x\in\mathds{Q}\\
		0&x\in\R\setminus\mathds{Q},\quad x\notin\R
		\end{cases}$\\
		Egal, wie Zerlegung $Z_n$ gewählt wird: es ist immer $\inf f(x)=0,$\\
		$ \sup f(x)=1$ in den Teilintervallen\\
		$\Rightarrow U_n=\sum_{i=1}^{n}(x_i-x_{i-1})\*0=0$\\
		$\quad O_n=\sum_{i=1}^{n}(x_i-x_{i-1})\*1\overset{\textrm{Teleksops.}}{=}x_n-x_0=1-0=1$\\
		$U_n$ und $O_n$ konvergieren \underline{nie} gegen denselben Wert!		
	\end{itemize}
	\subsection{Bemerkung}
	$\int_{a}^{b}f(x)\textrm{d}x$ existiert beispielweise, wenn $f\colon[a,b]\>\R$
	\begin{itemize}
		\item stetig ist oder
		\item monoton ist
		\item beschränkt und 'fast überall' stetig ist (Graph von $f$ besitzt nur endlich viele Sprungstellen)
	\end{itemize}
	\subsection{Bemerkung}
	Wir haben in 7.1 $f$ positiv gewählt.\\
	In diesem Fall ist $\int_{a}^{b}f(x)\textrm{d}x$ der Flächeninhalt der von $f$ und der $x$-Achse begrenzten Fläche.\\
	Andere Fälle:
	\begin{itemize}
		\item $A=\underbrace{\int_{a}^{c}f(x)\textrm{d}x}_{f\textrm{ positiv}}+\underbrace{\int_{c}^{b}-f(x)\textrm{d}x}_{\textrm{hier ist }f\textrm{ negativ, also }-f\textrm{ positiv}}$\\
		$\qquad=\int_{a}^{b}|f(x)|\textrm{d}x$
		\item $A=\int_{a}^{b}f(x)\textrm{d}x-\int_{a}^{b}g(x)\textrm{d}x$, falls $f\geq g$ auf $[a,b]$
	\end{itemize}
	Die folgenden Regeln sind leicht nachzuprüfen:\subsection{Satz (Rechenregeln für Integrale)}
	Seien $f,g$ auf $[a,b]$ integrierbar. Dann ist
	\begin{itemize}
		\item[a)] $\int_{a}^{b}\lambda\*f(x)\textrm{d}x=\lambda\*\int_{a}^{b}f(x)\textrm{d}x\qquad\forall\lambda\in\R$
		\item[b)] $\int_{a}^{b}[f(x)+g(x)]\textrm{d}x=\int_{a}^{b}f(x)\textrm{d}x+\int_{a}^{b}g(x)\textrm{d}x$
	\end{itemize}
	a und b zusammen heißen Linearität.
	\begin{itemize}
		\item[c)] $\int_{a}^{b}f(x)\textrm{d}x=\int_{a}^{c}f(x)\textrm{d}x+\int_{c}^{b}f(x)\textrm{d}x\quad\forall c\in[a,b]$\\
		(Zwischenstelle)
		\item[d)] $f(x)\leq g(x)\quad\forall x$, dann auch\\
		$\int_{a}^{b}f(x)\textrm{d}x\leq\int_{a}^{b}g(x)\textrm{d}x$\\
		(Monotonie)
		\item[e)] falls $m\leq f(x)\leq M\quad\forall x\in[a,b]$, dann ist \\
		$m(b-a)\leq\int_{a}^{b}f(x)\textrm{d}x\leq M(b-a)$\\
	\end{itemize}
	\subsection{Satz (Mittelwertsatz der Integralrechnung)}
	$f\colon[a,b]\>\R$ stetig\\
	Dann existiert $\xi\in[a,b]$ mit $\int_{a}^{b}f(x)\textrm{d}x=f(\xi)(b-a)$\\
	\subsubsection*{Beweis}
	$f$ stetig auf $[a,b]\overset{\textrm{Minimaxth.}}{\underset{\textrm{5.20}}{\Rightarrow}}f$ nimmt auf $[a,b]$ Minimum $m$ und Maximum $M$ an, d.h.\\
	\begin{align*}
		&m\leq f(x)\leq M&&\\
		&\overset{\textrm{7.6 d}}{\Rightarrow}m(b-a)&\leq\int_{a}^{b}f(x)\textrm{d}x&\leq M(b-a)\\
		&\Rightarrow m&\leq\underbracket{\frac{1}{b-a}\*\int_{a}^{b}f(x)\textrm{d}x}_{y\textrm{ aus ZWS}}&\leq M\\
		&\overset{\textrm{ZWS}}{\underset{\textrm{5.16}}{\Rightarrow}}&\exists\xi\in[a,b]\textrm{ mit }f(\xi)=\frac{1}{b-a}\*\int_{a}^{b}f(x)\textrm{d}x&
	\end{align*}
	\hfill$\square$
	\subsubsection*{Stammfunktionen und der HDI (Haupdsatz der Differential- und Integralrechnung)}
	Wir können Flächeninhalte bisher nur über Riemannsummen/-integrale ($U_n,O_n,\lim$ usw.) berechnen, das ist umständlich.
	\subsection{Definition (Stammfunktion)}
	%Sei $f\colon[a,b]\>\R$. Dann nennt man jede auf $(a,b)$ diffbare Funktion $F\colon[a,b]\>\R$, für die $F^\prime(x)=f(x)\quad\forall x\in(a,b)$ gilt, eine \underline{Stammfunktion} von $f$ über $[a,b]$.[Korrigiert]
	\begin{itemize}
		\item[a)] Sei $g\colon[a,b\>\R]$ eine stetige Funktion. Der Grenzwert $\lim\limits_{h\>0^+}\frac{g(a+h)-g(a)}{h}\eqqcolon g^\prime_+(a)$ heißt (falls existent) \underline{rechtsseitige Ableitung} von $g$ in $a$.\\
		Analog: $\lim\limits_{h\>0^-}\frac{g(b+h)-g(b)}{h}\eqqcolon g^\prime_-(b)$ ist \underline{linksseitige Ableitung} von $g$ in $b$.
		\item[b)] Stammfunktion\\
		 Sei $f\colon[a,b]\>\R$.\\Dann nennt man eine diffbare Funktion $F\colon[a,b]\>\R$ \underline{Stammfunktion} von $f$ über $[a,b]$, wenn gilt:
		\begin{align*}
			F^\prime(x)&=f(x)\qquad\forall x\in(a,b)\\
			F^\prime_+(a)&=f(a)\\
			F^\prime_-(b)&=f(b)
		\end{align*}
		[Korrektur: Unterschied: $F^\prime$ ist auch für $a,b$ definiert!]
	\end{itemize}
	\subsection{Beispiel}
	\begin{itemize}
		\item[a)] $f(x)=3\qquad F(x)=3x$ oder $3x+1;\quad3x+2;$ etc.
		\item[b)] $f(x)=x^2\qquad F(x)=\frac{1}{3}x^3+c$ für $c\in\R$\\
		allgemein $f(x)=x^n\qquad F(x)=\frac{1}{n+1}x^{n+1}+c$
		\item[c)] $f(x)=\sin x\qquad F(x)=-\cos x+c$ usw.
	\end{itemize}
	\subsection{Bemerkung}
	Stammfunktionen sind nicht eindeutig bestimmt!\\
	Ist $F$ eine Stammfunktion von $f$, dann auch $F+c$ für ein beliebiges $c\in\R$ (denn: $F^\prime=f$ und $(F+c)^\prime=F^\prime+0=f$).\\
	Sind $F$ und $G$ Stammfunktionen von $f$, so können sie sich nur durch eine Konstante unterscheiden:
	\begin{align*}
		(F-G)^\prime&=F^\prime-G^\prime=f-f=0\\
		&\underset{\textrm{6.19 iii}}{\Rightarrow} F-G\textrm{ ist konstant}
	\end{align*}
	Der Zusammenhang zwischen dem in 7.1/7.2 behandelten Integral von $f$ und einer Stammfunktion von $f$ wird im folgenden Satz hergestellt.
	\subsection{Satz (HDI, Hauptsatz der Differential- und Integralrechnung)}
	$f\colon[a,b]\>\R$ stetig.
	\begin{itemize}
		\item[a)] Dann ist $F(x)=\int_{a}^{x}f(t)\textrm{d}t\qquad x\in[a,b]$ eine Stammfunktion von $f$ (also diffbar, $F^\prime(x)=f(x)$) und es gilt:
		\item[b)] $\underbrace{\int_{a}^{b}f(x)\textrm{d}x}_{\textrm{bish. Flächeninh., nur mit } U_n, O_n \textrm{berechenbar}}=F(b)-F(a)\overset{\textrm{Schreibweise}}{=}[F(x)]^{x=b}_{x=a}=[F(x)]^b_a$
	\end{itemize}
	\subsubsection*{Beweis:}
	\begin{itemize}
		\item[a)] (zu Zeigen: das so definierte $F$ ist Stammfunktion, also $F^\prime=f$)
		\begin{align*}
			\frac{F(x+h)-F(x)}{h}&=\frac{1}{h}\*(\int_{a}^{x+h}f(t)\textrm{d}t-\int_{a}^{x}f(t)\textrm{d}t)\\
			&=\frac{1}{h}\*(\int_{x}^{a}f(t)\textrm{d}t+\int_{a}^{x+h}f(t)\textrm{d}t)\\
			&\overset{\textrm{7.16 c}}{=}\frac{1}{h}\*\int_{x}^{x+h}f(t)\textrm{d}t
		\end{align*}
		Nach dem MWS der Integralrechnung (7.7) existiert ein $\xi\in[x,x+h]$ mit $\frac{F(x+h)-F(x)}{h}=f(\xi)$. Für $h\>0$ erhalten wir $(\xi\in[x,x+h])$ also $\xi\>0$ für $h\>0$:\\
		$\underbrace{\frac{F(x+h)-F(x)}{h}}_{\>F^\prime(x)}=\underbrace{f(\xi)}_{\>f(x)}$ ($f$ ist stetig)\\
		Also ist $F$ Stammfunktion von $f$
		\item[b)] folgt aus a)
	\end{itemize}
	\hfill$\square$
	\subsection{Beispiel}
	Der HDI hilft bei der Berechnung von Integralen:
	\begin{itemize}
		\item[a)] $\int_{0}^{3}\underbrace{2}_{f(x)}\textrm{d}x=[\underbrace{2x}_{F(x)}]^3_0=\underbrace{2\*3}_{F(3)}-\underbrace{2\*0}_{F(0)}=6$
		\item[b)] $\int_{1}^{2}x^2+x+1\textrm{d}x=[\frac{1}{3}x^3+\frac{1}{2}x^2+x]^2_1=(\frac{1}{3}\*2^3+\frac{1}{2}\*2^2+2)-(\frac{1}{3}\*1^3+\frac{1}{2}\*1^2+1)=\frac{29}{6}$
		\item[c)] $\int_{0}^{\frac{\pi}{2}}\sin x\textrm{d}x=[-\cos x]^{\frac{\pi}{2}}_0=0-(-1)=1$
	\end{itemize}
	\subsection{Bemerkung}
	\begin{itemize}
		\item[a)] Der HDI besagt:\\
		Die Integration ist die inverse Operation zur Differentiation.
		\item[b)] Schreibweise:\\
		Um auszudrücken, dass $F(x)$ eine Stammfunktion von $f(x)$ ist ($F^\prime(x)=f(x)$), schreibt man auch $F(x)\underbrace{=\int}_{\textrm{'ist Stammfunktion von'}}f(x)\textrm{d}x$\\
		Ohne Grenzen heißt das auch 'unbestimmtes Integral'.\\
		Beispiel: $\int5x^2\textrm{d}x=\frac{5}{3}x^3(+c)$; $\int\sin x\textrm{d}x=-\cos x$ usw.\\
		Achtung:
		\begin{align*}
			\sin x&=\int\cos x\textrm{d}x&=\sin x+1\\
			'\Rightarrow'0&&=1
		\end{align*}
	Gleichheitszeichen wird hier (in der Schreibweise) nicht im üblichen Sinne benutzt! Gleichheit gilt nur für ABleitungen
	\item[c)] Es spielt keine Rolle, welche Stammfunktion man im HDI b) bei der Berechnung von $\int_{a}^{b}f(x)\textrm{d}x$ benutzt.\\
	$F(b)+c-(F(a)+c)=F(b)-F(a)$
	\end{itemize}
	\subsubsection*{Techniken zur Bestimmung von Stammfunktionen}
	\subsection{Stammfunktionen elementarer Funktionen}
	(Beweis durch Differenzieren)
	\begin{itemize}
		\item $\int x^\alpha\textrm{d}x=\frac{1}{\alpha+1}x^{\alpha+1}\qquad a\neq -1$
		\item $\int\frac{1}{x}\textrm{d}x=\ln|x|\qquad x\neq 0$
		\item $\int\ln x\textrm{d}x=x\*\ln x-x\qquad\textrm{vgl. 7.15 c)}$
		\item $\int\e^x\textrm{d}x=\e^x$
		\item $\int a^x\textrm{d}x=\frac{1}{\ln a}a^x\qquad a>0,a\neq 1$
		\item $\int\sin x\textrm{d}x=-\cos x$
		\item $\int\cos x\textrm{d}x=\sin x$
		\item $\int\frac{1}{\cos^2x}\textrm{d}x=\tan x$
	\end{itemize}
	\subsection{Partielle Integration}
	Es gilt:
	\begin{align*}
		\int \underbracket{f^\prime(x)\*g(x)}_{2}\textrm{d}x=\underbracket{f(x)\*g(x)}_{1}-\underbracket{\int f(x)\*g^\prime(x)\textrm{d}x}_{3}
	\end{align*}
	Diese Methode ist abgeleitet von der Produktregel der Differentialrechnung:
	\begin{align*}
		(\underbracket{f(x)\*g(x)}_{1})^\prime=\underbracket{f^\prime(x)\*g(x)}_{2}+\underbracket{f(x)\*g^\prime(x)}_{3}
	\end{align*}
	Bem.: Polynome von kleinem Grad sind oft eine gute Wahl für $g(x)$, s. Beispiele
	\subsubsection*{Beispiel}
	\begin{itemize}
		\item[a)] \begin{align*}
		\int\underbrace{\sin x}_{f^\prime(x)}\*\underbrace{x}_{g(x)}\textrm{d}x&=(-\cos x)\*x-\int\underbrace{(-\cos x)}_{f^\prime(x)}\*\underbrace{1}_{g(x)}\textrm{d}x\\
		&=(-\cos x)\*x+\sin x
		\end{align*}
		($f^\prime(x)=-\cos x, g^\prime(x)=1\>\textrm{ gut}$)
		\item[b)] \begin{align*}
			\int\underbrace{x^2}_{g(x)}\*\underbrace{\e^x}_{f^\prime(x)}\textrm{d}x&=\e^x\*x^2-\int\e^x\*2x\textrm{d}x\\
			&=\e^x\*x^2-2\*\int\underbrace{\e^x}_{f^\prime(x)}\*\underbrace{x}_{g(x)}\textrm{d}x\\
			&=\e^x\*x^2-2\*(\e^x\*x-\int\e^x\*1\textrm{d}x)\\
			&=\e^x\*x^2-2\e^xx+2\e^x\\
			&=\e^x\*(x^2-2x+2)
		\end{align*}
		\item[c)] \begin{align*}
			\int\ln x\textrm{d}x&=\int\underbrace{1}_{f^\prime(x)}\*\underbrace{\ln x}_{g(x)}\textrm{d}x\\
			&=x\*\ln x-\int x\*\frac{1}{x}\textrm{d}x\\
			&=x\*\ln x-\int 1\textrm{d}x\\
			&=x\*\ln x-x
		\end{align*}
		\item[d)] 'Phoenix aus der Asche'\\
		Integral, das eine Funktion enthält, die beim partiellen Integrieren in absehbarer Zeit wieder erscheint (aus ihrer Asche wieder auftaucht).
		\begin{align*}
			\int\underbrace{\e^x}_{f^\prime(x)}\*\underbrace{\cos x}_{g(x)}\textrm{d}x&=\e^x\cos x+\int\underbrace{\e^x}_{f^\prime(x)}\*\underbrace{\sin x}_{g(x)}\textrm{d}x\\
			&=\e^x\cos x+\e^x\sin x-\int\e^x\*\cos x\textrm{d}x\\
			\Rightarrow2\*\int\e^x\cos x\textrm{d}x&=\e^x(\cos x+\sin x)\\
			\Rightarrow \int\e^x\cos x\textrm{d}x&=\frac{\e^x(\cos x+\sin x)}{2}
		\end{align*}
	\end{itemize}
	\subsection{Substitution}
	Es gilt:
	\begin{align*}
		\int\underbracket{f^\prime(g(x))\*g^\prime(x)\textrm{d}x}_{2}=\underbracket{f(g(x))}_{1}
	\end{align*}
	Diese Methode ist aus der Kettenregel der Differentialrechnung abgeleitet:
	\begin{align*}
		(\underbracket{f(g(x))}_{1})^\prime=\underbracket{f^\prime(g(x))\*g^\prime(x)}_{2}
	\end{align*}
	Man hat mit ihr gute Aussichten auf Erfolg, wenn im Integral neben einer Funktion auch deren Ableitung vorkommt.
	\subsubsection*{Beispiel}
	\begin{itemize}
		\item[a)] $\int\underbrace{\e^{\overbrace{x^2}^{g(x)}}}_{f(x)}\*\underbrace{2x}_{g^\prime(x)}\textrm{d}x=\e^x$, da:
		innere Funktion: $g(x)=x^2$\\
		$\qquad\quad g^\prime(x)=2x$\\
		äußere Funktion: $f^\prime(u)=\e^u$\\
		$\qquad\quad f^\prime(g(x))=\e^{x^2}$\\
		$\Rightarrow\quad f(u)=\e^u$\\
		$\qquad\quad f(g(x))=\e^{x^2}$\\
	Falls nicht alles so gut passt wie hier: hinbiegen.\\
	 Oft einfacher: formales Rezept
		\begin{align*}
			\int f^\prime(g(x))\*g^\prime(x)\textrm{d}x
		\end{align*}
		Substitution: $u=g(x)$; $\frac{\textrm{d}u}{\textrm{d}x}=g^\prime(x)$ ('leite die Funktion $u$ ab nach der Variablen $x$')\\
		$\Leftrightarrow g^\prime(x)\textrm{d}x=\textrm{d}u$
		\begin{align*}
			\Rightarrow \int f^\prime(g(x))\*g^\prime(x)\textrm{d}x=\int f^\prime(u)\textrm{d}u=f(u)
		\end{align*}
		Rücksubstitution: $u=g(x)$\\
		$\Rightarrow f(u)=f(g(x))$
		\item[b)] $\int\underbracket{x}_{}\*\sqrt{2x^2+1}\underbracket{\textrm{d}x}_{}$\\
		Substitution: $u=2x^2+1$; $\frac{\textrm{d}u}{\textrm{d}x}=4x\Leftrightarrow \textrm{d}u=4x\textrm{d}x$\\
		$\frac{\textrm{d}u}{4}=\underbracket{x\textrm{d}x}_{}$
		\begin{align*}
			\int x\sqrt{2x^2+1}\textrm{d}x&=\int\sqrt{u}\frac{\textrm{d}u}{4}\\
			&=\frac{1}{4}\int\underbrace{\sqrt{u}}_{u^{\frac{1}{2}}}\textrm{d}u\\
			&=\frac{1}{4}\*\frac{2}{3}u^{\frac{3}{2}}
		\end{align*}
		Rücksubstitution: $u=2x^2+1$
		\begin{align*}
			&=\frac{1}{6}\*(2x^2+1)^{\frac{3}{2}}
		\end{align*}
		\item[c)] $\int\underbracket{g^\prime(x)}_{}\*g(x)\underbracket{\textrm{d}x}_{}$\\
		Substitution: $u=g(x)$; $\frac{\textrm{d}u}{\textrm{d}x}=g^\prime(x)\Leftrightarrow \textrm{d}u=\underbracket{g^\prime(x)\textrm{d}x}_{}$\\
		$\int g^\prime(x)\*g(x)\textrm{d}x=\int u\textrm{d}u=\frac{1}{2}u^2$\\
		Rücksubstitution: $u=g(x)$\\
		$\int g^\prime(x)g(x)\textrm{d}x=\frac{1}{2}(g(x))^2$
		\item[d)] $\int\frac{g^\prime(x)}{g(x)}\textrm{d}x$: Übung
	\end{itemize}
	\subsection{Bemerkung (Bestimmtes Integral berechnen)}
	z.B. $\int_{0}^{1}x\sqrt{2x^2+1}\textrm{d}x\qquad f(x)=x\sqrt{2x^2+1}$\\
	bestimme zuerst Stammfunktion:\\
	$\int x\sqrt{2x^2+1}\textrm{d}x\overset{\textrm{7.16 b)}}{=}\frac{1}{6}(2x^2+1)^{\frac{3}{2}}=F(x)$\\
	Benutze den HDI:\\
	$\int_{0}^{1}x\sqrt{2x^2+1}\textrm{d}x=F(1)-F(0)=\frac{1}{6}(3)^{\frac{3}{2}}-\frac{1}{6}(1)^{\frac{3}{2}}=\frac{\sqrt{3}}{2}-\frac{1}{6}$
	\subsubsection*{Uneigentliche Integrale}
	Bei der bestimmten Integration hatten wir vorausgesetzt, dass
	\begin{itemize}
		\item die zu integrierende Funktion beschränkt ist
		\item die Intervall-(Integrations-)grenzen endlich sind.
	\end{itemize}
	Was passiert, wenn diese Voraussetzungen nicht erfüllt sind?\\
	Z.B.: $\int_{0}^{1}\frac{1}{\sqrt{x}}\textrm{d}x$.\\
	Für jedes beliebig kleine $\epsilon>0$ existiert $\int_{\epsilon}^{1}\frac{1}{\sqrt{x}}\textrm{d}x$, also kann man den Grenzwert $\lim\limits_{\epsilon\>0+}\int_{\epsilon}^{1}\frac{1}{\sqrt{x}}\textrm{d}x$ untersuchen:
	\begin{align*}
		\lim\limits_{\epsilon\>0+}\int_{\epsilon}^{1}\frac{1}{\sqrt{x}}\textrm{d}x&=\lim\limits_{\epsilon\>0+}\int_{\epsilon}^{1}x^{-\frac{1}{2}}\textrm{d}x\\
		&=\lim\limits_{\epsilon\>0+}[2x^\frac{1}{2}]_\epsilon^1\\
		&=\lim\limits_{\epsilon\>0+}(2-2\sqrt{\epsilon})\\
		&=2
	\end{align*}
	\subsection{Definition (Uneigentliches Integral)}
	Sei $f\colon(a,b]\>\R,\qquad a\in\R\cup\{-\infty\}$.\\
	$f$ heißt über $(a,b]$ \underline{uneigentlich integrierbar}, falls der Grenzwert $\lim\limits_{T\>a+}\int_{T}^{b}f(x)\textrm{d}x$ existiert (und alle Integrale $\int_{T}^{b}f(x)\textrm{d}x$).\\
	Entsprechend definiert man uneigentliche Integrale über $[a,b)$ mit $b\in\R\cup\{\infty\}$ als $\lim\limits_{T\>b-}\int_{a}^{T}f(x)\textrm{d}x$ und über $(a,b)$ als $\lim\limits_{T\>a+}\int_{T}^{c}f(x)\textrm{d}x+\lim\limits_{T\>b-}\int_{c}^{T}f(x)\textrm{d}x$ für ein festes $c\in(a,b)$.
	\subsection{Beispiel}
	\begin{itemize}
		\item[a)] $\int_{0}^{1}\frac{1}{\sqrt{x}}\textrm{d}x=2$ (s.o.)
		\item[b)] \begin{align*}
			\int_{1}^{\infty}\frac{1}{x^2}\textrm{d}x&=\lim\limits_{T\>\infty}\int_{1}^{T}\frac{1}{x^2}\textrm{d}x\\
			&=\lim\limits_{T\>\infty}[-\frac{1}{x}]_1^T\\
			&=\lim\limits_{T\>\infty}(-\frac{1}{T}-(-\frac{1}{1}))\\
			&=0+1\\
			&=1
		\end{align*}
		\item[c)] \begin{align*}
			\int_{1}^{\infty}\frac{1}{x}\textrm{d}x&=\lim\limits_{T\>\infty}\int_{1}^{T}\frac{1}{x}\textrm{d}x\\
			&=\lim\limits_{T\>\infty}[\ln|x|]_1^T\\
			&=\lim\limits_{T\>\infty}(\ln T-\underbrace{\ln 1}_{=0})
		\end{align*}
		existiert nicht ($\ln T\>\infty$ für $T\>\infty$), das Integral divergiert bestimmt gegen $\infty$, der Flächeninhalt ist unendlich groß.\\
		Ebenso: $\int_{0}^{1}\frac{1}{x}\textrm{d}x$ divergiert.
		\item[d)] allgemein gilt:\\
		$\int_{0}^{1}\frac{1}{x^\alpha}\textrm{d}x\begin{cases}\textrm{divergiert für }&\alpha\geq1\\
		=\frac{1}{1-\alpha}\textrm{ für }&\alpha<1
		\end{cases}$\\
		$\int_{1}^{\infty}\frac{1}{x^\alpha}\textrm{d}x\begin{cases}=\frac{1}{\alpha-1}\textrm{ für }&\alpha>1\\
		\textrm{divergiert für }&\alpha\leq1
		\end{cases}$
	\end{itemize}
	\newpage
	\section{Vektorräume}
	\subsection{Definition (Reelle Vektorräume)}
	Ein \underline{$\R$-Vektorraum V} ist eine nichtleere Menge, deren Elemente \underline{Vektoren} genannt werden (Bezeichnung mittels kleiner lateinischer Buchstaben, $v,w,x,y,...$), auf der eine Addition $+$ definiert ist, $+\colon V\times V\>V$; und eine Multiplikation mit reellen Zahlen ('Skalare') (Bezeichnung mittels kleiner griechischer Buchstaben $\alpha, \beta, \gamma, \lambda,\mu,...$), $\*\colon\R\times V\>V$, so dass gilt:
	\begin{itemize}
		\item[(1.1)] $u+v+w=u+(v+w)\qquad\forall u,v,w\in V$
		\item[(1.2)] Es existiert ein Vektor $\O\in V$ ('\underline{Nullvektor}') mit $v+\O=\O+v=v\qquad\forall v\in V$
		\item[(1.3)] Zu jedem $v\in V$ existiert ein Vektor $-v\in V$ mit $v+(-v)=\O$
		\item[(1.4)] $u+v=v+u\qquad\forall u,v\in V$
	\end{itemize}
	(Diese Eigenschaften (1.1) bis (1.4) kann man zusammenfassen als '$(V,+)$ ist eine kommutative Gruppe').
	\begin{itemize}
		\item[(2.1)] $\overset{\textrm{Addition in }\R}{(\lambda+\mu)}\*v=\lambda\*v\overset{\textrm{Addition in }V}{+}\mu\*v\qquad\forall\lambda,\mu\in\R,v\in V$
		\item[(2.2)] $\lambda(v+w)=\lambda v+\lambda w\qquad\forall\lambda\in\R,v,w\in V$
		\item[(2.3)] $\overset{\textrm{Multiplikation in }\R}{(\lambda\*\mu)}\*v=\lambda\*\overset{\textrm{Multiplikation mit Skalar}}{(\mu\*v)}\qquad\forall\lambda,\mu\in\R,v\in V$
		\item[(2.4)] $1\*v=v\qquad\forall v\in V$
	\end{itemize}
	\subsection{Beispiel}
	\begin{itemize}
		\item[a)] trivialer Vektorraum Nullraum: $V=\{\O\}$\\
		Es gilt $\O+\O\coloneqq\O,\quad\lambda\*\O\coloneqq\O\qquad\forall\lambda\in\R$
		\item[b)] $V=\R^n$, Raum aller 'Spaltenvektoren' der Länge $n$ über $\R$, Elemente haben die Form $\vec{x_1\\...\\x_n}$ mit $x_1,...,x_n\in\R$.\\
		$\O=\vec{0\\...\\0},\quad\vec{x_1\\...\\x_n}+\vec{y_1\\...\\y_n}=\vec{x_1+y_1\\...\\x_n+y_n},\quad\lambda\*\vec{x_1\\...\\x_n}=\vec{\lambda\*x_1\\...\\\lambda\*x_n}$
		\item[c)] $\R$ ist ein $\R$-Vektorraum.\\
		Vektoren: reelle Zahlen.\\
		Skalare: reelle Zahlen.\\
		$\O=0$
		\item[d)] Funktionenraum:\\
		$M\neq\emptyset$ Menge. $V=\mathcal{F}(M,\R)\coloneqq\{f\colon M\>\R\}$\\
		Menge der auf $M$ definierten reellen Funktionen.\\
		Für $f,g\in V,\quad\lambda\in\R$ sei
		\begin{itemize}
			\item $f+g\colon M\>\R,\quad(f+g)(x)=f(x)+g(x)\quad\forall x\in M$
			\item $\lambda\*f\colon M\>\R,\quad(\lambda\*f)(x)=\lambda\*f(x)\quad\forall x\in M$
		\end{itemize}
		Dann ist $V$ mit $\R,+,\*$ ein Vektorraum. Nullvektor ist $f=0\colon M\>\R,\quad f(x)=0\qquad\forall x\in M$.\\
		(kurz: $f\equiv0$, identisch Null)
	\end{itemize}
	\subsection{Lemma}
	Sei $V$ ein $\R$-Vektorraum, $v\in V,\quad\lambda\in\R$
	\begin{itemize}
		\item[a)] $0\*v=\O$
		\item[b)] $\lambda\*\O=\O$
		\item[c)] Zu jedem $v\in V$ ist der Vektor $-v$ aus (1.3) in 8.1 eindeutig bestimmt.
		\item[d)] $(-1)\*v=-v$
	\end{itemize}
	\subsubsection*{Beweis}
	\begin{itemize}
		\item[a)] \begin{align*}
			\O\overset{\textrm{(1.3)}}{=}\overbrace{\underbracket{0\*v}_{}}^{x}+\overbrace{(-0\*v)}^{-x}&=\underbracket{(0+0)v}_{}+(-0\*v)\\
			&\overset{\textrm{(2.1)}}{=}(0\*v+0\*v)+(-0\*v)\\
			&\overset{\textrm{(1.1)}}{=}0\*v+(0*v+(-0\*v))\\
			&\overset{\textrm{(1.3)}}{=}0\*v+\O\\
			&\overset{\textrm{(1.2)}}{=}0\*v
		\end{align*}
		\item[b)] Wie a), starte mit $\O=\lambda\*\O+(-\lambda\*\O)$, erhalte $\O=\lambda\*\O$
		\item[d)] \begin{align*}
			\underbracket{v+(-1\*v)}_{}&=1\*v+(-1\*v)\\
			&\overset{\textrm{(2.1)}}{=}(1+(-1))v\\
			&=0\*v\\
			&\overset{\textrm{a)}}{=}\O\\
			&\overset{\textrm{(1.3)}}{=}v+(-v)
		\end{align*}
		Addiere auf beiden Seiten $-v$:
		\begin{align*}
			\underbracket{v+(-1)v}_{}+(-v)&=v+(-v)+(-v)\\
			&\Rightarrow-1\*v=-v		
		\end{align*}
		\item[c)] Angenommen, zu $v\in V$ gibt es $-v$ und $-v^\prime$ mit $v+(-v)=\O$ und $v+(-v^\prime)=\O$. Dann ist $v+(-v)=v+(-v^\prime)\overset{+(-v)\textrm{auf beiden Seiten}}{\Rightarrow}-v=-v^\prime$
	\end{itemize}
	\hfill$\square$
	\subsection{Definition}
	Sei $V$ ein $\R$-Vektorraum.\\
	Eine Teilmenge $U\subseteq V,\quad U\neq\emptyset$ heißt \underline{Unter(vektor)raum von $V$}, falls $U$ bezüglich der Addition auf $V$ und der Multiplikation mit Skalaren selbst ein Vektorraum ist.
	\subsection{Beispiel}
	\begin{itemize}
		\item[a)] $V=\R^2,\quad U=\{\vec{0\\0}\}$ ist Unterraum von $V$
		\item[b)] $V=\R^2,\quad U=\{\vec{1\\2}\}$ ist kein Unterraum von $V$, z.B. (1.2) ist verletzt, Addition funktioniert auch nicht: $\vec{1\\2}+\vec{1\\2}=\vec{2\\4}\notin U$
		\item[c)] $V=\R^2,\quad U=\{\vec{\lambda\\0}|\lambda\in\R\}$ ist ein Unterraum von $V$ (prüfe alle Eigenschaften von Definition 8.1) $\>$ umständlich, einfacher geht es mit 8.6
	\end{itemize}
	\subsection{Satz (Unterraumkriterium)}
	Sei $V$ ein $\R$-Vektorraum, sei $\emptyset\neq U\subseteq V$.\\
	Dann ist $U$ Unterraum von $V$ genau dann, wenn gilt ($\Leftrightarrow$):
	\begin{itemize}
		\item[(1)] $v\in U,\quad\lambda\in\R\Rightarrow\lambda\*v\in U$
		\item[(2)] $v,w\in U\Rightarrow v+w\in U$
	\end{itemize}
	(oder äquivalent: $\forall v,w\in U, \forall\lambda,\mu\in\R$ ist $\lambda\*v+\mu\*w\in U$)\\
	Man sagt: $U$ ist abgeschlossen bezüglich der Vektoraddition und der Multiplikation mit Skalaren.
	\subsubsection*{Beweis}
	\begin{itemize}
		\item[$\Rightarrow$] ist klar, da $U$ laut Definition 8.4 selbst Vektorraum
		\item[$\Leftarrow$] rechne die Vektorraumaxiome nach (Definition 8.1, also z.B. $\O\in U$,...)
	\end{itemize}
	\hfill$\square$
	\subsection{Beispiel}
	\begin{itemize}
		\item[a)] $\quad$\\
		\begin{minipage}[c]{0.5\textwidth}
			$V$ ist ein $\R$-Vektorraum, $\O\neq v\in V$.\\
			Dann ist $G=\{\lambda\*v|\lambda\in\R\}$ ein Unterraum.\\
			$V=\R^2,\R^3$: $G$ ist Gerade durch Nullpunkt (geometrisch), z.B.\\ $v=\vec{2\\1},w=\vec{1\\2}$\\
			Aber: $G^\prime=\{w+\lambda\*v|\lambda\in\R,\quad w\in V\}$ ist kein Unterraum für $w\neq \mu\*v,\quad \mu\in\R$. Warum? Z.B. $\O\notin G^\prime$
		\end{minipage}
		\begin{minipage}[c]{0.25\textwidth}$\quad$
			\begin{tikzpicture}[scale=0.5]
			\draw[->]
			(-0.2,0) -- (5,0) node[right] {$x$};
			\draw[->]
			(0,-0.5) -- (0,5) node[above] {$f(x)$};
			\draw[color=blue, thick, ->]
			(0,0)--(2,1)node[below]{$v$};
			\draw[color=blue, thick, ->]
			(0,0)--(1,2)node[above]{$w$};
			\draw[color=blue, ->]
			(0,0)--(4,2)node[below]{$\quad2\*v$, $G$};
			\draw[color=blue, ->]
			(0,0)--(2,4)node[above]{$2\*w$};
			\draw[color=blue, ->]
			(0,1.5)--(3,3)node[right]{$G^\prime$};
			\end{tikzpicture}
		\end{minipage}
	\item[b)] $V=\R^3,\qquad U_1=\{\vec{x_1\\x_2\\x_3}\in\R^3|x_1+x_2-x_3=0\}$ ist Unterraum. Wir zeigen (1), (2) aus 8.6:
	\begin{itemize}
		\item $U_1\neq\emptyset$, z.B. $\O=\vec{0\\0\\0}\in U_1$, denn $\overset{x_1}{0}+\overset{x_2}{0}-\overset{x_3}{0}=0$
		\item[(1)] Sei $\lambda\in\R,\quad v=\vec{v_1\\v_2\\v_3}\in U_1$, d.h. $v_1+v_2-v_3=0$\\
		Prüfe: Ist $\lambda\*v\in U_1$? $\lambda\*v=\vec{\lambda\*v_1\\\lambda\*v_2\\\lambda\*v_3}$
		\begin{align*}
			\lambda\*v_1+\lambda\*v_2-\lambda\*v_3&=\lambda(v_1+v_2-v_3)\\
			&=\lambda\*0\\
			&=0
		\end{align*}
		Also ist $\lambda\*v\in U_1$
		\item[(2)] Seien $v=\vec{v_1\\v_2\\v_3},\quad w=\vec{w_1\\w_2\\w_3}\in U_1$, d.h. $v_1+v_2-v_3=0,\quad w_1+w_2-w_3=0$. Gilt $v+w\in U_1$?  $v+w=\vec{v_1+w_1\\v_2+w_2\\v_3+w_3}$
		\begin{align*}
			(v_1+w_1)+(v_2+w_2)-(v_3+w_3)&=\underbrace{(v_1+v_2-v_3)}_{=0}+\underbrace{(w_1+w_2-w_3)}_{=0}\\
			&=0
		\end{align*}
		Also $v+w\in U_1$
		\item Geometrische Interpretation:\\ \begin{align*}
		U_1&=\{\vec{x_1\\x_2\\x_1+x_2}|x_1,\quad x_2\in\R\}\\
		&=\{x_1\*\vec{1\\0\\1}+x_2\*\vec{0\\1\\1}|x_1,\quad x_2\in\R\}
		\end{align*}
		D.h. $U_1$ ist die Ebene durch $O=\vec{0\\0\\0}$ mit den Richtungsvektoren $\vec{1\\0\\1}$ und $\vec{0\\1\\1}$
	\end{itemize}
	\item[c)] $U_2=\{\vec{x_1\\x_2\\x_3}\in\R^3|x_1+x_2-x_3=1\}$ ist kein Unterraum. Z.B. $\vec{0\\0\\0}=\O\notin U_2$: $0+0-0=0\neq1$.\\
	Anderes Argument: Sei $\lambda\in\R,\quad x=\vec{x_1\\x_2\\x_3}\in U_2$, d.h. $x_1+x_2-x_3=1$. Gilt $\lambda\*x\in U_2$? $\lambda\*x=\vec{\lambda x_1\\\lambda x_2\\\lambda x_3}$
	\begin{align*}
		\lambda x_1+\lambda x_2-\lambda x_3&=\lambda\underbrace{(x_1+x_2-x_3)}_{=1}\\
		&=\underbrace{\lambda=1}_{\textrm{nur für }\lambda=1}
	\end{align*}
	$\Rightarrow$ nicht erfüllt für $\lambda\neq1$.\\
	Geometrische Interpretation:\\
	\begin{align*}
		U_2&=\{\vec{x_1\\x_2\\x_1+x_2-1}|x_1,\quad x_2\in\R\}\\
		&=\{\vec{0\\0\\-1}+x_1\*\vec{1\\0\\1}+x_2\*\vec{0\\1\\1}|x_1,\quad x_2\in\R\}
	\end{align*}
	Ebene durch $\vec{0\\0\\-1}$ mit Richtungsvektoren $\vec{1\\0\\1}$ und $\vec{0\\1\\1}$
	\item[d)] $U_3=\{\vec{x_1\\x_2\\x_3}\in\R^3|x_1^2+x_2^2+x_3^2\leq1\}$ ist kein Unterraum, z.B.\\
	$\vec{1\\0\\0}\in U_3,\qquad1^2+0^2+^2\leq1\quad\checkmark$, aber\\
	$2\*\vec{1\\0\\0}=\vec{2\\0\\0}\notin U_3$, denn $2^2+0^2+0^2\nleq1$\\
	Geometrische Interpretation:\\
	$U_3$ ist eine Kugel um $\vec{0\\0\\0}$ mit Radius 1
	\item[e)] $I\subseteq\R$ Intervall\\Menge $C(I)$ ($C$: continuous, stetig) der stetigen Funktionen auf $I$ ist Unterraum von $\mathcal{F}(I,\R)$ (vgl. Beispiel 8.2d)).\\
	Menge der diffbaren Funktionen auf $I$ ist Unterraum von $C(I)$.
	\end{itemize}
	\subsection{Satz}
	$V$ ist ein $\R$.Vektorraum, $U_1,U_2$ sind Unterräume von $V$.
	\begin{itemize}
		\item[a)] $U_1\cap U_2=\{u\in V|u\in U_1\wedge u\in U_2\}$ ist Unterraum von $V$.
		\item[b)] $U_1+U_2\coloneqq\{u_1+u_2|u_1\in U_1\wedge u_2\in U_2\}$ \underline{Summe} von $U_1,U_2$ ist Unterraum von $V$\\
	(das ist nicht die Vereinigung $U_1\cap U_2$!)
	\end{itemize}
	\subsubsection*{Beweis}
	Prüfe Unterraumkriterium 8.6
	\begin{itemize}
		\item[a)] Übung: Prüfe $\O\in U_1\cap U_2$? $\checkmark$, (1), (2)
		\item[b)] \begin{itemize}
			\item $U_1+U_2\neq\emptyset$, denn $U_1+U_2\ni\O=\underbrace{\O}_{\in U_1}+\underbrace{\O}_{\in U_2}$
			\item Seien $v=u_1+u_2, \quad u_1\in U_1,\quad u_2\in U_2$ und\\
			$w=u_1^\prime+u_2^\prime,\quad u_1^\prime\in U_1,\quad u_2^\prime\in U_2$,\\
			also $v,w\in U_1+U_2$ und $\lambda,\mu\in\R$.\\
			\begin{align*}
				\Rightarrow\qquad\lambda v+\mu v&=\lambda(u_1+u_2)+\mu(u_1^\prime+u_2^\prime)\\
				&=\underbrace{\lambda u_1+\mu u_1^\prime}_{\in U_1}+\underbrace{\lambda u_2+\mu u_2^\prime}_{\in U_2}
				&\in U_1+U_2
			\end{align*}
		\end{itemize}
	\end{itemize}
	\subsection{Bemerkung}
	\begin{itemize}
		\item[a)] lässt sich für unendlich viele Unterräume ausweiten
		\item[b)] lässt sich für endlich viele Unterräume ausweiten
		\item $U_1\cup U_2$ ist im Allgemeinen \underline{kein} Unterraum
	\end{itemize}
	\subsection{Beispiel}
	\begin{itemize}
		\item $v=\vec{1\\0}\in\R^2\qquad\qquad G_1=\{\lambda v|\lambda\in\R\}$
		\item $w=\vec{2\\1}\in\R^2\qquad\qquad G_2=\{\mu w|\mu\in\R\}$
	\end{itemize}
	(vgl. 8.7a), Geraden durch $\vec{0\\0}$, Unterräume
	\begin{itemize}
		\item $G_1+G_2$ ist Ebene
		\item $G_1\cap G_2$ ist $\O=\vec{0\\0}$
	\end{itemize}
\end{document}
